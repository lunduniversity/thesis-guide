\section{Litteraturstudier}\label{litteraturstudier}

\section{Varför litteraturstudier?}\label{varfuxf6r-litteraturstudier}

Litteraturstudier är en viktig del av ett examensarbete. Först och
främst är det en grundbult i god vetenskaplig metodik, som beskrivs i
kapitel 2. Väl genomförda litteraturstudier stödjer målet att bygga
vidare på befintlig kunskap och minskar risken att förbise redan gjorda
lärdomar. Genom att redovisa relevanta källor öppet blir det lättare för
en oberoende granskare att förstå utgångspunkterna och för avnämare (t
ex andra forskare, näringsliv och organisationer) att använda och bygga
vidare på resultaten. Relaterat arbete med diskussion kring källor finns
ofta i ett bakgrundsavsnitt eller teorikapitel av rapporten.

En väl genomförd analys av kunskapsfronten inom ett visst område är ett
viktigt bidrag i sig. Ofta finns det många olika relaterade studier med
varierande metoder, förutsättningar och resultat. Att bilda sig en
helhetsuppfattning om nuläget i kunskapsbildningen är ytterst värdefullt
som utgångspunkt för examensarbetet. En litteraturstudie kan utgöra hela
examensarbetet, även om detta är vanligare inom andra ämnesområden än de
tekniska.

Att genomföra grundliga litteraturstudier utgör ett omfattande arbete
som kräver noggrann planering och djup ämneskunskap. Detta kapitel
behandlar följande viktiga frågor:

\begin{itemize}
\item
  Hur hittar man relevant litteratur?
\item
  Vilken trovärdighet har olika källor?
\item
  Hur sammanställs resultaten av litteraturstudierna?
\end{itemize}

Litteraturstudier är en iterativ process där olika aktiviteter varvas,
såsom nyckelordsbestämning, sökning, urval, bedömning, och
sammanställning. I början av examensarbetet är litteraturen viktig för
att lära sig mer om ämnet. När tydliga delproblem och avgränsade
frågeställningar klarnat kan ytterligare studier av litteratur ske med
mer specifikt fokus. När resultat föreligger är det lämpligt att
återvända till litteraturen och jämföra med vad andra kommit fram till.
På detta sätt blir iterativa litteraturstudier en integrerad del i
examensarbetet. Det blir också ett sätt att lära sig behärska ett
viktigt verktyg för lärande och utveckling som den blivande ingenjören
har nytta av genom hela yrkeslivet.

\section{Att värdera källor}\label{att-vuxe4rdera-kuxe4llor}

Olika källor har olika trovärdighet. Vissa utsagor är baserade på
enskilda personers tyckande eller subjektiva erfarenheter. Vissa
resultat har genomgått omfattande vetenskaplig granskning. Annan
information kan vara irrelevant för sammanhanget, direkt vilseledande
eller till och med rena falsarier. Att ifrågasätta olika källors
trovärdighet och relevans är centralt i alla undersökande och
problemlösande arbeten som görs på ett vetenskapligt sätt. För varje
källa bör man fråga sig:

\begin{itemize}
\item
  Är materialet granskat och i så fall hur och av vem?
\item
  Vem står som garant för trovärdigheten?
\item
  Är undersökningsmetodiken trovärdig?
\item
  Är resultaten framtagna i ett sammanhang som är relevant för mina
  frågeställningar?
\item
  Har resultaten blivit bekräftade eller lett till erkännande och blivit
  refererade i andra trovärdiga sammanhang?
\end{itemize}

När resultatens trovärdighet och relevans kan bedömas blir de lättare
att relatera till och bygga vidare på. Även osäkra resultat, agiterande
ställningstaganden och preliminära indikationer kan vara värdefulla,
bara man är medveten om begränsningarna. Det är också oerhört graverande
om ett examensarbete skulle visa sig bygga på falsifikat eller plagiat
och detta undviks bäst genom en systematisk och öppet redovisad
litteraturstudie baserad på trovärdiga källor i vetenskaplig anda, där
god etik är ledstjärna.

\begin{enumerate}
\def\labelenumi{\arabic{enumi}.}
\item
  Den vetenskapliga granskningsprocessen
\end{enumerate}

För att kunna värdera vetenskapliga publikationer är det viktigt att
förstå hur den vetenskapliga granskningsprocessen (eng.
\emph{scientific} \emph{peer reviewing}) går till. Det finns en mängd
olika vetenskapliga forum som möjliggör publicering av
forskningsresultat av varierande mognadsgrad och kvalitet. Alla äkta
vetenskapliga forum har dock det gemensamt att publikation föregås av en
omfattande och gedigen granskningsprocess med syfte att garantera
publikationens höga vetenskapliga kvalitet.

Utvalda forskare som är ledande, erfarna och betrodda inom respektive
område samordnar granskningsprocessen. Dessa forskare agerar redaktörer
(eng. \emph{editor}) eller ordförande (eng. \emph{program chair}) i
programkommittéer (eng. \emph{program committee}) och tilldelar flera
(ofta minst tre) granskare till varje inskickat manuskript (eng.
\emph{submitted paper}). Granskarna är också framstående forskare med
specifik kompetens inom området som manuskriptet berör. Baserat på
granskarnas utlåtande fattas beslut om huruvida manuskriptet kan
publiceras som det är, om revideringar måste göras innan artikeln kan
accepteras, eller om publicering ska avslås. Granskarna är anonyma i
relation till författarna och ska agera helt oberoende och utan andra
hänsyn än rent vetenskapliga. Om jävsituationer skulle uppstå är
granskaren skyldig att avsäga sig granskningsuppdraget.

Redaktions- och programkommittéer definierar specifika kriterier som
granskarna ska utgå från i sin granskning. Dessa kriterier innefattar
ofta följande punkter (jämför kapitel 2.2.1 om vetenskaplighet):

\begin{itemize}
\item
  Ämnets och resultatens nyhetsvärde
\item
  Forskningsfrågeställningarnas relevans
\item
  Trovärdigheten och ändamålsenligheten i forskningsmetod
\item
  Referenslistans fullständighet och relevans
\item
  Presentationens kvalitet
\item
  Resultatens betydelse för vidare forskning och praktisk användning
\end{itemize}

Granskningen resulterar både i värderande bedömningar och konkreta
förbättringsåtgärder och manuskripten blir ofta väsentligt förbättrade
genom de oberoende granskarnas insats. Kvalitetssäkringen av
vetenskapliga resultat blir på detta sätt en ständigt pågående process
som alla forskare deltar i och som är en avgörande förutsättning för
vetenskaplig utveckling.

\begin{enumerate}
\def\labelenumi{\arabic{enumi}.}
\item
  Vetenskapligt granskade källor
\end{enumerate}

En vetenskaplig publikation kan vara av olika form och presenteras i
olika forum. En publikation kan t ex vara en artikel (eng. \emph{paper})
eller en sammanfattande presentation i bilder och text i plakatform
(eng. \emph{poster}). Publikationerna presenteras i och granskas inom
ramen för vetenskapliga forum som drivs utan kommersiella hänsyn.

Ofta ges den slutliga publikationen ut av ett förlag som samtidigt tar
ansvar för spridning och arkivering. Många förlag drivs av kommersiella
intressen och tar betalt för publikationerna. Förlagen är inte ansvariga
för granskningsprocessen, som genomförs av forskarna själva. I vanliga
fall tillåter copyrightavtalen att forskarna själva sprider sina egna
publikationer gratis i icke-kommersiellt syfte, t ex via universitetens
hemsidor eller efter direkt förfrågan. Därför kan man ofta få tillgång
till forskningsrapporter gratis genom att söka på författarnas hemsidor
eller kontakta författarna direkt.

Exempel på typiska vetenskapliga forum är:

\begin{itemize}
\item
  \emph{Tidskrifter} (eng. \emph{journal}): Tidskriftsartiklar är ofta
  längre än övriga artiklar och representerar mogna forskningsresultat.
  Granskningsprocessen för tidskriftsartiklar är normalt mer rigorös än
  för andra forum och manuskripten förbättras ofta i flera omgångar.
  Tidskrifter ges ut med jämna mellanrum med flera nummer årligen. Fokus
  och innehåll bestäms av en redaktionskommitté som leds av en
  chefredaktör.
\item
  \emph{Konferenser} (eng. \emph{conference}): På vetenskapliga
  konferenser träffas forskare och presenterar och diskuterar färska
  forskningsresultat inom ett väl avgränsat område. Presentationerna
  bygger på artiklar som genomgått den vetenskapliga
  granskningsprocessen och sammanställs i en konferensskrift (eng.
  \emph{proceedings}) och ges normalt ut i bokform. Det finns både
  internationella och lokala konferenser, där de internationella ofta
  håller högre kvalitet och har större påverkan på forskningsfronten.
  Konferensens innehåll bestäms av en programkommitté som leds av en
  eller flera ordförande.
\item
  \emph{Workshops}: I samband med konferenser arrangeras ofta olika
  workshops där de allra nyaste forskningsrönen diskuteras. Även på
  workshops baseras presentationerna normalt på artiklar som granskats,
  men fokus ligger mer på nyhetsvärde än på resultatmognad.
  Workshopartiklar innehåller ofta delresultat i ännu ej avslutade
  studier och oprövade resultat diskuteras. Workshops ger också ofta ut
  en skrift i bokform eller ibland enbart på webben. Det är vanligt att
  konferensartiklar eller tidskriftsartiklar bygger vidare på
  delresultat som tidigare publicerats på workshops.
\item
  \emph{Posters}: Inom vissa vetenskapsområden är det vanligt att
  konferenser ger möjlighet att presentera resultat i form av posters.
  På avsedd plats sätter forskarna upp plakat med illustrationer och
  kortfattad text som beskriver forskningsresultaten. Ofta innehåller
  konferensskriften korta sammanfattningar av posterpresentationen.
  Vanligtvis har posters granskats vetenskapligt på liknande sätt som
  övriga publikationer.
\item
  \emph{Kortare artiklar} (eng. \emph{short papers}, \emph{position
  papers, extended abstracts}): Ofta bjuder konferenser och workshops
  möjlighet att presentera forskning under andra mindre rigorösa former.
  Vissa workshops baserar acceptans på en mindre rigorös granskning av
  sammanfattningar (\emph{extended abstract}), där acceptansbeslutet
  sker innan den slutgiltiga artikeln är komplett. Ibland ges möjlighet
  att beskriva forskningsplaner eller forskningsmässiga
  ställningstaganden i diskussionsartiklar (\emph{position papers}).
  Ibland väljer man att ge möjlighet till artiklar som inte är
  tillräckligt bra för det ordinarie programmet men som ändå innehåller
  intressanta resultat genom att de publiceras som kortare artiklar
  (\emph{short papers}). Gemensamt för dessa kortare artiklar är att de,
  även om de genomgått viss vetenskaplig granskning, inte kan
  förutsättas ha samma höga kvalitetsnivå som artiklar på konferenser
  och i tidskrifter.
\end{itemize}

Källorna är presenterade i en ungefärlig fallande trovärdighets- och
kvalitetsordning, men detta är endast en generaliserad tumregel. I
enskilda fall kan mycket väl en konferens ha mer tyngd och bättre
vetenskapligt renommé än en tidskrift. Forumets trovärdighet bygger i
stor utsträckning på trovärdigheten hos de forskare som ansvarar för och
deltar i granskningsprocessen. Olika discipliner har dessutom olika
traditioner. Inom vissa ämnen har konferenser enbart posters medan andra
ämnen helt saknar denna tradition, medan många kombinerar både posters,
workshops och rapporter vid en gemensam konferens.

Ofta kan workshops vara extra intressanta då den allra senaste
forskningen ofta presenteras här först och det ibland tar åratal för
tidskriftspublikationer att från inskickat manuskript, via eventuell
revidering nå slutlig publicering. Det är vanligt att man efter en
konferens eller workshop väljer ut de allra bästa artiklarna och bjuder
in författarna att utöka rapporten för publicering i en tidskrift som
valt att dedicera ett specialnummer åt en viss konferens eller workshop.
Normalt sker då ånyo en vetenskaplig granskning för att säkerställa
kvaliteten.

\begin{enumerate}
\def\labelenumi{\arabic{enumi}.}
\item
  Andra källor i vetenskapliga sammanhang
\end{enumerate}

Ofta stöter man på andra rapporter i vetenskapliga sammanhang som inte
är vetenskapligt granskade på samma sätt som tidigare nämnda källor, men
som ändå ofta har hög trovärdighet. Exempel kan vara:

\begin{itemize}
\item
  \emph{Inbjudna artiklar} (eng. \emph{invited paper}): Etablerade
  forskare blir ibland inbjudna att skriva en artikel om en specifik
  frågeställning. Här ges möjlighet att t ex fritt sammanfatta området
  och ge en personlig syn på forskningsfronten av en erkänt god
  forskare.
\item
  \emph{Inbjudna talare} (eng. \emph{key note}): Intressanta talare
  bjuds ofta in till vetenskapliga konferenser och en sammanfattning av
  deras presentation finns ibland i proceedings. Inbjudna talare är inte
  alltid forskare, de kan också vara företagsledare, politiker eller
  andra intressanta personligheter inom området.
\item
  \emph{Paneldebatter} (eng. \emph{panel}): Ett vanligt inslag vid
  konferenser är paneldebatten där experter med olika ståndpunkter
  diskuterar en frågeställning. En debattsammanfattning finns ibland i
  proceedings.
\item
  \emph{Erfarenhetsartiklar} (eng. \emph{experience papers}): I
  tillämpade forskningsområden är det vanligt att praktiskt verksamma
  personer inom näringsliv och organisationer bjuds in att bidra med
  erfarenheter och problem som kan ligga till grund för vidare
  forskning. Dessa artiklar skrivs normalt av icke-forskare och har
  därför inte en vetenskaplig metodik. Dessa granskas separat utifrån
  deras värde som indata från omvärlden för att ge praktisk relevans i
  forskningen.
\item
  \emph{Examensarbetesrapporter:} Många examensarbeten har en hög
  kvalitet och vetenskaplig relevans. De är också normalt granskade av
  vetenskapligt verksamma lärare. Ofta kan examensarbeten ligga till
  grund för framtida forskning och det är inte ovanligt att det går att
  bygga vidare på resultat från examensarbeten och publicera
  vetenskapliga artiklar baserat på resultaten.
\item
  \emph{Uppslagsverk} (eng. \emph{encyclopedia}): Vissa uppslagsverk är
  sammanställda genom att välrenommerade forskare att bjuds in av en
  redaktionskommitté att skriva artiklar under utvalda uppslagsord.
  Artiklarna i sådana verk granskas normalt noga och har hög
  trovärdighet.
\item
  \emph{Antologier}: Ofta ger vetenskapliga förlag ut antologier där
  framstående forskare får agera redaktör för en bok i ett visst område.
  Redaktören bjuder in andra forskare att skriva olika kapitel i boken,
  som blir en redigerad sammanläggning av olika författares skrifter.
  Normalt granskas artiklarna inom författargruppen. Vem som skrivit
  varje kapitel framgår tydligt.
\end{itemize}

Om källan, trots sitt vetenskapliga sammanhang, inte genomgått den
fullständiga vetenskapliga granskningsprocessen, så är det ännu
viktigare att bilda sig en egen uppfattning om källans trovärdighet. Det
gäller att ställa sig frågor i relation till hur materialet har tagits
fram och av vem. Att undersöka källans trovärdighet är alltså viktigt
även om den är framtagen i ett vetenskapligt sammanhang.

\begin{enumerate}
\def\labelenumi{\arabic{enumi}.}
\item
  Övriga källor
\end{enumerate}

Det finns självklart många källor som inte är vetenskapliga men som ändå
kan vara värdefulla som bas för examensarbeten. Exempel kan vara:

\begin{itemize}
\item
  Läroböcker
\item
  Journalistiskt material
\item
  Industrikonferenser
\item
  Partsinlagor och ställningstaganden (eng. \emph{white paper})
\item
  Företagsintern information
\item
  Webbsidor
\end{itemize}

Här är det extra viktigt att reda ut källans trovärdighet, samt arbetets
kvalitet och relevans för sammanhanget. Ofta kan partsinlagor verka nog
så vederhäftga, men kommer i ett annat ljus om man funderar på syftet
bakom inlagan. Kommersiella, politiska eller andra hänsynstaganden kan
göra att kritiska synvinklar, brister eller viktiga begränsningar
utelämnats. Journalistiskt material är ofta framtaget under tidspress.

Ogranskade webbsidor är riskfyllda som grund för examensarbetet.
Internet är ett fantastiskt verktyg och här kan man snabbt och enkelt
hitta många vetenskapliga rapporter och annan trovärdig information. Men
det finns också mycket skräp och tyvärr är det vanligt med direkt
vilseledande information på internet. Sökmotorer hjälper till att söka
brett men inte att värdera informationens kvalitet. Läsaren måste
självständigt bilda sig en uppfattning om kvaliteten, vilket inte är
lätt då webbsidor ofta är utformade just för att ge ett trovärdigt
intryck.

Öppna uppslagsverk på webben (t ex Wikipedia) innehåller mängder av
aktuella termer som ännu inte nått de etablerade uppslagsverken, men det
är svårt att veta vad som går att lita på. Om du hittar något på en
hemsida som inte är vetenskapligt granskat eller som inte är från en
trovärdig källa bör du antagligen inte använda denna källa, utan i
stället söka vidare efter säkrare information. Om du ändå behöver
använda dig av material som bara finns på internet och inte är granskat
av oberoende experter så bör du tydligt ange detta och referera på ett
speciellt sätt med datum för när du besökte länken, så att läsaren av
din rapport ser att detta är en osäker referens och själv kan ta
ställning till trovärdigheten.

\section{Att söka litteratur}\label{att-suxf6ka-litteratur}

Att genomföra litteratursökning är ett omfattande detektivarbete i flera
steg. För att veta vad man ska söka efter behöver man förstå området och
känna till terminologin. Ofta sker sökprocessen i flera omgångar där den
ena källan leder till den andra. En typisk litteratursökning kan
innehålla följande övergripande steg, där de senare stegen ofta
upprepas:

\begin{itemize}
\item
  \emph{Sök brett:} I början av litteratursökningen är det bra att söka
  brett och genom att använda många olika vägar och nyckelord. Exempel:
  Prata med handledaren, utgå från referenslistor i läroböcker, sök på
  internet och i artikeldatabaser, kontakta bibliotekarier. Det är också
  bra att ta med angränsande och relaterade ämnen i sökningen för att
  garantera bredden.
\item
  \emph{Välj ut:} Baserat på en översiktlig läsning av källorna sker
  normalt ett urval. De mest relevanta källorna studeras därefter
  djupare och den förståelse som då erhålls ligger till grund för
  fördjupad litteratursökning.
\item
  \emph{Sök djupt:} Att söka mer fokuserat efter speciellt relevanta
  källor görs baserat på terminologin inom området. Sökmotorer och
  artikeldatabaser är en god hjälp men det gäller att prova varianter,
  synonymer och relaterade termer. En bra utgångspunkt är att följa upp
  referenslistan i de artiklar som är särskilt relevanta. Ett annat sätt
  är att utgå från författare som visar sig särskilt aktiva inom området
  och söka på deras namn för att hitta artiklar som man kanske missat.

  Har man väl hittat en referens som man vill läsa gäller det att få tag
  i själva rapporten. Ofta har universitetsbiblioteken tecknat avtal med
  de stora forskningsförlagen för att ge forskare och studenter
  elektronisk tillgång till forskningsartiklar. Till de stora, breda,
  vinstdrivande forskningsförlagen inom teknik hör Springer/Kluwer,
  Elsevier och Wiley. Det finns också branchspecifika organisationer med
  egna, ofta icke vinstdrivande förlag, t ex IEEE och ACM. Om det är
  svårt att få tag i artikeln kan man, som tidigare nämnts, kontakta
  författaren direkt och be om en kopia av artikeln. Var då noga med att
  ange syftet med användningen av artikeln, då forskaren ofta endast har
  lov att distribuera artikeln i icke-kommersiellt syfte.

  Medan man letar behöver man fortlöpande också läsa de artiklar man
  hittar för att få nya ledtrådar till var man kan leta vidare. Efter
  ett tag får man bra träning i att läsa och bilda sig en uppfattning om
  en artikel. En tränad artikelläsare läser sällan från början till
  slut, utan genomför läsningen i olika omgångar. Det kan exempelvis gå
  till så här: Först studeras titel och sammanfattning. Om artikeln
  verkar intressant läses inledningen och slutsatserna för att skapa en
  förståelse för målsättningar och resultat. Därefter kanske man
  fortsätter med översiktlig läsning av metodavsnitt och analysdelar. Om
  vissa delar är särskilt relevanta läses dessa på djupet. Medan man
  läser är det bra att markera och anteckna intressanta iakttagelser.

  \begin{enumerate}
  \def\labelenumi{\arabic{enumi}.}
  \item ~
    \section{Litteratursammanställning}\label{litteratursammanstuxe4llning}
  \end{enumerate}
\end{itemize}

Det är mycket viktigt att dokumentera litteraturstudiearbetet. Mängden
nyckelord, författare och artiklar blir snabbt stort och för att hålla
reda på allt behövs listor med sammanfattningar och grupperingar i
övergripande kategorier och undergrupper. Vilka kriterier som använts
vid urvalet är också viktigt att dokumentera. Denna information behövs
vid författandet av litteratursammanställningen i rapporten och vid
upprättandet av referenslistan. Nedan följer exempel på vad en
dokumentation av litteratursökningen kan innehålla:

\begin{itemize}
\item
  Författare
\item
  Titel
\item
  Källa, forum, förlag, konferensnamn etc.
\item
  Sidhänvisning
\item
  Datum för publikation
\item
  Nyckelord, kategorisering
\item
  Omdöme om relevans, metodkvalitet, etc.
\item
  Resultatsammanfattning
\item
  Relation till övriga artiklar i sammanställningen
\item
  Datum för när källan hittades och värderades
\item
  Andra karaktäristika som är relevanta för arbetet, t.ex. storlek på
  population, mätteknik, etc.
\end{itemize}

Ofta skrivs en litteratursammanfattning i ett teoriavsnitt som på ett
strukturerat och uttömmande sätt beskriver befintligt arbete inom
området med utgångspunkt från litteratursökningen. (Se vidare kapitel 7
ang. rapportstruktur.) Dock finns det anledning att genom hela rapporten
göra källhänvisningar. Följande exempel visar hur källor ur
litteratursammanställningen bör användas i rapporten:

\begin{itemize}
\item
  \emph{Motiverande referens}: Kan t ex röra sig om att hitta belägg för
  att frågeställningar är viktiga eller att undersökningar behöver göras
  för att komplettera nuvarande kunskapsläge. Det sker ofta i samband
  med beskrivningar av målsättningar, frågeställningar eller problem.
\item
  \emph{Relaterande referens}: Kan t ex röra sig om att visa hur arbetet
  bygger vidare på befintligt arbete, vilket ofta sker i
  bakgrundsavsnitt och teorikapitel (även kallat relaterat arbete).
  Ibland görs relaterande referenser för att tydliggöra val av
  utgångspunkt och särskilja arbetets angreppssätt från andra studier.
\item
  \emph{Jämförande referens}. Avser källhänvisning för att jämföra de
  egna metoderna eller resultaten med andras och reda ut diskrepanser
  och samstämmigheter. Detta görs ofta i ett diskussionsavsnitt eller i
  ett resultatkapitel.
\item
  \emph{Systematiska litteraturstudier}. Litteraturstudier kan efter
  analys och syntes ge en samlad helhetsbedömning som är ett värdefullt
  bidrag i sig. Kanske ämnet inte tidigare är sammanställt på ett
  systematiskt sätt ur denna synvinkel. Exempel på hur systematiska
  litteraturstudier kan utgöra viktig forskning i sig beskrivs av
  Kitchenham (2004). En väl genomförd och omfattande litteraturstudie
  kan utgöra hela examensarbetet eller bli en egen separat del av
  examensarbetet med både metod, analys och resultatkapitel.

  \begin{enumerate}
  \def\labelenumi{\arabic{enumi}.}
  \item ~
    \section{Sammanfattning}\label{sammanfattning}
  \end{enumerate}
\end{itemize}

Litteraturstudierna lägger grunden för den ämnesfördjupning som krävs
för att genomföra examensarbetet. Litteratursökning är ett
detektivarbete som innebär att leta i databaser och i ämnesspecifika
forum t ex vetenskapliga publikationer av olika slag.
Litteratursökningen sker iterativt och ofta i flera omgångar under
examensarbetets gång. Genom att följa referenser vidare och genom att
lära sig terminologin inom området kan sökningen bli mer och mer
förfinad. Centralt för litteraturstudiearbetet är förmågan att värdera
källans trovärdighet och innehållets relevans. Detta är inte enkelt och
kräver djup förståelse för ämnet och erfarenhet av att läsa fackartiklar
och bedöma vetenskaplighet. Resultatet av litteraturstudier dokumenteras
i rapporten och litteratursammanställningen ingår i rapportens
referenslista på ett sådant sätt att läsaren kan bedöma omfattningen och
vid behov söka upp referenser och ta del av underlaget.
