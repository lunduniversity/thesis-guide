\section{Verktyg för genomförande}\label{verktyg-fuxf6r-genomfuxf6rande}

När man har valt den övergripande metodiken för sitt examensarbete och
ska gå till det praktiska genomförandet, behöver man en ''verktygslåda''
av olika metoder. Beroende på ämnesområde och metodik behövs
naturligtvis olika verktyg som är anpassade till situationen. I detta
kapitel ges en översikt över en del viktiga verktyg, samt hänvisningar
till mer fördjupade beskrivningar. Verktygen är grupperade i följande
kategorier:

\begin{itemize}
\item
  \emph{Utvärdering:} Hur kan utvärderingar genomföras? Hur kan
  utvärderingar användas som grund för beslutsfattande?
\item
  \emph{Kvalitetsdimensioner:} Vilka olika egenskaper kan utvärderas för
  olika typer av utvärderingsobjekt?
\item
  \emph{Datainsamling:} Hur kan insamling av data ske? Hur kan mätningar
  göras?
\item
  \emph{Prototyputveckling:} Vilken roll kan prototyper spela?
\item
  \emph{Modellering:} Hur bygger man en modell av ett fenomen, t~ex för
  simulering och hur validerar man modellens beteende?
\item
  \emph{Analys:} Hur kan insamlad data analyseras kvantitativt och
  kvalitativt?
\item
  \emph{Resultatvalidering:} Hur kan resultatens giltighet bedömas?

  \begin{enumerate}
  \def\labelenumi{\arabic{enumi}.}
  \item ~
    \section{Utvärdering}\label{utvuxe4rdering}
  \end{enumerate}
\end{itemize}

En viktig del av de övergripande utbildningsmålen med examensarbetet är,
som beskrivits i kapitel 2, att lära sig hur en utvärdering av det egna
bidraget genomförs på ett vetenskapligt sätt. Utvärderingen syftar till
att bedöma det egna bidragets kvalitet, utifrån de målsättningar som
definierats som utgångspunkt för examensarbetet. Vad som utgör det egna
bidragets kvalitet och hur denna ska utvärderas beror helt på
situationen. De konkreta kriterier och förutsättningar som gäller för
utvärderingen avgörs därmed från fall till fall. Metodiken för
utvärderingen bör passa in i den övergripande metodiken för
examensarbetet, som beskrivs i kapitel 3. En utvärdering på vetenskaplig
grund innebär bland annat att utvärderingen bör:

\begin{itemize}
\item
  ha en i förväg upprättad \emph{utvärderingsplan} där detaljerade steg
  i utvärderingen identifieras och kriterier och förutsättningar
  definieras i enlighet med examensarbetets övergripande metodik,
\item
  vara \emph{förutsättningslös} \emph{och objektiv}, i meningen att
  utvärderingen inte är styrd av förutfattade meningar om önskvärda
  resultat och ej har syftet att försköna eller svartmåla verkligheten,
\item
  innefatta en heltäckande och ärlig \emph{redovisning} av både
  genomförande och utvärderingsresultat (såväl positiva som negativa),
  som möjliggör oberoende värdering så att utomstående själva kan
  granska och bedöma trovärdigheten.

  \begin{enumerate}
  \def\labelenumi{\arabic{enumi}.}
  \item
    Utvärderingsprocessen
  \end{enumerate}
\end{itemize}

Om examensarbetet är av implementerande karaktär, där examensarbetarna t
ex tar fram en prototyp eller vidareutvecklar en befintlig produkt, är
utvärdering av specifika kvalitetsegenskaper ofta centrala. Exempelvis
kan ett examensarbete handla om att lägga till en ny funktion i ett
existerande system och utvärdera hur denna funktion påverkar systemets
totala prestanda. I detta exempel blir mätningar av prestanda genom
modellering och simulering central för utvärderingen.

Om examensarbetet är av undersökande eller utredande karaktär behöver
slutsatsernas trovärdighet och giltighet (validitet) utvärderas, och
därmed avgöra om de duger som t.ex. beslutsunderlag eller grund för
vidare undersökningar. Denna typ av utvärdering baseras ofta på en
dialog med intressenter som berörs eller drar nytta av resultaten eller
med andra personer som kan bidra med en resultatbedömning. Här är
intervjuer och enkäter värdefulla verktyg. Exempelvis kan resultaten
från en utredning om en arbetsprocess redovisas för beslutfattare som
ansvarar för den undersökta processen och frågor ställas genom
intervjuer eller enkäter angående resultatens rimlighet och möjligheten
att använda utredningsresultaten i arbetet med att effektivisera och
vidareutveckla processen.

Med diskussionerna om god vetenskaplig metodik i kapitel 2 i minnet, i
linje med vald metodik enligt kapitel 3, och i beaktande av
verktygslådan av metoder som presenteras i detta kapitel, så är det
lämpligt att genomförandet av en utvärdering beaktar följande steg och
frågeställningar:

\begin{itemize}
\item
  \emph{Karaktärisering av utvärderingsobjekt}: Vad är det som ska
  utvärderas? Vilka kvalitetsdimensioner har utvärderingsobjektet?
\item
  \emph{Definition av utvärderingssyfte:} Varför gör vi utvärderingen?
  Vilka kvalitetsdimensioner är viktiga? Vilka frågor vill vi ha svar på
  i utvärderingen?
\item
  \emph{Val av utvärderingsmetod:} Hur ska vi genomföra utvärderingen?
  Hur genomförs datainsamlingen? Hur ska vi mäta på
  utvärderingsobjektet?
\item
  \emph{Genomförande av utvärderingen:} Detta steg innebär att vi
  genomför de praktiska moment som definieras av de övriga
  utvärderingsstegen. Data från utvärderingen sparas på lämpligt sätt.
\item
  \emph{Analys av resultaten:} Hur ska vi kvantitativt och/eller
  kvalitativt analysera resultaten? Vilka statistiska metoder kan vi
  använda?
\item
  \emph{Validering av resultaten:} Vilka begränsningar i resultatens
  giltighet finns? Vilka är hoten mot validiteten? Hur kan vi minska
  eller helt undvika validitetshot?
\item
  \emph{Presentation av resultat:} Hur visualiserar vi resultaten? Vilka
  slutsatser kan vi dra? Hur kan resultaten presenteras så att de blir
  en bra grund för beslutsfattande?
\end{itemize}

Valet av utvärderingsmetod styrs av utvärderingsobjektets karaktäristika
och av utvärderingens syfte. Här kan, beroende på sammanhanget, en eller
flera metoder ur detta kapitel vara användbara. Man kan till exempel med
fördel kombinera kvantitativa mätningar av specifika egenskaper hos en
prototyp med kvalitativa intervjuer med noggrant utvalda pilotanvändare.

\begin{enumerate}
\def\labelenumi{\arabic{enumi}.}
\item
  Utvärdering som beslutsunderlag
\end{enumerate}

Ofta utgör utvärderingar underlag för beslutsfattande. Ett examensarbete
kan till exempel som delresultat ge rekommendationer om vilket av flera
alternativ som är bäst enligt angivna förutsättningar baserat på
utvärderingsresultat av respektive alternativ.

När man ska fatta beslut är det ofta avgörande att på ett relevant sätt
kunna väga samman olika kvalitetsdimensioner och egenskaper. Sådana
avvägningar kan vara mycket svåra och kräver ofta ingående kunskap om
det specifika problemet. Dessutom finns det ofta komplexa beroende
mellan olika egenskaper. Till exempel så kan en produkts \emph{säkerhet}
påverkas negativt om det finns dolda fel i produkten, vilket i sin tur
har med \emph{tillförlitlighet} att göra. Ett annat exempel är att
\emph{användbarheten} av en produkt kan påverkas negativt av
säkerhetsåtgärder såsom lås eller inloggningsfunktioner.

Beslutsunderlag bör åskådliggöra det man ska fatta beslut om och på ett
lättöverskådligt sätt visualisera informationen som ligger till grund
för beslutet. Ett sätt att presentera utvärderingsresultat som grund för
beslut är att ställa upp \emph{kriterietabeller}. Figur 6.1 visar ett
exempel på hur en kriterietabell kan konstrueras.

Raderna i exemplet på kriterietabell innehåller de kriterier som används
i utvärderingen, t ex baserat på några väl valda kvalitetsdimensioner
som anses viktiga i just detta sammanhang. Kolumnerna innehåller de
olika alternativ som utvärderats med avseende på kriterierna. I
tabellelementen finns indikatorer som med ett mätetal eller på annat
sätt karaktäriserar ett visst alternativ med avseende på ett visst
kriterium.

Vilka kriterier man ska använda är helt avgörande för utfallet. Vissa
kriterier är lättare att mäta än andra. Ibland låter man subjektiva
bedömningar ingå i bedömningen av kriterierna. Mätskalor som kan
användas när kriterier ska ges värden behandlas i kapitel 6.3.5.

Tabell 6.1. Kriterietabell

\begin{longtable}[]{@{}llll@{}}
\toprule
\begin{minipage}[t]{0.24\columnwidth}\raggedright\strut
\textbf{Kvalitetskriterier för\\
musikspelare}\strut
\end{minipage} & \begin{minipage}[t]{0.24\columnwidth}\raggedright\strut
\emph{Alternativ 1}:

SuperPlayer\strut
\end{minipage} & \begin{minipage}[t]{0.24\columnwidth}\raggedright\strut
\emph{Alternativ 2}:

MyPlayer\strut
\end{minipage} & \begin{minipage}[t]{0.24\columnwidth}\raggedright\strut
\emph{Alternativ 3}:

MiniPlayer\strut
\end{minipage}\tabularnewline
\emph{Stöd för filformat} & Bäst & Stödjer alla viktiga format & Saknar
många viktiga format\tabularnewline
\emph{Överföringshastighet {[}kbps{]}} & 100 & 50 & 1000\tabularnewline
\emph{Utseende} & Elegant & Plastig & Diskret\tabularnewline
\emph{Ljudkvalitet} & 4 & 2 & 5\tabularnewline
\emph{Storlek {[}bxhxd, mm{]}} & 50x100x10 & 50x10x50 &
5x10x100\tabularnewline
\emph{Pris {[}EUR{]}} & 999 & 99 & 1500\tabularnewline
\bottomrule
\end{longtable}

Inom områdena beslutsteori och beslutsstödsystem finns en uppsjö olika
metoder och angreppssätt för att sammanställa beslutsunderlag (Edlund et
al. 1999). Ett typiskt exempel på en metod för att prioritera bland
många olika objekt är \emph{Analytical Hierarchy Process} (Saaty 1980),
där parvisa jämförelser och matrisberäkningar används för att ta fram
relativa prioriteter på en ratio-skala. Enklare metoder på samma tema
utgörs av rankning genom olika former av sortering eller genom att
kategorisera jämförelseobjekten, till exempel genom att beskriva varje
objekt på vars ett kort och låta olika intressenter dela upp korten i
olika högar efter givna eller egendefinierade kriterier (eng. \emph{card
sorting}).

\section{Kvalitetsdimensioner }\label{kvalitetsdimensioner}

Kvalitet är ett mångfacetterat begrepp med många olika tolkningar som
beror av sammanhanget. Olika intressenter kan uppfatta kvalitet på olika
sätt och när ett objekt ska utvärderas bör man noga tänka igenom vilka
kvalitetsdimensioner som är relevanta och för vilka avnämare (kunder,
marknadssegment, användare, sponsorer, beslutsfattare etc). Vad som är
relevanta kvalitetsdimensioner beror också allmänt på om
utvärderingsobjektet är en fysisk produkt, en tjänst, en programvara, en
utvecklingsmetod eller en tillverkningsprocess.

Tekniska system är ofta en komplex sammansättning av olika delar som
samverkar. En produkt kan bestå av både elektronik, mekanik och
programvara. Produktionen av produkten kanske kräver en komplex
tillverkningsprocess som i sin tur kan innehålla flera komplexa tekniska
system. Runt produkten kanske det finns ett tjänsteutbud med ett
specialiserat IT-stöd och en supportorganisation med telefonisystem och
webbstöd. Försäljning och distribution av produkten kanske också sker
med stöd av tekniska system.

På detta sätt ingår högteknologiska produkter ofta i en komplicerad
struktur och att reda ut vilka kvalitetsdimensioner som är viktiga ur
olika aspekter kräver ofta att man tar hänsyn till samverkan mellan
flera system. I efterföljande delkapitel presenteras olika exempel på
kvalitetsdimensioner ur olika synvinklar. Först presenterar vi
dimensioner för systemkvalitet som baseras på standarden ISO9126 (2001).
Därefter ger vi exempel på dimensioner som kan vara relevanta för fysisk
produktkvalitet, tjänstekvalitet och processkvalitet.

\emph{Hur kan man ha användning för dessa kvalitetsdimensioner i ett
examensarbete?} I utvärderingsdelen av ett examensarbete måste man ta
ställning till vilka kriterier som utvärderingen ska ske utifrån. Vad
menas med ''bra'' och ''dåligt'' i det specifika sammanhanget? För att
svara på dessa frågor behöver man ofta göra en grundlig utredning av vad
god kvalitet innebär och detta medför i sin tur att kvalitetsdimensioner
behöver definieras, undersökas och mätas. Vilka kvalitetsdimensioner är
viktigast? Vilka mätetal karaktäriserar på ett giltigt sätt en viss
kvalitetsdimension? Med utgångspunkt i relevanta kvalitetsdimensioner
kan verktyg för datainsamling tillämpas, t ex GQM och mätskalor (se
vidare kapitel 6.3). Kanske behövs simuleringar för att undersöka hur
kvalitetsdimensioner samverkar? Då prototyputveckling sker (kapitel 6.4)
är en djuplodande undersökning av kvalitetsdimensioner avgörande i
planeringen och genomförandet. Analys (kapitel 6.5) och
resultatvalidering (6.6) är ofta beroende av och kopplade till mätningar
av kvalitetsdimensioner.

Användningen av kvalitetsdimensioner är sällan oproblematisk. Det är
svårt att hitta entydiga definitioner av de olika dimensionerna så att
de inte överlappar varandra. Olika personer tolkar begreppen på olika
vis och inom olika discipliner finns olika tradition i språkbruket. Det
är ofta svårt att mäta kvalitetsdimensionerna och även om man lyckas
definiera giltiga mått så är det inte alltid att ett sådant mått fångar
hela kvalitetsupplevelsen ur användarsynvinkel. Ofta krävs många olika
mått för att täcka olika aspekter av samma kvalitetsdimension. Olika
kvalitetsdimensioner hänger ofta ihop och påverkar varandra genom
komplexa inbördes beroenden både positivt och negativt. Till exempel kan
användarvänligheten påverkas negativt av säkerhetshöjande funktioner. En
snabbare produkt kanske innehåller komponenter som är mindre
tillförlitliga. Om prestanda ökas så blir också användbarheten bättre.
Förståelse för denna typ av komplexa beroenden kan betraktas som en
kärnkompetens för ingenjörer och förmåga att göra goda avvägningar
mellan kvalitetsdimensioner i en produktutvecklingssituation är en
central del i begreppet ingenjörsmässighet.

De listor med kvalitetsdimensioner som återges i efterföljande
delkapitel är tänkta att utgöra startpunkter för djupare analyser. Även
om de inte kan göra anspråk på att vara fullständiga används de
efterföljande listorna med fördel som checklistor och inspirationskällor
i arbetet med att ta fram specifika definitioner och mått på kvalitet.

\begin{enumerate}
\def\labelenumi{\arabic{enumi}.}
\item
  Systemkvalitet
\end{enumerate}

Den internationella standarden ISO9126 (2001) föreslår en
kvalitetsmodell ämnad att karaktärisera produktkvalitet för system som
helt eller delvis bygger på programvara, men många av modellens
kvalitetsdimensioner är tillämpbara allmänt på komplexa tekniska system
som integrerar fysiska komponenter, tjänster och processer.

Programvara är en speciell produkt, bland annat på grund av att
kostnaden för mångfaldigande är i princip noll. Vidare är programvara
svår att visualisera och följer inte fysikens lagar, exempelvis åldras
inte programvara. Det är kanske på grund av programvarans speciella
natur som man sett ett extra stort behov av att ta fram standardiserade
kvalitetsdimensioner med en omfattande detaljnivå speciellt för
programvara, för att på så sätt underlätta synliggörandet av kvaliteten
hos denna immateriella produkt. I takt med att allt fler
ingenjörsprodukter innehåller allt mer programvara blir
kvalitetsmodeller för programvara allt viktigare.

ISO9126-modellen är indelad i en hierarki med övergripande
kvalitetsaspekter med tillhörande underkategorier och mätetal. Vi
redovisar nedan modellens kvalitetsdimensioner i två nivåer (men
hänvisar till standarden för detaljer kring mätetal):

\textbf{Funktionalitet} (eng. \emph{functionality}). Hur är förmågan att
tillhandahålla funktioner som uppfyller uttalade och underförstådda
behov under specifika omständigheter? Denna dimension fokuserar på
\emph{vad} systemet gör för att uppfylla behov, medan övriga dimensioner
koncentrerar sig på \emph{hur} och i vilken utsträckning behov uppfylls.

\begin{itemize}
\item
  \emph{Lämplighet} (eng. \emph{suitability}): Vilken är förmågan att
  tillhandahålla en lämplig uppsättning funktioner för specifika
  användningsändamål?
\item
  \emph{Noggrannhet} (eng. \emph{accuracy}): Vilken är förmågan att
  tillhandahålla korrekta resultat med given precisionsgrad?
\item
  \emph{Samverkansförmåga} (eng. \emph{interoperability}): Hur väl
  samverkar delsystem med omgivande system?
\item
  \emph{(Informations)säkerhet} (eng. \emph{security}): Kan systemet
  skydda data så att endast auktoriserade personer kan läsa eller ändra
  den?
\end{itemize}

\textbf{Tillförlitlighet} (eng. \emph{reliability}). Hur god är förmågan
att upprätthålla en viss, specificerad (tillförlitlig, felfri,
tillgänglig) funktion under specifika omständigheter?

\begin{itemize}
\item
  \emph{Mognadsgrad} (eng. \emph{maturity}): Kan systemet undvika att
  felyttringar uppkommer till följd av latenta fel i systemet? Ett
  system med hög mognadsgrad har en lägre andel latenta fel och en lägre
  sannolikhet för att bete sig felaktigt. Latenta fel i programvara
  beror inte på åldrande utan på misstag i kravspecifikationer,
  designdokument eller programmering.
\item
  \emph{Feltolerans} (eng. fault \emph{tolerance}): Kan systemets
  funktion upprätthållas även om systemet påverkas av latenta fel eller
  otillbörliga intrång?
\item
  \emph{Återhämtningsförmåga} (eng. \emph{recoverability}): I vilken
  utsträckning kan systemet återhämta sig efter att felyttringar har
  uppkommit?
\end{itemize}

\textbf{Användbarhet} (eng. \emph{usability}). Hur lätt är systemet att
använda under specifika omständigheter för specifika användare?

\begin{itemize}
\item
  \emph{Begriplighet} (eng. \emph{understandability}): Förstår
  användarna vad systemet kan användas till och hur det används för en
  speciell uppgift under specifika omständigheter?
\item
  \emph{Lärbarhet} (eng. \emph{learnability}): Hur lätt är det för
  användarna att lära sig använda systemet?
\item
  \emph{Handhavande} (eng. \emph{operability}): Hur lätt är det för
  användarna att handha, kontrollera och styra systemet?
\item
  \emph{Attraktivitet} (eng. \emph{attractiveness}): Hur attraktivt
  upplevs systemet av användarna? Vilken är den subjektiva
  tillfredsställelsen i användandet?
\end{itemize}

\textbf{Effektivitet} (eng. \emph{efficiency}). Vilken är förmågan att
tillhandahålla ändamålsenlig prestanda i relation till mängden resurser
som krävs under givna omständigheter?

\begin{itemize}
\item
  \emph{Tidsbeteende} (eng. \emph{time behaviour}): Vilken är förmågan
  att ge snabb respons, korta beräkningstider och god
  genomströmningsgrad (eng. throughput rates)? Kallas ofta för snabbhet
  men även för prestanda (eng. speed, performance).
\item
  \emph{Resursutnyttjande} (eng. \emph{resource utilisation}): Vilken är
  förmågan att i ändamålsenligt utsträckning utnyttja tillgängliga
  resurser av olika typ (exempelvis primärminne, diskutrymme,
  bildskärmsminne etc.) Kallas ofta för kapacitet (eng. capacity).
\end{itemize}

\textbf{Underhållsbarhet} (eng. \emph{maintainability}). Hur lätt är det
att ändra systemet, exempelvis vid korrigeringar, anpassningar, eller
tillägg för att uppfylla nya krav?

\begin{itemize}
\item
  \emph{Analyserbarhet} (eng. \emph{analysability}): Hur lätt är det att
  diagnostisera bristfälligheter eller felorsaker? Hur lätt är det att
  identifiera vilka delar som behöver modifieras?
\item
  \emph{Ändringsbarhet} (eng. \emph{changeability}): Hur lätt är det att
  göra en specifik modifiering?
\item
  \emph{Stabilitet} (eng. \emph{stability}): Hur stor är risken att en
  modifiering orsakar oväntade effekter?
\item
  \emph{Testbarhet} (eng. \emph{testability}): Hur lätt är det att
  validera systemet? Hur lätt går det att avgöra om en modifiering
  uppfyller krav och förväntningar?
\end{itemize}

\textbf{Portabilitet} (eng. \emph{portability}). Hur lätt är det att
överföra systemet från en omgivning till en annan?

\begin{itemize}
\item
  \emph{Anpassningsbarhet} (eng. \emph{adaptability}): Hur lätt är det
  att anpassa systemet utan andra åtgärder än de som erbjuds av systemet
  självt? Hur lätt är det att utöka kapaciteten eller att skala upp
  systemet vid utbyggnad?
\item
  \emph{Installationsbarhet} (eng. \emph{installability}): Hur lätt är
  det att installera systemet i sin specifika omgivning?
\item
  \emph{Samexistens} (eng. \emph{co-existence}): Hur lätt är det för
  systemet att samverka med andra oberoende system i en gemensam
  omgivning under samutnyttjande av gemensamma resurser?
\item
  \emph{Utbytbarhet} (eng. \emph{replaceability}): Hur lätt är det att
  ersätta systemet med ett annat specifikt system i samma syfte i samma
  omgivning?
\end{itemize}

\textbf{Uppfyllandegrad} (eng. \emph{compliance}). För vart och ett av
kvalitetsdimensionerna ovan kan man ställa sig frågan: Hur uppfylls
denna kvalitetsdimension i enlighet med standarder, konventioner,
regleringar, lagar eller förordningar? Man talar då om t.ex.
\emph{reliability compliance} och avser därmed gällande praxis för
tillförlitlighetsregler i det specifika sammanhanget.

\textbf{Kvalitet i användande} (eng. \emph{quality in use}). Hur väl
stödjer systemet specifika användare i att uppnå specifika mål i
specifika användningssammanhang? Denna övergripande dimension berör
systemet som en komponent i ett större sammanhang där användare
samverkar med systemets olika delar i en helhet, t ex en integration av
delar som består av programvara, elektronik, eller mekanik.

\begin{itemize}
\item
  \emph{Verkningsfullhet} (eng. \emph{effectiveness}): Vilken är
  systemets förmåga att stödja användarna i uppfyllandet av specifika
  mål, på ett noggrant och fullständigt sätt, i det specifika
  användningssammanhanget?
\item
  \emph{Produktivitet} (eng. \emph{productivity}): Hur väl möjliggör
  systemet att användarna förbrukar en rimlig mängd resurser (tid,
  pengar), i relation till den uppnådda verkningsfullheten, i det
  specifika användningssammanhanget?
\item
  \emph{(Person)säkerhet} (eng. \emph{safety}): Hur är systemets förmåga
  att erbjuda en acceptabel säkerhetsnivå vad gäller risken att
  människor, verksamhet, system, egendom eller miljö kommer till skada,
  i det specifika användningssammanhanget?
\item
  \emph{Tillfredsställelse} (eng. \emph{satisfaction}): Hur god är
  förmågan att tillfredsställa användare i specificerade
  användningssammanhang?

  \begin{enumerate}
  \def\labelenumi{\arabic{enumi}.}
  \item
    Fysisk produktkvalitet
  \end{enumerate}
\end{itemize}

När en produkt eller del av produkt utgörs av fysiska komponenter, t ex
maskinelement eller elektronik, är kvaliteten beroende av komponenternas
fysikaliska egenskaper. Tillförlitligheten påverkas exempelvis av att
komponenter åldras och slits. Kvaliteten kan oftast karaktäriseras genom
fysikaliska mått, så som storlek, vikt, energi etc. Nedan följer exempel
på kvalitetsdimensioner som kan vara viktiga för en fysisk produkt
(Bergman och Klefsjö 2002):

\begin{itemize}
\item
  \emph{Driftssäkerhet:} Hur ofta inträffar fel? Hur alvarliga är felen?
\item
  \emph{Prestanda:} Vilken effekt och effektivitet har varan, i termer
  av t ex hastighet, livslängd, storlek?
\item
  \emph{Underhållsmässighet:} Hur lätt är det att upptäcka, lokalisera
  och avhjälpa fel?
\item
  \emph{Miljövänlighet:} Hur påverkar produkten miljön, exempelvis i
  form av avgaser eller återvinningsbarhet? Hur miljövänlig är
  produktionen?
\item
  \emph{Utseende:} Hur vacker uppfattas produkten vara av olika
  kundgrupper? Vilka estetiska värden skapas genom exempelvis design och
  färgval?
\item
  \emph{Felfrihet:} I vilken utsträckning är varan behäftad med fel
  eller brister?
\item
  \emph{Säkerhet:} Vilken är risken att varan orsakar skada på person
  eller egendom? Är varan speciellt skyddad mot yttre påverkan?
\item
  \emph{Hållbarhet:} Kan produkten användas, lagras och transporteras
  utan att den försämras eller kommer till skada?

  \begin{enumerate}
  \def\labelenumi{\arabic{enumi}.}
  \item
    Tjänstekvalitet
  \end{enumerate}
\end{itemize}

En tjänst är ett resultat som levereras av en utförare (leverantör) till
en mottagare (användare). Ofta bygger tjänster på en kombination av
arbetsinsatser utförda av människor och teknik i samarbete.
Interaktionen mellan utförare och användare är ofta central för hur
användaren uppfattar tjänstens kvalitet. Nedan följer exempel på
kvalitetsdimensioner som kan vara viktiga för en tjänst (Zeithaml et al.
1990, enl. Bergman och Klefsjö 2002):

\begin{itemize}
\item
  \emph{Pålitlighet}: Hur jämnt är tjänstens resultat, t ex avseende
  punktlighet och precision? Hur väl utförs det som utlovats?
\item
  \emph{Trovärdighet:} Kan man lita på leverantören?
\item
  \emph{Tillgänglighet:} Hur lätt är det att få kontakt med
  leverantören?
\item
  \emph{Kommunikationsförmåga:} Hur god är leverantörens förmåga att
  kommunicera med kunden på ett för kunden naturligt sätt som kunden
  förstår?
\item
  \emph{Tjänstvillighet:} Hur väl upplevs viljan att hjälpa kunden?
\item
  \emph{Artighet:} Hur är leverantörens uppförande i form av hövlighet,
  omtanke och vänlighet?
\item
  \emph{Inlevelseförmåga, empati:} Hur stor är leverantörens förmåga att
  leva sig in i kundens situation?
\item
  \emph{Omgivning:} Hur är kvaliteten hos den miljö i vilken tjänsten
  utförs, t ex utrustning och lokalernas utseende?

  \begin{enumerate}
  \def\labelenumi{\arabic{enumi}.}
  \item
    Processkvalitet
  \end{enumerate}
\end{itemize}

Många examensarbeten inom teknik handlar om att lösa problem, utveckla
nytt, eller att förbättra i ett sammanhang där människan och tekniken
samverkar. Det människor gör för att genomföra en viss verksamhet kallas
allmänt för \emph{process}. Processer innefattar uttalade och outtalade
aktiviteter som utförs av människor för att stödja verksamheten i en
organisation eller ett företag. Ofta kan det viktigaste bidraget i ett
examensarbete utgöras av en utredning av eller förslag på nya
arbetssätt, metoder eller aktiviteter. I dessa sammanhang är processens
kvalitet ofta central. Då kan följande kvalitetsdimensioner vara
relevanta:

\begin{itemize}
\item
  \emph{Effekt}: Hur stora är vinsterna med införandet av den nya
  metoden? Hur stora kvalitetsförbättringar kan uppnås? Kan något helt
  nytt åstadkommas som inte gick att genomföra tidigare?
\item
  \emph{Införandekostnad:} Vad kostar det att utbilda personalen? Vad
  blir kostnaden för omorganisation till följd av införandet?
\item
  \emph{Användbarhet:} Hur lätt är metoden att lära sig? Hur väl stödjer
  den användarna i sina arbetsuppgifter?
\item
  \emph{Automatiseringsgrad:} Hur stor andel av manuellt arbete kan
  undvikas? Hur väl låter sig processen stödjas av datorbaserade
  verktyg?
\item
  \emph{Acceptans}. I vilken utsträckning kommer användarna av metoden
  att acceptera införandet och stödja förbättringsansträngningarna?

  \begin{enumerate}
  \def\labelenumi{\arabic{enumi}.}
  \item ~
    \section{Datainsamling}\label{datainsamling}
  \end{enumerate}
\end{itemize}

Datainsamlingen i ett examensarbete kan ske på olika sätt. I följande
delkapitel presenteras hur man praktiskt går tillväga. Kontinuerligt
under arbetet bör man föra \emph{loggbok} för att dokumentera vad man
gör. Man kan undersöka förväntningar på arbetet och åsikter genom
\emph{enkäter} och \emph{intervjuer}. Att beskriva hur något görs är
ofta en startpunkt för ett examensarbete, t ex att beskriva ett
arbetsflöde. \emph{Observationer} är ett sätt att samla in data för
detta. \emph{Mätningar} kan omfatta såväl tekniska och fysikaliska
aspekter, som organisatoriska. Ett examensarbete kan också innefatta
\emph{data som andra samlat in}, och särskilda hänsyn behöver tas i
detta fall. Genom att systematisera datainsamlingen kan man få ut mer av
de mätningar man gör. \emph{Försöksplanering} är ett effektivt verktyg
för att åstadkomma detta. Slutligen tar vi upp \emph{etiska och
juridiska aspekter} på datainsamling och -användning.

\begin{enumerate}
\def\labelenumi{\arabic{enumi}.}
\item
  Loggbok
\end{enumerate}

När man genomför sitt examensarbete samlar man medvetet eller omedvetet
på sig en stor mängd data. Varje dag tar man in fakta från sin omvärld,
bearbetar denna och väljer mellan alternativa vägar. En viktig del i
examensarbetet är att kunna redogöra för dessa val och reflektera över
dem. Detta kräver att man på något sätt dokumenterat sitt arbete. En
\emph{loggbok} är ett enkelt och bra verktyg för att samla dessa data.

Loggboken behöver inte innehålla alla detaljer. Det viktigaste är att
dokumentera vad som gjorts och när. Möten, intervjuer etc är det viktigt
att ha anteckningar från, men också arbete internt inom examensarbetet,
t ex designbeslut bör dokumenteras. Om examensarbetet utförs till stor
del vid datorn kan man överväga en elektronisk loggbok, men t ex i möten
och fältstudier är en pappersbaserad loggbok att föredra.

En av examensarbetets delar som är viktig att dokumentera är inläsningen
av litteratur och annat bakgrundsmaterial. Det kan vara bra att
dokumentera i loggboken att den görs och när det sker, medan resultatet
av inläsningen dokumenteras i rapportens teorikapitel med tillhörande
referenslista.

Loggboken är en datakälla som används i arbetet, men den publiceras inte
i rapporten.

\begin{enumerate}
\def\labelenumi{\arabic{enumi}.}
\item
  Enkäter
\end{enumerate}

Enkäter kan användas för att samla in åsikter och uppfattningar från en
större grupp människor. En enkät är ett frågeformulär med huvudsakligen
fasta frågor, oftast med fördefinierade svarsalternativ. Den som svarar
på enkäten fyller i svaren själv och returnerar sedan enkäten. Man kan
distribuera en enkät på olika sätt (Ejlertsson 2005):

\begin{itemize}
\item
  \emph{Postenkät:} sänds ut på papper och returneras via frankerat
  svarskuvert.
\item
  \emph{Gruppenkät:} distribueras till personer som regelbundet samlas
  på ett ställe, t ex en arbetsplats.
\item
  \emph{Enkät till besökare:} delas ut till personer som självmant söker
  sig till en viss lokal, t ex en affär, eller en webbsida. Urvalet
  styrs naturligtvis av att personen valt att komma till denna lokal
  eller besöka denna webbsida.
\item
  \emph{Datorenkät:} enkäten eller en länk till enkäten på en server
  distribueras t ex via e-post.
\item
  \emph{Enkät för den intresserade:} delas ut som bilaga till en tidning
  eller med en produkt. Till samma kategori hör enkäter som ställs till
  tittare som ser på ett visst TV-program. Här har man ingen kontroll
  över vem som svarar på enkäten.
\end{itemize}

Grunden för hur enkätresultaten kan användas och generaliseras ligger i
hur \emph{urvalet} gjorts, dvs hur intervjupersonerna valts ut ur den
\emph{population} eller \emph{urvalsram} man vill undersöka. Som ett
första steg behöver man veta exakt vilken population undersökningen
gäller. Om undersökningen gäller alla studenter vid ett visst
universitet i ett visst ämne under en viss tidsperiod, får man utgå från
en lista över dessa. Om undersökningen gäller en bredare population, t
ex ungdomar i Sverige i åldern 15-20 år blir det svårare att få en bra
förteckning över population. Den svenska offentlighetsprincipen gör dock
att man kan få ut ganska mycket information som finns hos myndigheter.

Ur sin förteckning över den undersökta populationen väljer man ut vilka
enkäten ska sändas till, enligt någon av följande principer (Ejlertsson
2005; Rosengren och Arvidson 2002).

\begin{itemize}
\item
  \emph{Totalundersökning:} Man sänder enkäten till samtliga i
  populationen. Detta är praktiskt möjligt endast för små populationer.
\item
  \emph{Obundet slumpmässigt urval:} Man väljer med hjälp av slumptal ut
  en delmängd ur populationen. Varje individ har samma sannolikhet att
  väljas ut för att resultaten ska bli representativa.
\item
  \emph{Systematiskt urval:} Man väljer ut var N:te individ om man ska
  välja ut enligt proportionen 1/N. Risken med detta angreppssätt är om
  det finns någon periodicitet i listan, t ex om man väljer var 20:e
  lägenhetsinnehavare i femvåningshus med fyra lägenheter i varje plan,
  så väljer man hela tiden ut personer som bor på samma våning.
\item
  \emph{Klusterurval} (kallas också tvåstegsurval): Om populationen är
  naturligt grupperad i kluster, t ex boende i trappuppgångar, kan man
  först välja ut vilka trappuppgångar man ska studera, och sedan vilka
  av de boende i dessa trappuppgångar man sänder enkäten till. Urvalen
  sker i båda stegen med slumpmässigt urval. Klusterurvalet kan
  underlätta den praktiska hanteringen av enkäten, eftersom man kan få
  de utvalda personerna geografiskt mer samlade.
\item
  \emph{Stratifierat urval} påminner om klusterurval, men här finns det
  en systematisk skillnad mellan klustren. Klustren tillhör olika
  kategorier, eller strata. Om delar av populationen bor i hyreshus och
  delar av den i villor, utgör dessa två naturliga strata. Ur dessa
  strata väljer man sedan personer för enkäten. Stratifierat urval är
  särskilt tillämpligt om kategorierna har olika storlek. Urvalsandelen
  behöver inte vara lika stor i alla strata.
\end{itemize}

När man valt ut sitt urval för undersökningen vill man att så många som
möjligt av dessa faktiskt svarar på enkäten, dvs. att \emph{bortfallet}
är så litet som möjligt. De som inte svarar alls på enkäten brukar
benämnas \emph{externt bortfall}, och enstaka frågor i enkäten som inte
besvaras benämns \emph{internt bortfall}. Man bör inte kompensera för
ett bortfall genom att välja ut nya personer, eftersom det påverkar
slumpmässigheten i urvalet. Däremot kan påminnelser och belöningar, t ex
i form av vinstchanser, minska det externa bortfallet. Om bortfallet
trots påminnelser är stort, kan man dra ett slumpmässigt urval ur
bortfallet, ringa upp dessa och ställa några nyckelfrågor, för att på så
sätt kategorisera bortfallet. Det interna bortfallet reduceras främst
genom hur enkäten utformas.

Själva kärnan i genomförandet av en enkät är utformningen av frågorna.
Det finns en lång rad aspekter att ta hänsyn till (Ejlertsson 2005):

\begin{longtable}[]{@{}ll@{}}
\toprule
\begin{minipage}[t]{0.48\columnwidth}\raggedright\strut
Enkelhet i språket

Entydiga frågor

Precisa i tid och rum

Ej ledande frågor

Undvik (dubbla) negationer

Undvik kunskapsfrågor

En fråga åt gången

Ett svar åt gången\strut
\end{minipage} & \begin{minipage}[t]{0.48\columnwidth}\raggedright\strut
Korta, koncisa frågor

Ej alltför känsliga frågor

Ömsesidigt uteslutande svar

Uttömmande svarsalternativ

Ordning på svarsalternativen

Symmetri i svarsalternativen

Neutrala svarsalternativ

Undvik hypotetiska frågor\strut
\end{minipage}\tabularnewline
\bottomrule
\end{longtable}

En enkät kan innehålla frågor som förväntar kvantitativa svar, t ex
''hur gammal?'', ''hur ofta?'' Lite mer komplicerat blir det när man
vill få svar på frågor om åsikter, attityder och känslor. Då får man
konstruera mätskalor, t ex exempel visuell analogskala och
Likert-skalan.

I en visuell analogskala presenteras två ytterligheter längs en linje
som t ex är 100 mm lång, se figur 6.1. Personen som besvarar enkäten får
markera sitt svar med ett kryss längs linjen. Svaret tolkas i termer av
avstånd från ena ändpunkten.

Med en Likert-skala (uttalas {[}Lickert{]}) låter man svarspersonen ta
ställning till ett påstående. Skalan är fem- eller sjugradig och spänner
från ''instämmer helt'' till ''instämmer inte alls'', alternativt ''tar
helt avstånd'', se tabell 6.2.

När man utformat enkäten är det viktigt att man provar den på en mindre
grupp innan den distribueras till hela urvalsgruppen. Därigenom kan man
få synpunkter på frågornas utformning och hjälp att identifiera
eventuella oklarheter och felaktigheter.

Inbjudan till enkäten bör ske i ett följebrev, antingen på papper eller
via e-post. Följebrevet bör innehålla följande information (Ejlertsson
2005):

\begin{itemize}
\item
  Syfte med undersökningen
\item
  Varför personen valts ut
\item
  Information om att enkäten är frivillig
\item
  Svarsrutiner och tidsgränser
\item
  Konfidentialitet
\item
  Kontaktperson
\end{itemize}

Brevet kan med fördel skrivas på ett företags eller en högskolas
brevpapper och signeras av den som gör examensarbetet och av
handledaren. Det bidrar till ett intryck av professionalism som kan öka
svarsfrekvensen. Enkäter som skickas ut via e-post bör skickas från en
företags- eller högskoleadress, inte från privatadresser.

Påminnelsebrev som sänds ut bör vara ganska kortfattade, sammanfatta
syftet med studien och hänvisa till den tidigare utsända enkäten.

Tabell 6.2. Enkätfrågor med Likert-skala

\begin{longtable}[]{@{}lllllll@{}}
\toprule
Påstående & Instämmer helt & Instämmer delvis & Neutral & Instämmer inte
& Instämmer inte alls & Inte tillämpligt\tabularnewline
Jag tycker att mina studier är mycket stimulerande & & X & & &
&\tabularnewline
\ldots{} & & & & & &\tabularnewline
Mina lärare är mycket intressanta att lyssna till & X & & & &
&\tabularnewline
\bottomrule
\end{longtable}

\begin{enumerate}
\def\labelenumi{\arabic{enumi}.}
\item
  Intervjuer
\end{enumerate}

För att få in bakgrundsmaterial till ett examensarbete och att få
synpunkter på ett förslag till lösning kan man genomföra intervjuer. En
intervju är en mer eller mindre systematisk utfrågning av
intervjupersoner kring ett visst tema. Frågesvaren antecknas eller
spelas in på något ljudmedium. Intervjuer kan ske via telefon eller i
direkt möte mellan intervjuaren och intervjupersonen. Liksom för enkäten
väljs intervjupersonerna ut genom \emph{urval} ur en population. Om
intervjun utgör en kvalitativ studie som inte fokuserar på
representativitet är inte urvalets fokus på slumpmässigt urval, utan mer
på att urvalet täcker den variation som finns i populationen. Urvalet
sker då med stratifiering, dvs ett antal kategorier av personer
definieras och intervjupersoner väljs ut från dessa kategorier. T ex kan
kategorier vara män-kvinnor, nyanställda-erfarna, chefer-teknikpersoner.
Eftersom urvalet då inte är slumpmässigt kan man inte dra generella
slutsatser om populationen som man gör sitt urval från. Däremot kan man
utforska området kvalitativt på djupet.

Intervjuer kan som nämnts i kapitel 3.3 ha olika grad av struktur.
Tabell 6.3 presenterar en översikt av den öppet riktade, den
halvstrukturerade och den strukturerade intervjuformen.

Tabell 6.3. Översikt över olika typer av intervjuer (Lantz 1993 och
Rosengren och Arvidson 2002)\footnote{Källorna presenterar också en helt
  öppen typ, men den utelämnas här eftersom examensarbeten i sig antas
  ha en avgränsning som utesluter den helt öppna intervjutypen.}

\begin{longtable}[]{@{}llll@{}}
\toprule
& \textbf{Öppet riktad} & \textbf{Halvstrukturerad} &
\textbf{Strukturerad}\tabularnewline
Mål & Individens upplevelse av ett fenomens kvaliteter & Individens
upplevelse av kvantiteter och kvaliteter & Intervjuaren söker kunskap om
relationen mellan begrepp, om samband\tabularnewline
Uppläggning & Intervjuguide, öppet inom valda frågeområden & Blandat
fasta frågor med bundna svar, och öppna frågor & Fasta frågor med bundna
svar\tabularnewline
Syfte & Utforskande &
\vtop{\hbox{\strut Beskrivande/}\hbox{\strut förklarande}} &
\vtop{\hbox{\strut Beskrivande/}\hbox{\strut förklarande}}\tabularnewline
\bottomrule
\end{longtable}

En \emph{öppet riktad} intervju styrs av en intervjuguide med
frågeområden. Frågorna kan ställas med olika formuleringar och i olika
ordning i olika intervjuer. Intervjun kan till stor del styras av vilka
delområden som intervjupersonen är mest benägen att berätta om. Man ska
dock vara medveten om risken att personen gör det för att den inte vill
prata om andra delområden. För att säkerställa att varje delområde får
ett minimum av behandling i varje intervju, kan man avsätta hålltider
för olika delområden inom intervjuns ram. En öppet riktad intervju bör
spelas in på ett ljudmedium, eftersom den öppna karaktären gör att man
kan komma att få information inom helt andra områden än man tänkt sig
från början. Det är då angeläget att kunna gå tillbaka till det
inspelade ljudet för att höra vad intervjupersonen faktiskt sa. Denna
typ av intervju är också kvalitativ till sin karaktär, och då utgör ord
och beskrivningar de data man får ut från intervjun.

I den \emph{halvstrukturerade} intervjun blandar man öppet riktade
frågor med fasta frågor som har bundna svarsalternativ. För frågorna med
fasta svarsalternativ är det viktigt att man ställer dem med samma
formuleringar och i samma ordning i varje intervju. Annars riskerar man
med sina formuleringar att påverka intervjupersonen på olika sätt.

Den \emph{strukturerade} intervjun är i princip en muntlig enkät.
Fördelen med att genomföra en enkät muntligt är att den svarande inte
behöver fylla i svaren själv, samt att man har möjligheten att få oklara
frågor förtydligade. Risken för internt bortfall minskar om man genomför
enkäten som en strukturerad intervju. Nackdelen är naturligtvis att det
tar mycket mer tid för intervjuaren att kontakta varje person och gå
igenom frågelistan.

De olika typerna av intervjuer kan kombineras inom en studie. T~ex kan
en öppet riktad intervju användas för att ta fram underlag för en
frågelista som sedan används i en strukturerad intervju eller som enkät.
Man kan också följa upp en enkät med en öppet riktad intervju till
personer som svarat på ett visst sätt på enkäten, för att öka
förståelsen kring det undersökta fenomenet.

Genomförandet av en intervju kan delas in i fyra faser:

\begin{itemize}
\item
  Sammanhang
\item
  Inledande frågor
\item
  Huvudfrågor
\item
  Sammanfattning
\end{itemize}

Intervjun börjar med att intervjuaren beskriver intervjuns
\emph{sammanhang}. Vad är syftet? Varför är personen utvald för
intervju? Hur hanteras och bearbetas det sagda? Om intervjun ska spelas
in ska man i denna fas söka samtycke till det från intervjupersonen.

Som \emph{inledning} till frågorna bör man ställa några grundläggande
och neutrala frågor, t ex om ålder, utbildning och arbetsuppgifter.
Frågesvaren behövs för att sätta intervjupersonen i rätt kontext, men de
bidrar också till att få igång samtalet med intervjupersonen kring
frågor som har enkla och raka svar.

\emph{Huvudfrågorna} i intervjun bör ställas i en ordning som upplevs
logisk för den som blir intervjuad, vilket kanske inte är samma ordning
som intervjuaren finner mest logisk. Mot slutet av huvudfrågorna är det
lämpligt att gå över mot närliggande och neutrala frågor igen, särskilt
om intervjun berört personliga frågor, för att skapa en positiv stämning
och möjliggöra fortsatt samverkan.

Till sist \emph{sammanfattas} intervjun av intervjuaren i korta drag,
och möjlighet ges för intervjupersonen att lägga till något som den
upplever saknas. Slutligen repeteras förutsättningarna för intervjun och
rutiner för eventuell återmatning till intervjupersonen bestäms.

Intervjuer kan spelas in på valfritt ljudmedium, men för den kommande
analysen är digitala ljudmedier att föredra, eftersom det är lättare att
navigera i dem. En bärbar PC med mikrofon ger ofta tillräcklig
ljudkvalitet för en intervju. Det är viktigt att i förväg prova ut att
tekniken fungerar även för långa intervjuer.

Som komplement till inspelningen kan man föra anteckningar under
intervjun. Detta är särskilt lämpligt om man är två personer som
genomför intervjun. I minnesanteckningarna får man en första indikation
på vad man uppfattar som viktigt i det som kommer fram i intervjun.

Efter intervjun bör man transkribera det inspelade materialet, dvs
skriva ut det sagda, ord för ord. Detta är en tidskrävande och mödosam
process -- en timmes intervju tar 8-10 timmar att skriva ut -- men det
lägger grunden den djupgående analysen av materialet, se kapitel 6.6.2.

\begin{enumerate}
\def\labelenumi{\arabic{enumi}.}
\item
  Observationer
\end{enumerate}

För att studera ett fenomen eller ett skeende i ett examensarbete kan
man använda direkta observationer. Det innebär att man med sina sinnen
eller med tekniska hjälpmedel samlar in data om vad som sker i olika
situationer. Som observatör kan man ha olika grad av interaktion med det
studerade fenomenet, från att vara en aktiv deltagare i fenomenet till
att vara en ren observatör. De som blir observerade kan ha olika grad av
medvetenhet om att de är observerade. Observatören kan vara allt från
helt maskerad till helt öppen som observatör. Tabell 6.4 sammanfattar de
fyra fall som uppträder när man kombinerar dessa två faktorer (Rosengren
och Arvidson 2002).

Tabell 6.4 Fyra kategorier av observationer.

\begin{longtable}[]{@{}ll@{}}
\toprule
& \textbf{Kunskap om att vara observerad}\tabularnewline
\textbf{Interaktion} & \emph{Hög}\tabularnewline
\emph{Hög} & Observerande deltagare\tabularnewline
\emph{Låg} & Deltagande observatör\tabularnewline
\bottomrule
\end{longtable}

En \emph{observerande deltagare} försöker att bli så integrerad i den
observerade gruppen som möjligt. Gruppen är väl medveten om att
observatören finns där. Datainsamlingen sker t ex via
loggboksanteckningar.

En \emph{fullständigt deltagande} observatör är också en integrerad del
av den observerade gruppen. Däremot försöker man så lite som möjligt
visa att man är en observatör. Även i detta fall sker datainsamlingen
via loggbok och liknande.

Den \emph{deltagande observatören} finns med i sammanhanget, utan att
vara en riktig del av det. Man gör inga försök att dölja att man är en
observatör, utan data kan samlas in med öppna metoder t ex intervjuer.
Man kan också samla in data med ''tänk-högt''-metoder. Det innebär att
den observerade personen uppmanas berätta högt hur den resonerar kring
sitt handlande, vilket spelas in för senare analys.

Den \emph{fullständiga observatören} tar inte del i verksamheten, och är
idealt helt osynlig. Datainsamling sker helt dolt, t ex via kamera och
bandupptagning.

I de fall där observatören är synlig finns naturligtvis en risk att man
påverkar det observerade fenomenet. I de fall där de observerade
personerna är omedvetna om observatören, blir de etiska
frågeställningarna snabbt aktuella.

\begin{enumerate}
\def\labelenumi{\arabic{enumi}.}
\item
  Mätningar
\end{enumerate}

''Att mäta är att veta'' är en devis som ibland citeras för att
argumentera för kvantitativa mätningar. I ett examensarbete behöver man
ofta mäta såväl fysikaliska storheter, som mjukare aspekter t~ex på
organisationer. Utan att värdera olika slags mätningar presenteras här
en del översiktliga råd kring kvantitativa mätningar.

Mätningar innebär att koppla tal eller beteckningar till attribut som
beskriver något fenomen. Mätningarna görs för att beskriva fenomenet
enligt definierade regler (Fenton och Pfleeger 1996). När det gäller
fysikaliska mätningar är detta naturligt för de flesta. Längden på t ex
ett bord är 1,00 m, vilket är definierat som den sträcka, som ljuset
tillryggalägger i absolut vakuum under 1/299 792 458 sekund. Vidare
anger antalet värdesiffror den noggrannhet som mätningen är utför med.

Data som samlas in kan betraktas som ett mätvärde på en skala. Det finns
fyra skaltyper, som nedan presenteras från ''låg'' till ''hög'':

\begin{itemize}
\item
  \emph{Nominalskala:} kategorisering av det observerade i olika
  klasser, t ex röd, gul, grön.
\item
  \emph{Ordinalskala:} rangordning av entiteter baserad på ett
  kriterium, t ex bättre än, mer komplex.
\item
  \emph{Intervallskala:} en rangordning där också skillnaden mellan
  nivåerna har en mening, t ex temperatur på en Celsiusskala.
\item
  \emph{Kvotskala:} en skala där det finns en meningsfull nollpunkt.
  Därmed får kvoten mellan två mätningar en innebörd, t~ex temperatur på
  en Kelvinskala.
\end{itemize}

Mätningar kan vara \emph{direkta} eller \emph{indirekta}. Direkta
mätningar är t~ex längdmätningar. En sträcka mäts med en linjal och
mätresultatet läses av direkt. Hastighetsmätningar är indirekta. Man
mäter en sträcka, samt tiden det tar att förflytta sig denna sträcka,
och räknar ut hastigheten som kvoten mellan sträcka och tid.

I mätningar av fysikaliska fenomen är det viktigt att vara noggrann med
mätutrustningen. I alla mätningar uppträder fel som man vill reducera så
mycket som möjligt för att få fram det verkliga värdet. Mätfel brukar
delas in i tre grupper (Jönsson och Reistad 1987), se figur 6.2:

\begin{itemize}
\item
  \emph{Grova fel:} rejäla misstag som beror på fel i avläsning, fel i
  protokollförandet och liknande. De utmärks genom att radikalt avvika
  från den övriga mätserien.
\item
  \emph{Systematiska fel:} fel som alltid uppträder och som orsakas av
  någon konstant störning på mätsituationen.
\item
  \emph{Tillfälliga fel:} små slumpmässiga variationer kring det sanna
  värdet.
\end{itemize}

En fördjupad beskrivning av elektriska mätsystem ges t ex av Bengtsson
(2003).

Mätningar som berör människor eller organisationer medför andra krav. T
ex att mäta arbetstid kan göras väldigt olika, beroende på syftet, hur
begreppet definieras och hur mätningarna utförs. Menar man den effektiva
tiden då man arbetar aktivt med en arbetsuppgift? Menar man tiden man
befinner sig på arbetsplatsen? Syftar mätningen till att effektivisera
arbetet, dvs att några personer kan friställas? Rapporterar den
anställde tiden själv, eller har man stämpelklocka? Osv.

För att få lite tydligare struktur kring mätningar är det lämpligt att
använda GQM-metoden (Goal/Question/Metric) (Basili et al 1994, van
Solingen och Berghout 1999). GQM-metoden består av tre huvudfaser, se
figur 6.3:

\begin{itemize}
\item
  \emph{Definition av mätningar} definierar och dokumenterar mål,
  frågor, mått och hypoteser (mer om detta nedan).
\item
  \emph{Datainsamling} samlar in data från det mätta objektet.
\item
  \emph{Tolkning av data} tolkar data i form av svar på de ställda
  frågorna.
\end{itemize}

Definitionsfasen, som har gett GQM-metoden dess namn, är den mest
centrala. För mätningarna definieras ett antal övergripande mål. Ett mål
kan definieras enligt följande mall där de kursiva orden utgör den
generella mallen:

\emph{Analysera} tjänsten X

\emph{med syfte att} förstå och förbättra

\emph{med fokus på} pålitligheten i leveransen

\emph{med utgångspunkt från} kunden

\emph{i kontexten} av företag Y

Utifrån måldefinitionen ställs ett antal frågor som preciserar målet.

\begin{itemize}
\item
  Vilka arbetssteg genomlöps för att utföra tjänsten?
\item
  Hur lång tid tar varje steg?
\item
  Hur lång tid planeras för varje steg?
\item
  Hur är stegen beroende av varandra?
\item
  Hur stora är variationerna mellan olika kunder?
\end{itemize}

För varje fråga ställer man upp en hypotes, vad man tror svaret är. Det
viktiga är att man konkretiserar ett möjligt svar för att sedan gå över
till det tredje steget och definiera mått.

Måtten specificerar de data som ska samlas in. För mätningar som utförs
manuellt och på organisationer är det viktigt att man också definierar
\emph{vem} som ska samla in mätdata, och \emph{när} de ska samlas in.

Som slutresultat från definitionsfasen får man ett ''GQM-träd'', se
figur 6.4. Trädet definieras från mål via frågor till mått och måtten
tolkas som svar på frågorna och leder till att målet uppfylls.

\begin{enumerate}
\def\labelenumi{\arabic{enumi}.}
\item
  Data som andra samlat in
\end{enumerate}

För att snabba upp undersökningsprocessen eller för att överhuvudtaget
få tillgång till data kan man behöva använda sådana data i sitt
examensarbete som andra samlat in. Rosengren och Arvidson (2002, s
374-393) skiljer på fyra typer av data som andra samlat in:

\begin{itemize}
\item
  \emph{Bearbetat material:} data som samlats in och bearbetats i ett
  vetenskapligt sammanhang, t ex i akademiska publikationer och
  avhandlingar.
\item
  \emph{Tillgänglig statistik:} data som samlats in och bearbetats men
  där man inte dragit några slutsatser från analysen. Statistiska
  centralbyrån (www.scb.se) publicerar regelbundet statistik kring olika
  företeelser i samhället.
\item
  \emph{Registerdata:} data som samlats in för något syfte och är
  tillgänglig i obearbetat format. Ett kundregister i ett företag är ett
  exempel på registerdata.
\item
  \emph{Arkivdata:} data som inte är systematiserad som data, t ex
  protokoll, korrespondens och projektdokumentation.
\end{itemize}

Gemensamt för dessa data är att de har samlats in med ett annat syfte än
den aktuella studien har, och att de är insamlade av andra personer.
Därför är det viktigt att kritiskt värdera såväl material som analyser.
I övrigt kan man använda samma metoder för analys av denna typ av data,
som för data som man samlat in själv.

\begin{enumerate}
\def\labelenumi{\arabic{enumi}.}
\item
  Försöksplanering
\end{enumerate}

Om man i sitt examensarbete ska undersöka olika faktorers påverkan på
ett fenomen och man kan styra faktorer eller kombinationer, får man
snabbt stora mängder av kombinationer av faktorer. Detta gäller t~ex för
experiment och simuleringsstudier. I stället för att variera en faktor i
taget och undersöka dess effekt kan man använda systematiska metoder för
försöksplanering som gör att man kan variera flera faktorer samtidigt
och ändå få ut kunskap om såväl hur de påverkar fenomenet en och en, som
hur de påverkar varandra (Bergman och Klefsjö 2002 s 83-98; Montgomery
2005). Teorin bakom försöksplanering är ganska omfattande och här
beskriver vi endast den praktiska användningen av metoden.

Man börjar med att upprätta en försöksplan som beskriver vilka faktorer
man vill variera och vilka värden eller nivåer på faktorerna man vill
undersöka. Faktorerna kan påverka en och en, men kan också påverka i
kombination med varandra, t~ex parvis. Detta kallas att faktorerna
samspelar. Som ett exempel visas här ett försök med tre faktorer (A, B
och C) med vardera två nivåer (+,--). Tabell 6.5 visar de delförsök som
behöver utföras för ett fullständigt faktorförsök. I tabellen visas
också vilka nivåer som de samspelande faktorerna får (AxB, AxC, BxC och
AxBxC), samt mätresultatet på ett tänkt försök (y). De samspelande
faktorerna får det tecken som produkten av de samspelande faktorerna
har. Den första raden visar t ex att om A och B är negativa blir
samspelets tecken positivt.

Tabell 6.5 Delförsök för ett exempel på fullständigt faktorförsök med
tre faktorer.

\begin{longtable}[]{@{}ll@{}}
\toprule
Delförsök & Faktorer och samspel\tabularnewline
& A\tabularnewline
1 & --\tabularnewline
2 & +\tabularnewline
3 & --\tabularnewline
4 & +\tabularnewline
5 & --\tabularnewline
6 & +\tabularnewline
7 & --\tabularnewline
8 & +\tabularnewline
Skattad effekt & f(A)\tabularnewline
\bottomrule
\end{longtable}

Effekten av en faktor skattas genom medelvärdet av de parvisa
skillnaderna mellan försök med nivå + respektive --. För faktor A, t~ex
är skattningen medelvärdet av
\emph{y\textsubscript{2}--y\textsubscript{1},
y\textsubscript{4}--y\textsubscript{3},
y\textsubscript{6}--y\textsubscript{5}} och
\emph{y\textsubscript{8}--y\textsubscript{7}}. Dessa fyra par har
faktorerna B och C på samma nivå medan A varieras från + till --.
Skattningen av effekten av faktor A blir då enligt formeln:

På samma sätt beräknas skattningen av effekten av de andra faktorerna
eller samspelet mellan faktorerna, genom en fjärdedel av summan av
mätresultaten multiplicerat med tecknet för motsvarande faktor eller
samspel mellan faktorerna.

För ett \emph{reducerat faktorförsök} med tre faktorer och två nivåer
kan försöksmatrisen se ut enligt tabell 6.6.

Tabell 6.6 Delförsök för ett exempel på reducerat faktorförsök med tre
faktorer.

\begin{longtable}[]{@{}ll@{}}
\toprule
Delförsök & Faktorer och samspel\tabularnewline
& A\tabularnewline
1 & --\tabularnewline
2 & +\tabularnewline
3 & --\tabularnewline
4 & +\tabularnewline
Skattad effekt & f(A)\tabularnewline
\bottomrule
\end{longtable}

Faktorerna skattas nu med samma principiella formel, som för faktor A
och fyra delförsök blir enligt följande:

Det man förlorar jämfört med det fullständiga faktorförsöket är att nu
kan man inte skilja från effekten av faktorn C och samspelet mellan A
och B -- båda kolumnerna har samma tecken i alla rader. Om man vill
undersöka detta vidare kan man lägga till nya delförsök. I studie med
flera faktorer kan man börja med en serie delförsök och sedan undersöka
de faktorer som har störst absolutbelopp av den skattade effekten genom
nya försöksserier. Man kan också skatta variansen för effekten genom att
upprepa försöken igen. Den teoretiska bakgrunden och mer kring praktisk
användning av faktorförsök kan t ex återfinnas i Montgomery (2005).

\begin{enumerate}
\def\labelenumi{\arabic{enumi}.}
\item
  Etik och juridik i datainsamlingen
\end{enumerate}

I ett examensarbete ställs man förr eller senare inför etiska
överväganden. Det gäller i relationen till uppdragsgivaren och till
personerna som lämnar uppgifter till studien på ett eller annat sätt.
Det finns sällan klara ja- eller nej-svar på etiska frågor. Just därför
är det viktigt att på förhand tänka igenom vilka riktlinjer man ska
följa i sin studie.

Etiska frågor som uppkommer i relation till uppdragsgivaren för studien
kan handla om publicering av det insamlade materialet. När det gäller
sekretessfrågor bör detta regleras i ett avtal mellan uppdragsgivaren
och den som genomför studien. Detta avtal får dock inte förhindra
publicering av resultat, bara under vilka former det ska publiceras, t
ex anonymiserat. En viktig etisk aspekt är att man följer ett sådant
uppställt avtal. Men om studien visar resultat som är till
uppdragsgivarens nackdel, hur hanterar man det? Eller om man under
studien upptäcker något som är olagligt?

Som examensarbetare bör man naturligtvis vara lojal mot sin
uppdragsgivare, men lojaliteten har också sina gränser. Om man
misstänker att uppdragsgivaren är utanför lagens ramar bör man, om inte
annat så för sin egen del, slå larm. En person som slår larm till
myndigheter eller pressen brukar betecknas ''whistle-blower'', alltså en
som ''visslar i visselpipan''. Det är ett svårt beslut att fatta, och
kan få konsekvenser för anställning och framtida karriär, men valet att
inte larma kan få ännu större konsekvenser. I en valsituation bör
handledaren från högskolan kunna ge stöd.

Etiska frågor i relation till personerna som lämnar uppgifter till
undersökningen är fler, men de flesta är lite mer konkreta och därmed
lättare att ta ställning till.

En grundprincip är att deltagande i studier ska vara frivilligt. En
person ska inte tvingas att delta i studien mot sin vilja. I t ex
enkätstudier är detta ganska tydligt. Enkäter skickas ut till
medlemmarna i urvalsgruppen och de returneras besvarade endast om
personen så vill. I studier av observationstyp är gränserna mindre
klara. En fullständigt deltagande observatör eller en fullständig
observatör (se tabell 6.4) vill ju märkas så lite som möjligt och önskar
därför inte informera de observerade personerna om studien. Den etiska
principen bör ändå överväga. Man bör ha ett medgivande, men detta
medgivande kan omfatta rätten att observera personen inom ett
tidsintervall, utan att exakt specificera när observationen sker inom
tidsintervallet. I vissa länder, t ex i USA, finns det lagkrav på
dokumenterat medgivande från personerna i form av en s k ''consent
form'' (Robson 2002, s 380). Studier där de observerade personernas
identiteter inte samlas in omfattas naturligtvis inte av denna princip,
t ex observationer i en trafiksituation av hur många som använder
cykelhjälm.

En viktig etisk princip är att skydda individens integritet. Detta kan
åstadkommas genom att i så hög grad som möjligt använda kodad
information om individer. I stället för att skriva den intervjuades namn
kan ett kodnamn eller en beteckning användas. Eftersom den svenska
offentlighetsprincipen gäller för högskolor och universitet bör man göra
detta även i sådant som betraktas som arbetsmaterial. När det gäller
kodning är det värt att notera att en karaktärisering av en person eller
ett företag kan avslöja lika mycket som namnet. T ex en person i en
liten grupp människor kan lätt identifieras med hjälp av kön och
yrkesroll och ''ett stort svenskt företag i
telekommunikations­branchen'' får nog de flesta svenskar att associera
till Ericsson.

Data som samlats in ska användas för det syfte som är överenskommet med
uppgiftslämnarna. Uppgifter om t ex en projektgrupps effektivitet, är
ofta byggd på data om individerna. Om överenskommelsen med
uppgiftslämnarna gäller gruppens data med syfte att förbättra
arbetsprocessen, får man inte kartlägga individernas prestationer som
underlag för lönesättning eller liknande.

Behandling av personuppgifter som helt eller delvis är automatiserad
regleras av personuppgiftslagen (SFS 1998:204). Utöver dessa juridiska
frågor finns det flera etiska aspekter att ta hänsyn till, särskilt med
tanke på att det är så lätt att automatiskt samla in data via
elektroniska verktyg, t ex via kameror och ''avlyssningsprogram'' på
datorn.

Som nämnts ovan gäller offentlighetsprincipen för svenska universitet
och högskolor, vilken regleras av tryckfrihetsförordningen (SFS
1949:105). ''Till främjande av ett fritt meningsutbyte och en allsidig
upplysning skall varje svensk medborgare ha rätt att taga del av
allmänna handlingar'' (2 kap, 1 §). Allmänna handlingar omfattar såväl
dokument som bilder och ljudupptagningar som finns hos en myndighet, dvs
vid en högskola eller ett universitet. Det är viktigt att ha med detta i
beräkningen, dels när man sätter upp sekretessavtal med en
uppdragsgivare, dels när man avtalar med uppgiftslämnare om skyddet av
deras uppgifter och identiteter.

\section{Prototyputveckling}\label{prototyputveckling}

I många examensarbeten är målet att utveckla en del av en produkt eller
en metod. Det kan t ex röra sig om att utveckla programvara till ett
företag eller att utveckla en metod för att skatta parametrar till en
simuleringsmodell. Det som karakteriserar ett examensarbete i detta
sammanhang är att specifikationen för det som ska utvecklas inte är
färdig när arbetet börjar. Om den hade varit det så hade inte
utbildningsmålen med examensarbetet kunnats tillgodoses. Istället är
målet att specificera det som utvecklas medan utvecklingen sker. En
metod för att göra detta är att utveckla prototyper med ett evolutionärt
angreppssätt.

Ett evolutionärt angreppssätt åskådliggörs i figur 6.5 (Sommerville
2007). Arbetet börjar med en idé om vad som ska göras, men det är ännu
inte helt klart exakt vad resultatet kommer att bli. En första
specifikation tas fram och baserat på den tas en första produkt fram.
Denna produkt är en prototyp som kan utvärderas med målsättningen att
kunna definiera nästa version av specifikationen. Den uppdaterade
specifikationen används sedan för att utveckla nästa version av
produkten vilken i sin tur kan utvärderas. På så sätt fortsätter man
tills man kan utveckla en slutgiltig produkt eller en slutgiltig
specifikation. När en ny produktversion utvecklas med detta angreppssätt
så kan detta göras som en vidareutveckling av den förra versionen, men
det kan också vara möjligt att börja från början utan att använda sig av
den förra versionen. Särskilt när den sista versionen utvecklas så kan
det vara lämpligt att göra en nystart, eftersom man då eftersträvar en
produkt som har en bra design som inte har försämrats av de ständiga
förändringar som skett under det tidigare arbetet.

Prototyperna kan utvecklas på olika sätt med olika metoder. Man kan
givetvis utveckla dem med samma metoder som ska användas för den
slutgiltiga versionen, men eftersom man, särskilt för de första
versionerna, vill fokusera på vissa egenskaper så kan man ha lägre krav
när det gäller andra egenskaper. Man kan i prototypen t~ex bortse från
krav på svarstider, minnesåtgång och använda metoder som går snabbt att
använda, men som inte är lämpliga för den slutgiltiga produkten. En
simuleringsmodell kan ses som en prototyp som kan utvecklas i syfte att
senare leda till en slutgiltig produkt.

Två exempel på examensarbeten åskådliggör hur arbetet kan gå till:

\begin{itemize}
\item
  \emph{Utveckling av programvaruprodukt:} Examensarbetet börjar med en
  idé om ett program för att lagra incidenter och erfarenheter från ett
  produktionsföretag. Företaget initierar examensarbetet och
  examensarbetarna börjar med att ta fram en första specifikation som
  granskas och godkänns av företaget. Efter det tar examensarbetarna
  fram en prototyp som endast består av de mest basala funktionerna.
  Representanter för företaget utvärderar detta program genom att prova
  att lägga in ett antal realistiska men påhittade incidenter i
  systemet. Efter detta kan de identifiera ett antal förändringar som
  måste göras och ett antal ytterligare funktioner som måste läggas till
  i systemet för att det ska vara användbart. Baserat på detta utvecklar
  examensarbetarna nästa version, vilken åter kan utvärderas av
  företaget. Baserat på detta kan examensarbetarna definiera en
  slutgiltig specifikation som blir det slutgiltiga resultatet av
  arbetet. Företaget kan sedan arbeta vidare med denna specifikation för
  att utveckla en produkt som de kan använda i sin verksamhet. För
  större uppgifter är det i många fall inte möjligt ett ta fram en
  slutgiltig produkt inom ramen för ett examensarbete. Evolutionär
  utveckling är en viktig del i lättrörlig programvaruutveckling,
  ''Extreme Programming'' (Beck 1999).
\item
  \emph{Utveckling av skattningsmetod:} I ett forskningsprojekt behöver
  man stöd för att skatta parametrar till ett simuleringsprogram. Två
  examensarbetare tar fram en första beskrivning av en metod för att
  skatta parametrar. Efter det tar man fram instruktioner,
  pappersbaserade formulär och annat material för att stödja metoden och
  låter personer från forskningsprojektet prova metoden. Baserat på
  detta kan examensarbetarna uppdatera metodbeskrivningen. Som ett nästa
  steg tar de fram ett fullständigt web-baserat stöd för metoden, med
  manualer, instruktioner, databasstöd, etc för att forskningsprojektet
  i framtiden ska kunna utföra denna typ av skattningar. I detta exempel
  var det möjligt att inte bara ta fram en slutgiltig specifikation,
  utan även en slutgiltig version av själva produkten, dvs
  skattningsmetoden.

  \begin{enumerate}
  \def\labelenumi{\arabic{enumi}.}
  \item ~
    \section{Modellering}\label{modellering}
  \end{enumerate}
\end{itemize}

I ett examensarbete kan man ta fram en modell av något fenomen för att
kunna analysera fenomenet eller för att modellen har ett värde i sig.
Exempel på fenomen som man kan studera i ett examensarbete är hur en
webbserver ska konstrueras för att fungera även under hög belastning
eller hur en serviceorganisation bäst kan arbeta för att serva sina
kunder. En modell är en abstraktion av ett fenomen, där man tagit med de
viktigaste aspekterna utifrån det perspektiv man vill modellera, se
figur 6.6. En modell innebär alltid en förenkling av fenomenet -- det är
själva poängen med att bygga en modell att inte alla detaljer finns med.
Denna modell behöver sedan valideras, dvs jämföras med fenomenet eller
en annan modell av fenomenet för att se att de viktiga aspekterna finns
med i modellen och fångar dem på ett rimligt sätt.

En modell kan vara allt från en beskrivning av ett arbetsflöde i form av
ett blockschema, till en differentialekvation som beskriver en aspekt av
ett fysikaliskt fenomen. Modeller kan se olika ut och tas fram för
mycket olika syften. Arbetsbeskrivningen kan syfta till att förstå och
kommunicera till medarbetarna hur en organisation fungerar.
Differentialekvationen kan syfta till att lösa ett optimeringsproblem
genom simuleringar. Oavsett vilken typ av modeller som tas fram finns
det tre viktiga steg som är gemensamma: 1) Modelldesign, 2)
Implementation och 3), Validering. Beskrivningen nedan är generaliserad
från riktlinjer om simuleringsmodeller (Kellner et al 1999). Fördjupad
kunskap om modellering och simulering förmedlas t ex av Law och Kelton
(2000).

\begin{enumerate}
\def\labelenumi{\arabic{enumi}.}
\item
  \includegraphics[width=4.32708in,height=2.11111in]{media/image1.jpeg}Modelldesign
\end{enumerate}

En modell kan användas för olika syften, t ex för att utvärdera ny
teknik, för planering, utbildning och kommunikation. Beroende på syftet
blir modellens olika egenskaper viktiga. En modell som ska användas för
att utvärdera prestanda hos en webbserver under hög belastning kan
beskrivas som en kö som anropen läggs i och en betjänare som utför
uppgifterna i kön. Modellen behöver inte ta upp varje enskilt anrop till
servern, utan kan modellera anropen som en stokastisk fördelning.
Betjänaren behöver kanske inte för ett visst syfte ta hänsyn till att
tiden för betjäning varierar, utan man kan använda ett medelvärde.
Modellen behöver dock vara exakt formulerad, t ex i ett programspråk.
För en modell som ska användas för att beskriva och kommunicera hur
människor arbetar behöver inte beskrivningen vara så exakt utan kan
dokumenteras i ett blockdiagram med rutor, pilar och text i naturligt
språk.

Modelldesignen innebär beslut kring tre viktiga områden, som alla avgörs
med hänsyn till modellen syfte:

\begin{itemize}
\item
  \emph{Avgränsning:} För att få en modell av hanterbar storlek och
  komplexitet behöver modellens omfång avgränsas. Det första steget är
  att tydligt specificera det fenomen som ska modelleras, och definiera
  vad som inte ska modelleras. Omfånget ska vara tillräckligt stort för
  att fånga de viktiga aspekterna av fenomenet, men inte för stort så
  att modellen blir ohanterlig.
\item
  \emph{Indata och utdata:} Vilka in- och utdata ska tas med i modellen?
  Man bör först och främst välja de data som antas ha störst påverkan på
  det modellerade fenomenet. Om man bygger en simuleringsmodell
  motsvarar utdata de parametrar man vill analysera genom simuleringen,
  och indata de parametrar man vill variera.
\item
  \emph{Abstraktionsnivå:} En detaljerad modell har fördelen att många
  aspekter av fenomenet kan fångas. Å andra sidan riskerar en detaljerad
  modell bli så komplex att man tappar överblicken över helheten. En
  detaljerad modell blir också svårare att validera.
\end{itemize}

Det finns en uppsjö av olika typer av modelldesigner. Nedanstående lista
ger några exempel på hur man kan uttrycka sin modell (Kellner et al
1999):

\begin{itemize}
\item
  Tillståndsbaserade processer
\item
  Diskret händelse-processer
\item
  Systemdynamikmodeller
\item
  Petri-nät
\item
  Kömodeller

  \begin{enumerate}
  \def\labelenumi{\arabic{enumi}.}
  \item
    Implementation
  \end{enumerate}
\end{itemize}

Modellens design implementeras på olika sätt beroende på modellens
syfte. För en modell som utgörs av en processbeskrivning innebär
implementationen att denna kommuniceras till användarna i form av ett
seminarium och en webbsida. För en simuleringsmodell innebär
implementationen att man omsätter modellens design i ett exekverbart
simuleringsprogram.

Det finns många programmiljöer som kan användas för att bygga
simuleringsmodeller, t ex matlab/simulink\footnote{www.mathworks.com}
för simulering av dynamiska system. Fördelen med att implementera
modellen i en programmiljö är att denna tar hand om mycket av den
funktionalitet som krävs för att administrera själva simuleringsmodellen
och man kan ägna sin tid och energi åt själva simuleringsmodellen.

\begin{enumerate}
\def\labelenumi{\arabic{enumi}.}
\item
  Validering
\end{enumerate}

Ett viktigt, men ofta försummat steg i modellering är validering av
modellen. Detta innebär att man återkopplar modellen till det
ursprungliga fenomenet och undersöker om modellen är en korrekt och
giltig beskrivning av det modellerade fenomenet.

För en modell som har människor som primära användare, t ex en
processbeskrivning, utförs valideringen genom att man låter ett urval av
de framtida användarna sättas sig in i modellen och uttala sig om den.
Detta sker lämpligen i form av en halvstrukturerad intervju (se kapitel
6.3.3).

För en datorimplementerad modell innebär valideringen exekvering av
modellen med indata för några punkter i indatarymden som har kända
utdata. I exemplet med webbservern finns det en analytisk modell som är
giltig för vissa fördelningar av indata. Man kan köra
simuleringsmodellen med dessa indata och göra valideringen genom att
jämföra med den analytiska modellen för att se att simuleringsmodellen
ger samma utdata som den analytiska modellen för de kända
förutsättningarna. Om man har mätningar på ett riktigt system eller en
prototyp för vissa indata, så kan dessa utgöra referenspunkter för
validering av simuleringsmodellen.

De slutsatser man drar från en modell måste också valideras. Frågor man
bör ställa sig är t ex följande. Ger den ett rimligt resultat? Är
modellen känslig, så att resultaten är starkt beroende av vissa
parameterval? Genom att intervjua erfarna personer inom området kan man
få hjälp med rimlighetsbedömningen. Känslighetsanalysen kan utföras t ex
med hjälp av försöksplanering, se kapitel~6.3.7.

\section{Analys}\label{analys}

När man samlat in data på olika sätt behöver man analysera dem för att
se vad de visar. Denna analys kräver olika slags metoder. Vi delar här
in metoderna i två huvudkategorier, utifrån karaktären på insamlad data:
\emph{kvantitativ} och \emph{kvalitativ}.

\begin{enumerate}
\def\labelenumi{\arabic{enumi}.}
\item
  Kvantitativ analys
\end{enumerate}

Med kvantitativ analys menas att man analyserar kvantitativ data, dvs
data som kan representeras i termer av antal och siffervärden. Man
använder sig ofta av metoder från statistiken i detta arbete. Tanken med
detta delkapitel är inte att ge uttömmande information om kvantitativ
analys eftersom det skulle kräva för mycket utrymme. Istället visar vi
på vilka metoder från statistik och matematik som kan användas och
hänvisar till standardlitteratur i statistik, t ex Blom (2005).

Kvantitativa tekniker kan användas på två huvudsakliga sätt. Dels kan de
användas för att utforska data för att få förståelse och dels kan de
användas för att visa på samband och hypoteser som man tidigare ställt
upp.

Det finns ett antal mått som kan användas för att utforska och beskriva
en datamängd, t ex:

\begin{itemize}
\item
  \emph{Lägesmått}: Om data är på ordinalskala eller högre så kan man
  använda medianvärdet, dvs värdet i ''mitten'' av den sorterade
  mätserien. Om data är på högre nivå (intervallskala eller kvotskala)
  så kan man även använda medelvärdet.
\item
  \emph{Spridningsmått}: Om data följer en intervallskala eller
  kvotskala så kan man t ex undersöka variansen (Blom 2005).
\end{itemize}

För att utforska en datamängd kan man också beskriva den med grafiska
tekniker, t ex:

\begin{itemize}
\item
  \emph{Histogram:} Dessa diagram kan ritas för att förstå hur data är
  fördelat.
\end{itemize}

\begin{itemize}
\item
  \emph{''Box-plots'':} Även dessa diagram visar hur data är fördelat.
  Ett exempel på en boxplot visas till vänster i figur 6.7. \emph{m}
  motsvarar medianen av de data som åskådliggörs, \emph{ök} motsvarar
  den övre kvartilen av data, dvs medianen av alla datapunkter som är
  större än \emph{m}. \emph{uk} är den undre kvartilen, dvs medianen av
  alla datapunkter som inte är större än \emph{m}. \emph{ög} är en övre
  gräns för de typiska värdena, vilken räknas ut som den största
  datapunkten som är mindre än eller lika med \emph{ök}+1.5\emph{d}, där
  \emph{d}=\emph{ökuk} är boxens storlek. \emph{ug} är en undre gräns
  som räknas ut som den minsta datapunkten som är större än eller lika
  med \emph{uk}1.5\emph{d}. Alla datapunkter som är mindre än \emph{ug}
  eller större än \emph{ög} markeras som atypiska. Box-plots bygger inte
  på några starka antaganden om skalor eller fördelningar, vilket
  innebär att det i stort sett alltid är meningsfullt att rita denna typ
  av diagram. Det finns i litteraturen ett antal varianter av box-plots.
  Denna beskrivning följer Frigge (1989).
\item
  \emph{xy-diagram} är ett exempel på hur diagram kan användas för att
  få förståelse för hur två variabler samvarierar. Ett exempel på ett
  diagram av denna typ finns till höger i figur 6.7.
\end{itemize}

För att undersöka sambandet mellan två faktorer kan man beräkna
korrelationskoefficienten, vilken beskriver i hur stor utsträckning de
två faktorerna samvarierar. För att beskriva hur faktorerna samverkas
kan ett matematiskt samband beskrivas med hjälp av linjär regression.
Dessa begrepp beskrivs närmare bl a i Blom (2005).

Hypotesprövning kan användas för att utvärdera hypoteser. Detta görs
genom att man först definierar en nollhypotes och sedan försöker
förkasta den. Man kan t ex ha som nollhypotes att två dataserier har
samma medelvärde. Om man då kan förkasta denna hypotes på en viss
signifikansnivå så har man visat att medelvärdet inte är det samma.
Några vanliga begrepp i detta sammanhang är:

= \emph{P}(typ-1-fel) = \emph{P}(förkasta \emph{H\textsubscript{0}}
\textbar{} \emph{H\textsubscript{0}} sann)

= \emph{P}(typ-2-fel) = \emph{P}(förkasta ej \emph{H\textsubscript{0}}
\textbar{} \emph{H\textsubscript{0}} falsk)

Det högsta värdet man tillåter på α kallar man den signifikansnivå man
valt. Om man säger att man ska använda signifikansnivån 5\% så menar man
att sannolikheten för ett typ-1-fel får vara högst 5\% då man förkastar
en nollhypotes.

Ett enkelt exempel kan illustrera hur ett hypotestest kan utföras. Antag
att man har registrerat ett antal programvarufel under testning av
programmet. Dessa fel klassificeras som förstörande och icke
förstörande, där fel av den första typen förstör data i programvaran,
medan fel av den andra typen inte förstör data. Man har registrerat 19
fel, varav 4 var förstörande. Antag att man formulerar nollhypotesen som
att förstörande fel och icke förstörande fel är lika vanliga. Detta är
samma sak som att sannolikheten att ett registrerat fel är förstörande
och sannolikheten att det inte är förstörande den samma, dvs. 0,5. En
alternativhypotes kan formuleras som att förstörande fel inte är lika
vanliga som icke förstörande fel. Man har valt en signifikansnivå på
1\%, dvs 0.01.

Om nollhypotesen är sann så är sannolikheten att få så få förstörande
fel som 4:

P( 4 förstörande fel \textbar{} \emph{H\textsubscript{0}}) =

Detta blir ungefär 0,0096, vilket är mindre än 0,01, vilket medför att
man kan förkasta nollhypotesen baserat på denna data. Man kan konstatera
att om 5 av de 19 felen varit förstörande så hade sannolikheten varit
0,0318 vilket inte är mindre än 0,01 och man hade inte kunnat förkasta
nollhypotesen. Om man har hittat 19 fel så kan man alltså formulera
testet som att man kan förkasta nollhypotesen till fördel för den
formulerade alternativhypotesen om högst 4 fel är förstörande.

Några vanliga hypotestest kan sammanfattas enligt tabell 6.6. Mer
information om dessa test ges t ex av Blom (2005), Montgomery (2005) och
Siegel och Castellan (1988). I tabellen skiljer man mellan
icke-parametriska test som inte kräver några särskilda skalor eller
fördelningar och parametriska test som ofta kräver normalfördelning och
åtminstone intervallskala.

Ett viktigt steg i kvantitativ analys är att undersöka om data
innehåller felaktiga värden. Det kan t ex vara värden som beror på
missuppfattningar, mätfel, eller liknande. Dessa måste tas bort eller
rättas till så tidigt som möjligt under analysen. Exempel på tekniker är
att undersöka atypiska värden som t ex kan identifieras i ''box-plots''
eller xy-diagram.

Förutom de tekniker som nämnts i detta stycke så finns det ytterligare
en stor mängd tekniker som kan användas beroende på vilket ämnesområde
ett examensarbete är inom.

Tabell 6.6. Exempel på vanliga hypotestest.

\begin{longtable}[]{@{}lll@{}}
\toprule
\textbf{Undersökningsdesign} & \textbf{Parametriskt test} &
\textbf{Icke-parametriskt test}\tabularnewline
En faktor, två behandlingar & t-test &
\vtop{\hbox{\strut Mann-Whitney,}\hbox{\strut Chi-2}}\tabularnewline
\vtop{\hbox{\strut En faktor, två behandlingar,}\hbox{\strut parvisa
jämförelser}} & t-test (''paired design'') &
\vtop{\hbox{\strut Wilcoxon,}\hbox{\strut Sign-test}}\tabularnewline
\vtop{\hbox{\strut En faktor, mer än två}\hbox{\strut behandlingar}} &
ANOVA &
\vtop{\hbox{\strut Kruskal-Wallis,}\hbox{\strut Chi-2}}\tabularnewline
Mer än en faktor & ANOVA &\tabularnewline
\bottomrule
\end{longtable}

\begin{enumerate}
\def\labelenumi{\arabic{enumi}.}
\item
  Kvalitativ analys
\end{enumerate}

Analys av kvalitativ data är annorlunda till sin karaktär än analys av
kvantitativ data. Kvalitativ data utgörs av ord och beskrivningar, och
att räkna medelvärden och varians på sådana låter sig inte göras.
Däremot är \emph{existensen} av ord, begrepp och beskrivningar viktiga i
den kvalitativa analysen, och i vissa fall även \emph{frekvensen}.

De data som analyseras är textdokument, antingen transkriberade
intervjuer (utskrivna i text) eller arkivmaterial (dokument framtagna
för andra syften än den aktuella studien). Att transkribera en intervju
är en arbetskrävande process som kan ta upp till 10 gånger så lång tid
som själva intervjun. Det är därför lockande att göra sammanfattande
transkriberingar för att spara tid. Det kan man göra, men man riskerar
att tappa i precision och djup i analysen. Studiens syfte får avgöra hur
detaljrik transkriberingen ska vara.

Man kan grupper olika angreppssätt till kvalitativ analys i fyra
principiellt olika kategorier (Robson 2002, s 458), där endast de tre
första uppfyller kravet på vetenskaplighet.

\begin{itemize}
\item
  \emph{Kvasi-statistiska} \emph{metoder} bygger på att man räknar
  förekomsten av ord eller grupper av ord i olika texter. På så sätt kan
  man jämföra hur viktiga olika termer och koncept är för olika
  personer. Exempel på en kvasi-statistiska metod är innehållsanalys
  (eng \emph{content analysis}) (Robson 2002, s 351-359).
\item
  \emph{Mallbaserade metoder} utgår från en lista av nyckelord och söker
  förekomsten av dessa i den kvalitativa datan. Listan av nyckelord
  ställs samman utifrån teori och terminologi i det undersökta området.
  Textsegment ur texterna kopplas till dessa nyckelord genom markeringar
  i texten eller genom att placeras i en matris med nyckelorden som
  kolumner och intervjupersonerna eller de studerade dokumenten som
  rader. Denna metod kallas matrisanalys. Här fokuseras inte på hur
  många som säger vad, utan på vems som säger vad.
\item
  \emph{Editerande} \emph{metoder} syftar liksom de mallbaserade till
  att skapa kategorier av ämnen. Skillnaden är att de editerande
  metoderna inte utgår från några nyckelord, utan söker nyckelorden i
  själva datamaterialet. Analyspersonens tolkning av innehållet och
  mönstren i texterna utgör grund för kategorierna. \emph{Grounded
  theory} är ett exempel på en editerande metod.
\item
  \emph{Fördjupande metoder} är namnet på ett angreppssätt som bygger på
  att analyspersonen fördjupar sig i materialet och med hjälp av sin
  kreativitet och intuition drar slutsatser. Denna metod som inte låter
  sig beskrivas systematiskt, kan knappast räknas som vetenskaplig och
  kan därmed inte rekommenderas.
\end{itemize}

Processen i den kvalitativa analysen kan schematiskt beskrivas i fyra
steg. Kvalitativa studier är oftast av flexibel natur och stegen kan
följaktligen genomlöpas flera gånger.

\begin{itemize}
\item
  \emph{Datainsamling:} Detta steg omfattar intervjuer, inspelning av
  observationer, transkribering, arkivsökningar etc fram till dess att
  det finns ett (elektroniskt sökbart) dokument att analysera.
\item
  \emph{Kodning:} Viktiga uttalanden eller passager i dokument markeras
  och kopplas till ett eller flera nyckelord. I en intervju sägs ofta en
  del som inte innehåller någon relevant information och denna behöver
  inte tas med i kodningen. Å andra sidan bör man inte bara ta enstaka
  nyckelord ut sitt sammanhang, utan hellre markera hela passager i
  intervjun för att få dem i ett sammanhang.

  Ett alternativ till enkla nyckelord för kodning är s k
  protokollanalys. I en sådan analys utgörs kodningsschemat av ett s~k
  protokoll som kan vara en schematiserad beskrivning av en händelse
  eller en resonemangskedja. Insamlade data mappas mot protokollet för
  att karaktärisera skeendet eller resonemanget. Detta är särskilt
  lämpligt i observationsstudier, där handlings- eller
  resonemangsmönster observeras och kodas.
\item
  \emph{Gruppering:} De kodade textsegmenten grupperas så att man
  studerar t ex vad olika personer sagt om ett visst nyckelord. Om vissa
  uttalar sig positivt och andra negativt kring detta kan man gå vidare
  för att se om det finns ett mönster i vad man anser med avseende på
  andra nyckelord, t ex vilken roll man har. Grupperingen kan redovisas
  i en s k ''fyrfältare'', enligt illustration i tabell 6.7.
\end{itemize}

Tabell 6.7. Exempel på ''fyrfältare''

\begin{longtable}[]{@{}lll@{}}
\toprule
Förhållande till nyckelord X & Positiv & Negativ\tabularnewline
Chef & ID1, ID3, ID4 &\tabularnewline
Underordnad & ID6 & ID2, ID5, ID7, ID8\tabularnewline
\bottomrule
\end{longtable}

\begin{itemize}
\item
  \emph{Slutsatser:} Baserat på grupperade data kan man dra slutsatser.
  I exemplet ovan kan man se en tendens att chefer är mer positiva till
  fenomenet än underordnade.
\end{itemize}

I kvalitativa studier är det ofta meningslöst att söka slutsatser av
typen att ''80\% av de underlydande är negativa till fenomenet''. Sådana
slutsatser beror på urvalet av intervjupersoner, och i en kvalitativ
studie har man ofta ett litet urval. Man kan i stället söka slutsatser
på ett lite djupare plan, t ex hur man resonerar kring en fråga eller
vilken motivation man har för ett agerande.

I den kvalitativa analysen är begreppet \emph{spårbarhet} centralt.
Slutsatserna som dras från ett datamaterial ska kunna spåras tillbaka
till vilka uttalanden som ligger bakom och vilka grupperingar av
information som lett fram till slutsatsen. Detta ställer krav på
dokumentationen av analysen. Man ska i efterhand kunna följa hur
slutsatserna drogs från materialet.

\section{Resultatvalidering}\label{resultatvalidering}

Hur kan man säkerställa att resultaten man får fram är giltiga, dvs.
står för en så objektiv bild som möjligt av det observerade fenomenet?
Processen att säkerställa detta kallas validering. Resultaten som ska
valideras kan vara en beskrivning av ett fenomen i text, men det kan
också vara en simuleringsmodell som representerar ett fenomen i ett
organisatoriskt eller tekniskt system.

För studier med flexibel design är det lämpligt att ta hänsyn till
följande fem områden (efter Robson 2004, s 174):

\begin{itemize}
\item
  Loggning
\item
  Återkoppling
\item
  Tredje-parts-granskning
\item
  Triangulering
\item
  Långtidsstudier
\end{itemize}

Att föra loggbok över arbetsprocessen, dokumentera beslut och
tankegångar utgör grunden för att säkerställa att den genomförda studien
är väl genomförd och trovärdig. Särskilt viktigt är det att spara
dokumentation från datainsamling och analys, så att man i efterhand kan
undersöka hur vägen fram till en viss slutsats har gått. Under ett
examensarbete går man ibland in i en ''återvändsgränd'' där man inte
kommer längre utan man får ta ett steg tillbaka och försöka på något
annat sätt. Sådana ''återvändsgränder'' är viktiga att ha med i sin
loggbok, och ibland också i slutrapporten.

För att säkerställa att man har en solid faktagrund att stå på kan man
tillämpa återkoppling, alltså att uppgiftslämnare och andra experter får
ge kommentarer om observationerna motsvarar deras uppfattning av de
uppgifter som de har lämnat. Figur 6.8 visar en schematisk bild av
kommunikation mellan människor. Denna kan tillämpas på en undersökning
där en intervjuperson (sändare) säger något (budskap) till en som
intervjuar (mottagare). För att säkerställa att man uppfattat
intervjupersonen rätt ger man dem tillgång till en transkribering eller
sammanfattning av intervjumaterialet (återkoppling). Däremot behöver man
inte vara överens med intervjupersonerna om slutsatserna som dras från
materialet, och validering av slutsatser sker på andra sätt.

Att använda en tredje part som granskare av olika steg i en undersökning
bidrar till att reducera risken att man blir ''hemma-blind'' i studien.
En studentkollega eller en av handledarna för examensarbetet kan fungera
som tredje-parts-granskare i olika skeden av arbetet.

Genom att använda flera olika slags metoder för att samla in data och
analysera dem kan man få en bättre bild av det observerade fenomenet.
Detta kallas triangulering. Begreppet kommer från geometrin där man kan
bestämma en punkts position genom att mäta avståndet från två olika
referenspunkter. I överförd bemärkelse studerar man det observerade
fenomenet från olika perspektiv så kan man få en bättre beskrivning av
det.

Att bedriva en studie under en lång tid kan både bidra till och hota
validiteten i en studie. Risken med att göra korta studier är att man
inte förstår komplexiteten i det man observerar pga. att man har den
historiska kontexten etc. Å andra sidan är risken med att göra långa
studier, att man blir ''hemma-blind'' och blir en del av det observerade
fenomenet eller dess kontext. Ett examensarbete är oftast av en
fördefinierad längd, så därför kan man sällan påverka denna faktor inom
ramen för ett examensarbete.
