\section{Opposition}\label{opposition}

Mot slutet av examensarbetet ska arbetet granskas för att säkerställa
kvaliteten på arbetet och rapporten. Intern granskning av materialet är
viktig, där författaren själv grundligt går igenom rapporten och även
handledaren granskar och följer upp innehållet i syfte att ge
förbättringsförslag. Extern granskning sker av examinator i syfte att
avgöra om examensarbetet har tillräckligt hög kvalitet och rapporten kan
godkännas. Det är också normalt att extern granskning sker av en eller
flera oberoende opponenter i ett planerat oppositionsförfarande. Dessa
opponenter är studenter som själva utför eget examensarbete och
oppositionen ingår som ett viktigt lärandemoment med målet att träna
granskning av andras arbete och samtidigt få värdefull återkoppling från
andra. Att både ge och ta konstruktiv kritik är en grundbult i ett
vetenskapligt förhållningssätt.

Att genomföra opposition tar normalt ett par veckors arbete och innebär
att man noggrant läser en nästan färdig rapport och sammanställer kritik
i form av konstruktiva förbättringsförslag och intressanta
frågeställningar att diskutera. I samband med att examensarbetet
presenteras av författarna redogör även opponenterna muntligt för sin
kritik och deltar aktivt i diskussioner om arbetet med författarna.

Genom att läsa en annan rapport får man bra inblick i hur en rapport kan
skrivas, hur metodik och relaterat arbete knyts ihop med resultaten etc.
Konkreta exempel på hur andra har gjort är väldigt värdefulla i det egna
arbetet. För att som opponent få ut mesta möjliga att arbetsinsatsen kan
man därför med fördel göra detta i ett ganska tidigt skede i sitt eget
examensarbete.

Delstegen i oppositionsarbetet kan typiskt se ut som följer.

\begin{itemize}
\item
  \emph{Leta examensarbeten som börjar bli färdiga}: Många
  utbildningsprogram har en organisation för att förmedla
  oppositionsuppdrag, t ex via en hemsida. Det är normalt
  examensarbetarnas eget ansvar att hitta arbeten att opponera på. Det
  gör inget om ämnet för rapporten man opponerar på är ett helt annat än
  det man själv gör examensarbete inom -- det bidrar i så fall positivt
  till en breddning av den egna kompetensen.
\item
  \emph{Få tillgång till en nästan färdig rapport}: Detta sker typiskt
  genom att opponenterna kontaktar författarna direkt och kommer överens
  om när och hur en lämplig version av rapporten kan levereras.
\item
  \emph{Aktiva studier av rapporten}: Detta innebär att på djupet
  försöka förstå målsättningar, metoder och resultat. Detta kräver ofta
  att man följer upp nyckelreferenser och allmänbildar sig inom området.
  Följande frågor kan vara bra att ha i åtanke när man läser:

  \begin{itemize}
  \item
    Är målsättningar och problemformuleringar tydliga och relevanta?
  \item
    Är källor och referenser till relaterat arbete relevanta och
    välbeskrivna?
  \item
    Är metodiken för arbetet välbeskriven och metodvalet lämpligt?
  \item
    Är resultaten välunderbyggda, intressanta och välbeskrivna?
  \item
    Uppnår resultaten målsättningarna?
  \item
    Är arbetet genomfört på ett ingenjörmässigt och vetenskaplighet
    sätt? (se kapitel 2)
  \item
    Är rapportens språk och disposition bra?
  \item
    Vad är oklart?
  \item
    Vad kan förbättras?
  \item
    Vad hade kunnat göras annorlunda?
  \end{itemize}
\item
  \emph{Sammanställning av konstruktiv kritik och frågeställningar}.
  Medan man läser rapporten noterar man viktiga frågeställningar,
  eventuella brister och möjliga förbättringsområden. Kommentarerna bör
  vara riktade mot substansen i arbetet snarare än på smådetaljer och
  fokusera på förbättringsmöjligheter. Kritiken sammanställs i en
  skriftlig oppositionsrapport som överlämnas till författarna efter den
  muntliga oppositionen.
\item
  \emph{Genomförande av muntlig opposition}. När författarna har
  presenterat examensarbetet genomför opponenterna en muntlig
  opposition, ofta modererad av examinatorn. Om mer än en student agerar
  opponent är det bra att i förväg komma överens om hur
  oppositionstillfället delas upp. Ofta är den skriftliga
  oppositionsrapporten så innehållsrik att man får prioritera vad som
  tas upp under den muntliga oppositionen. Det viktigaste är att hela
  tiden försöka vara konstruktiv och bidra till en intressant och
  relevant diskussion. Nedan följer fler råd inför genomförandet av den
  muntliga oppositionen:

  \begin{itemize}
  \item
    Inled gärna med att presentera dig själv och din bakgrund, samt dina
    förkunskaper och ditt intresse för området.
  \item
    Börja med ett övergripande omdöme med tyngdpunkt på positiva delar i
    rapporten. Ge även kommentarer om den muntliga presentationen av
    rapporten.
  \item
    Välj frågor som stimulerar till diskussion.
  \item
    Välj frågor som ger de författarna möjlighet att komplettera och
    förklara.
  \item
    Välj frågor som ger författarna möjlighet att försvara eventuella
    brister och problem.
  \item
    Se till att oppositionen hinner täcka arbetet på bredden,
    innefattande målsättningar, metod, och resultat.
  \item
    Se till att oppositionen täcker någon del på djupet och ger
    författarna chans att diskutera konkreta aspekter av arbetet.
  \item
    Ställ inte frågor på smådetaljer och kommentera inte stavfel och
    lättåtgärdade eller triviala brister -- dessa lämnas till den
    skriftliga oppositionsrapporten.
  \item
    Om du inte tycker att författarna svarat på frågan eller bidragit
    till en uttömmande diskussion, ställ gärna följdfrågor eller försök
    förtydliga frågeställningen.
  \item
    Håll inte läxförhör och ställ inte frågor med syfte att enbart
    ''sätta dit'' författarna. Oppositionen är menat att vara ett
    givande intellektuellt utbyte och inte en rättegång.
  \item
    Avsluta gärna med en sammanfattning och ett värderande omdöme.
  \end{itemize}
\item
  \emph{Godkännande av opposition}. Normalt är det examinatorn för de
  granskade författarna som bedömer oppositionens kvalitet och meddelar
  denna till opponenternas examinator. Är oppositionen
  tillfredsställande registreras godkännande av oppositionsdelen.
\end{itemize}
