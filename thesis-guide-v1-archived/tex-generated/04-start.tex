\section{Start och målformulering}\label{start-och-muxe5lformulering}

\section{Förberedelser och regler}\label{fuxf6rberedelser-och-regler}

Alla utbildningsprogram har planer som definierar regler och riktlinjer
för examensarbetet, både generellt för typen av utbildning och specifikt
för det aktuella programmet. När man börjar närma sig slutet av
utbildningen är det därför klokt att undersöka exakt vad som gäller.
Normalt finns speciella krav för att en student ska få påbörja
examensarbetet, till exempel att ett visst antal poäng är avklarade
eller att en viss mängd obligatoriska kurser är godkända.

Varje utbildningsprogram har också ett antal ämnesområden som är
godkända som \emph{examensarbeteskurser}. Normalt är dessa ämnesområden
knutna till olika institutioner med utbildnings- och
forskningsverksamhet i dessa ämnen. För varje examensarbeteskurs finns
en kursplan som sätter ramarna för examensarbetet. Det finns normalt en
övergripande examensarbetskursplan för varje typ av utbildning vid
lärosätet. Det förekommer också specifika kursplaner för respektive ämne
som är godkänt för det specifika programmet. Om man vill göra
examensarbete inom ett ämne som inte är godkänt för
utbildningsprogrammet kan man med en god motivering söka och få dispens
för detta, om det av utbildningsledningen anses passa bra in i
studentens specifika examen.

I övergripande kursplaner regleras sådant som rör generella mål (i
enlighet med högskoleförordningen och lokala föreskrifter), allmänt om
innehållet (regler för t ex rapportering, presentation, opposition),
riktlinjer kring prestationsbedömning, handledning och betygsskala,
omfattning i antal poäng, samt vad som gäller för anmälan av påbörjat
examensarbete. I specifika kursplaner regleras det som gäller för ett
visst examensarbetesämne inom programmet, typiskt vilka kurser som är
förkunskapskrav för examensarbeten i detta ämne. Dessa kurser kan ingå i
en kurskedja som utgör en fördjupning, inriktning eller avslutning inom
programmet.

Varje student ordnar sin försörjning under examensarbetet på samma sätt
som under sin ordinarie studiegång, ofta genom studiemedel. Många gör en
del av examensarbetet utanför terminstid under sommarmånaderna. Detta
ställer speciella krav på planering. Man är normalt berättigad till
extra studiemedel under sommaren om man utför examensarbete på heltid,
men detta behöver man ansöka separat om.

Följande frågor bör du undersöka noga innan du börjar leta ämnesområde
och handledare till examensarbetet:

\begin{itemize}
\item
  Vilka generella mål gäller för examensarbetet på min utbildning?
\item
  Hur många poäng omfattar examensarbetet på mitt program?
\item
  Vad gäller för regler för att påbörja examensarbetet? Hur många poäng
  krävs?
\item
  Hur sker prestationsbedömning och betygsättning?
\item
  Vilka ämnen är godkända som examensarbeteskurser inom mitt program?
\item
  Vilka förkunskapskrav gäller för olika examensarbeteskurser? Vilka
  fördjupande kurser bör jag ha med i min examen för att göra
  examensarbete i ett visst ämne? Behöver jag innan eller parallellt med
  examensarbetet läsa speciella kurser?
\item
  Vilka regler gäller för studiemedel om jag gör examensarbetet utanför
  terminstid?

  \begin{enumerate}
  \def\labelenumi{\arabic{enumi}.}
  \item ~
    \section{Formulera mål}\label{formulera-muxe5l}
  \end{enumerate}
\end{itemize}

Att formulera mål för examensarbetet är en viktig process där innehåll
och omfattning diskuteras och skrivs ner. Måldiskussionen bör inbegripa
de olika perspektiv som de berörda rollerna representerar:

\begin{itemize}
\item
  Högskolans utbildningsmål och examinatorns uppföljning
\item
  Examensarbetarens personliga mål
\item
  Uppdragsgivarens mål med resultatens användning
\end{itemize}

Målen bör formuleras skriftligt i ett måldokument som alla berörda
parter tagit del av och godkänt ur sitt perspektiv. Goda förutsättningar
skapas på så sätt för ett lyckat slutförande.

\begin{enumerate}
\def\labelenumi{\arabic{enumi}.}
\item
  Utbildningsmål
\end{enumerate}

Varje lärosäte har generella och specifika mål som grundar sig på
högskoleförordningen och utbildningens övergripande mål. Så här kan ett
typiskt exempel på övergripande mål för examensarbete på ett
civilingenjörsprogram se ut (LTH 2006):

''I examensarbetet skall studenten visa förmåga att tillämpa och
sammanställa kunskaper och färdigheter förvärvade inom olika centrala
och kvalificerade kurser inom det aktuella utbildningsprogrammet. I
arbetet skall teknologen visa förmåga att identifiera, analysera och
lösa ett tekniskt eller vetenskapligt problem liksom att värdera
lösningen samt att presentera och dokumentera resultatet. Examensarbetet
skall vara fördjupande och visa att teknologen kan tillämpa vetenskaplig
och ingenjörsmässig metodik.

Examensarbetet är ett självständigt arbete. Det skall utföras ensamt
eller i grupp om två personer. Om examensarbetet gjorts i grupp skall
det framgå vad var och en har bidragit med. Examensarbetet skall göras i
något av de ämnen som anges i utbildningsplanen om inte
utbildningsnämnden medger undantag i det enskilda fallet.''

Målbeskrivningen innehåller ett antal nyckelord: \emph{tillämpa och
sammanställa kunskap}; \emph{identifiera}, \emph{analysera}, \emph{och
lösa problem}. Dessa mål är centrala ur ett ingenjörsperspektiv, då den
blivande ingenjörer i sin kommande yrkesroll förväntas bidra med sådana
färdigheter. Examensarbetet ska också erbjuda möjlighet till
\emph{fördjupning}. I det kommande yrkeslivet behövs ofta förmågan att
sätta sig in i ett nytt problem och gå på djupet med ett ämne. Detta
kräver att nya kunskaper förvärvas på de gamla kunskapernas grund, och
då är förmågan att hitta relevant litteratur och att sätta sig in i
vetenskapliga rapporter ofta avgörande (se vidare kapitel 5 om
litteraturstudier).

Målet om att tillämpa \emph{ingenjörsmässig metodik} är naturligtvis
centralt i en ingenjörsutbildning. Detta utbildningsmål är
mångfacetterat och innefattar en rad färdigheter som handlar om att på
ett systematiskt sätt göra en plan för arbetet och att sedan genomföra
arbetet på ett effektivt sätt. Ingenjörsarbete sker med syftet att
åstadkomma praktiskt användbara lösningar på relevanta delproblem, som i
sin tur går att sätta samman till en verkningsfull helhetslösning. Detta
systematiska tillvägagångssätt inbegriper ofta att följa en välbeskriven
process och att göra avvägningar mellan både tekniska och ekonomiska
begränsningar. Det systematiska arbetssättet hindrar inte att ingenjören
samtidigt är pragmatisk och väljer fungerande genvägar om det är
lämpligt. Det är vår erfarenhet att diskussionen om ingenjörsmässighet
ibland hamnar i skymundan i långa serier av enskilda kurser, och vi ser
examensarbetet som ett ypperligt tillfälle att på djupet behandla detta
viktiga utbildningsmål. Att bli en allt bättre ingenjör är en livslång
resa med examensarbetet som ett naturligt avstamp in i yrkesrollen. I de
kommande kapitlen om metodik, planering och uppföljning, samt
genomförande beskrivs viktiga pusselbitar för ingenjörsmässighet.

Målet om att tillämpa \emph{vetenskaplig metodik} är en grundbult för
högre utbildning och forskning och handlar ytterst om hur vi metodiskt
och säkert kan bygga ny kunskap som står på en solid grund, där fakta
och förståelse baseras på giltiga resultat. Vetenskaplig metodik hjälper
forsknings- och utvecklingsprojekt att minska risken för ogynnsamt
inflytande av ogrundade trosföreställningar, slarv, löst tyckande,
falska påståenden, fördomar eller ren vidskepelse.

Följande viktiga kännetecken förknippas ofta med god vetenskaplig
metodik:

\begin{itemize}
\item
  \emph{Möjliggör oberoende granskning}: En förutsättning för att
  kunskap ska kunna accepteras som giltig är att det är möjligt för en
  utomstående att värdera trovärdigheten. Därför bör särskild vikt
  läggas vid allt som kan underlätta sådan oberoende granskning (jämför
  följande punkter).
\item
  \emph{Bygger på och relaterar till befintlig kunskap}: Det är dålig
  vetenskaplig metodik att inte först undersöka vad andra kommit fram
  till inom ämnet. Att bygga på befintlig kunskap säkerställer att
  undersökningsresurserna fokuserar på nya viktiga problem och bättre
  förståelse av redan kända lösningar. När resultaten finns tillhanda är
  det viktigt att koppla tillbaka till den befintliga kunskapen och
  värdera vad som skiljer sig eller är i linje med det vi vet från
  tidigare undersökningar.
\item
  \emph{Påståenden och resultat är väl underbyggda}. Alla påståenden
  eller resultat ska vara troliggjorda och underbyggda genom utförliga
  resonemang och föreliggande fakta. Det ska vara lätt att förstå hur
  undersökningen leder till slutsatserna.
\item
  \emph{Utförlig redovisning av metodik}. Det är viktigt för möjligheten
  till oberoende granskning att metodiken redovisas öppet och utförligt.
  Hur man gått tillväga är helt avgörande för trovärdigheten i
  resultaten. Motiven bakom valet av specifikt tillvägagångssätt bör
  redovisas tydligt.
\item
  \emph{Redovisa källor öppet}. För att kunna värdera och bygga vidare
  på ett arbete måste man kunna värdera den befintliga kunskap som
  arbetet tagit avstamp i. Fullständiga referenser till relevant
  bakgrundsmaterial och andra vetenskapliga källor är därför mycket
  viktigt.
\item
  \emph{Tydliggör det egna bidraget.} Det är viktigt att det framgår
  tydligt vilken ny kunskap som har skapats i arbetet. Det är dessutom
  synnerligen dålig etik att inte tydligt redovisa vad som är utarbetat
  av andra och vad som är det egna bidraget.
\item
  \emph{Förutsättningslöshet och objektivitet}. Det är viktigt att
  resultaten tas fram och utvärderas på ett förutsättningslöst sätt,
  utan otillbörlig inverkan av förutfattade meningar eller uppgjorda
  mätningar. Både positiva och negativa resultat ska redovisas även om
  de råkar vara ''obekväma'' ur någons synvinkel. Det är önskvärt att
  vara så objektiv det går, och när subjektivitet förkommer så redovisas
  detta öppet.
\item
  \emph{Tydlig redovisning av begränsningar och validitetshot.} All ny
  kunskap är framtagen i ett visst sammanhang och dess giltighet är ofta
  begränsad till detta sammanhang. God vetenskaplig metodik innefattar
  att tydligt och öppet redovisa begränsningar och hot mot validiteten
  (giltigheten). Det finns många fällor och fallgropar när man utreder
  nya frågor och dessa bör redovisas utförligt och det måste tydliggöras
  hur de eventuellt kan äventyra resultatens giltighet.
\end{itemize}

Målet om god vetenskaplig metodik med alla dess implikationer ställer
speciella krav på innehållet i rapporten, vilket behandlas utförligare i
kapitel 7.

Det är examinatorns roll att säkerställa att examensarbetet uppfyller
utbildningsmålen. En hjälp för examinatorn vid granskningen av
måldokumentet kan vara att se till att följande tre huvuddelar finns
tydligt representerade:

\begin{itemize}
\item
  \emph{Fördjupning:} Ett examensarbete ska innehålla ett tydligt
  fördjupningsområde som bygger vidare på studentens tidigare studier.
  Denna del ska stödja utbildningsmål relaterade till bl a
  litteratursökning, samt teoretisk och praktisk fördjupning.
\item
  \emph{Eget bidrag:} Den problemlösande eller utredande delen ska
  innehålla ett tydligt eget bidrag, som ur någon synvinkel är
  nyskapande eller innovativt, t ex i form av en prototyp, en
  kartläggning, en mätmetod, en jämförande analys, ett kontrollerat
  laboratorieexperiment, en arbetsprocess eller ett metodförslag. Denna
  del ska stödja utbildningsmål relaterade till bl a ingenjörsmässig
  metodik och problemlösning.
\item
  \emph{Utvärdering:} Det egna bidraget ska utvärderas på ett
  vetenskapligt sätt och möjliggöra oberoende granskning. Detta
  innefattar i förväg planerat arbete med redovisade planer, metoder och
  utfall. Denna del ska stödja utbildningsmål relaterade till bl a
  vetenskaplig metodik.
\end{itemize}

Dessa tre huvuddelar sammanfattar vad som normalt krävs av ett godkänt
examensarbete i ett ingenjörsämne. Alla dessa delar ska vara
representerade i den skriftliga rapporten och i de muntliga
redovisningarna. Var och en av dessa tre delar bör stämmas av mot de
utbildningsmål som gäller på olika nivåer: nationellt i
högskoleförordningen, lokalt på lärosätet, och specifikt för
utbildningsprogrammet.

\begin{enumerate}
\def\labelenumi{\arabic{enumi}.}
\item
  Personliga mål
\end{enumerate}

Det är klokt att i förväg tänka igenom de egna personliga målen med att
utföra examensarbetet. Detta styr sedan hur man bäst går tillväga för
att leta ämnesområde och handledare. Här är exempel på vanliga
drivkrafter och personliga mål för enskilda studenter:

\begin{itemize}
\item
  Intresse för ett visst företag eller en viss bransch
\item
  Möjlighet att prova på yrkesrollen inför framtiden
\item
  Önskan att forska inom ett visst ämne
\item
  Meritering inför en framtida anställning genom profilering
\item
  Arbeta tillsammans med en pålitlig studiekamrat
\item
  Skapa internationella kontakter
\end{itemize}

Ofta är examensarbetet en viktig pusselbit för valet av första
arbetsgivare. Därför är det lämpligt att tänka igenom livet efter
exa-men i samband med att de personliga målen övervägs. Vill jag jobba
på ett stort eller litet företag? I vilken bransch? Vill jag flytta? Är
jag nyfiken på forskning? Vill jag ha fokus på teori eller praktik?
Tillämpning eller analys? Vilken typ av karriär passar mig bäst och
intresserar mig mest? Var har jag störst chanser på arbetsmarknaden? Att
fundera igenom denna typ av frågor är väsentligt för att styra
examensarbetet mot rätt mål sett ur examensarbetarens egen personliga
synvinkel.

Många gånger har man stor nytta av att vara två som gör examensarbete
ihop i ett \emph{pararbete}. Är man två kan man ta sig an en större
uppgift som därmed kanske blir ännu mer intressant, samtidigt som man
kan stötta och ha glädje av varandra. Om du inte har speciella
personliga mål som kräver att du gör examensarbetet ensam, bör du lägga
energi på att hitta en lämplig partner för ett pararbete. Vår erfarenhet
säger oss att pararbete ofta kommer längre både vad gäller resultatet
och lärandeutfallet för individen. Det verkar också vara så att man i
ett pararbete har lättare att navigera runt de svårigheter som
uppkommer. Dock kräver det ödmjukhet inför varandra och att man
diskuterar igenom samarbetsformerna. Det ska också för examinator och i
rapporten framgå tydligt vem som har gjort vad så att examinationen ändå
kan ske individuellt.

En del vill gärna avsluta sin utbildning med att göra sitt examensarbete
utomlands eller vid en annan svensk högskola. Sådana externa
examensarbeten ordnar studenterna själva. Dessa kräver en extra portion
engagemang och uthållighet av den enskilde studenten, men den egna
högskolans lärare kan ibland hjälpa till med kontakter och råd. Tänk
dock på att allra först ta kontakt med studievägledning och
utbildningsansvariga för att undersöka regler och ansökningsförfarande
för denna typ av specialstudier.

\begin{enumerate}
\def\labelenumi{\arabic{enumi}.}
\item
  Uppdragsgivarens mål
\end{enumerate}

Uppdragsgivaren kan ha flera olika mål och motiv med ett examensarbete.
Dessa mål kan vara knutna till ett specifikt problem, men målen kan även
vara kopplade till mer övergripande syften, exempelvis:

\begin{itemize}
\item
  \emph{Kompetenstillskott}. Ofta vill företag och organisationer få
  impulser utifrån och nya angreppssätt på viktiga problem. Ett sätt kan
  vara att anlita examensarbetare med färska kunskaper från en modern
  utbildning där studenterna har tillgodogjort sig ny teknik.
  Examensarbetet blir en möjlighet att göra något nytt under ordnade
  former med tillskott av extern kompetens.
\item
  \emph{Resursförstärkning}. Ofta finns det mer att göra på ett företag
  än vad tillgängliga resurser medger. Examensarbeten kan bli ett sätt
  att säkra resurser för viktiga utvecklingsprojekt som inte kan
  genomföras av befintlig personal. Detta kan t ex gälla
  produktutveckling eller utvärdering av nya arbetsformer.
\item
  \emph{Rekrytering}. Ett examensarbete ger goda möjligheter för både
  examensarbetarna och företaget att bilda sig en uppfattning om
  varandra. Många examensarbetare anställs av uppdragsgivaren om båda är
  nöjda med lärdomarna. Att ha en kontinuerlig ström av examensarbetare
  är för många företag en strategisk åtgärd för att säkerställa goda
  kandidater vid rekrytering av ingenjörer.
\item
  \emph{Samarbete}. Många företag och myndigheter har som strategiskt
  mål att ha ett långsiktigt samarbete med regionens lärosäten och
  samarbete med högskolor via examensarbeten ger en bra grund för
  gemensamma forskningsprojekt och ömsesidigt kunskapsutbyte. Både det
  omkringliggande näringslivet och högskolan tjänar på denna typ av
  långsiktigt samarbete. Högskolans uppdrag att samverka med samhället
  underlättas och företagen utvecklas och kan vara med och påverka.
\end{itemize}

Det primära målet för en uppdragsgivare är dock i regel förankrat i en
viss, specifik frågeställning. Det finns ofta ett viktigt problem som
behöver lösas eller ett angeläget ämne som behöver belysas. För den
enskilde uppdragsgivaren är ofta slutresultatet det viktigaste, inte hur
man kommer fram till det. Det är därför viktigt att uppdragsgivaren
granskar målformuleringar ur resultatsynvinkel. Är problemet relevant
för min organisation? Är det sannolikt att detta examensarbete ger
resultat som kommer till nytta? Finns det stödresurser nog i min
organisation för att ge goda förutsättningar för detta projekt? Hur tar
vi till vara resultatet i vår fortsatta verksamhet?

\begin{enumerate}
\def\labelenumi{\arabic{enumi}.}
\item
  Målkonflikter
\end{enumerate}

En genomtänkt målformuleringen gör det möjligt att på ett tidigt stadium
upptäcker eventuella motstridiga intressen mellan examinator,
uppdragsgivare och examensarbetare. En vanlig målkonflikt kan gälla att
uppdragsgivarens problemformulering saknar utvärderingsaspekter. Att
bara implementera en prototyp utan att utvärdera den på ett
vetenskapligt sätt är i strid mot utbildningsmålen. Ofta kan detta lösas
genom att det praktiska värdet av utvärderingen tydliggörs.

En annan konflikt kan gälla själva problemets natur. Problemet ska
varken vara för svårt eller för lätt. Det är inte meningen att
examensarbetarna ska bli resurstillskott i en organisation utan att de
lär sig något på avancerad nivå som är relevant för utbildningen. Det är
heller inte bra om examensarbetare får uppdrag som ligger vida över
deras förmåga. Det är alla inblandade intressenters ansvar att bevaka
målkonflikter och hitta lösningar som tillgodoser de olika perspektivens
önskemål på ett rimligt sätt.

En konflikt kan röra vad som får publiceras i rapporten efter
examensarbetet och vad som omfattas av sekretess. Det är viktigt att
diskutera och ta hänsyn till detta då ämnet väljs, så att resultaten
inte är sådana att de inte går att publicera i rapporten. Även om vissa
aspekter av arbetet är så konfidentiella att de inte kan tas upp i
rapporten så måste huvuddelen av resultaten kunna presenteras.

\begin{enumerate}
\def\labelenumi{\arabic{enumi}.}
\item
  Måldokument
\end{enumerate}

Ett bra sätt att säkerställa att alla inblandade parter har en gemensam
förståelse för varandras förväntningar på examensarbetet är att tidigt
ta fram en övergripande målformulering som kan utgöra ett slags kontrakt
mellan examensarbetare, handledare och examinator. Detta måldokument bör
inte vara ''hugget i sten'', utan kan med fördel ändras om alla parter
är med på en ändring av målbilden under arbetets gång. I diskussionerna
kring framtagandet av dokumentet får de berörda parterna möjlighet att
skapa samsyn kring vad examensarbetet går ut på samt förståelse för
varandras roller och åtaganden. Måldiskussionerna och måldokumentet gör
att varje roll kan ta ställning till sina respektive mål. Här är några
frågor som varje roll bör bevaka extra noga i den initiala
målformuleringen:

\begin{itemize}
\item
  \emph{Examensarbetare:} Är detta intressant och motiverande? Blir
  detta ett bra avslut på min utbildning?
\item
  \emph{Handledare och extern handledare:} Har jag kompetens att
  handleda inom detta ämne? Har de förväntade resultaten ett tydligt
  värde för uppdragsgivaren?
\item
  \emph{Examinator:} Uppfyller detta kraven för examensarbete vad gäller
  omfattning och kvalitet i förhållande till utbildningsmålen? Är det
  realistiskt att uppfylla målen inom rimlig tid?
\end{itemize}

Måldokumentet är typiskt en eller två sidor långt och kan t ex innehålla
följande:

\begin{itemize}
\item
  Arbetstitel
\item
  Inblandade intressenters namn och kontaktuppgifter
\item
  Övergripande mål, centrala frågeställningar
\item
  Beskrivning av bakgrunden och motiven
\item
  Delmål som behöver uppfyllas på väg mot de övergripande målen, t ex
  kartläggning, prototypframtagning, utvärdering
\item
  Förväntningar på resultaten och dess nytta
\item
  Ett par nyckelreferenser eller hänvisning till annat underlag som
  ligger till grund för arbetet
\item
  Resurser som behövs för examensarbetets genomförande, t ex
  arbetsplats, utrustning, material
\item
  Planerat start- och slutdatum för arbetet

  \begin{enumerate}
  \def\labelenumi{\arabic{enumi}.}
  \item ~
    \section{Leta ämnesområde och
    handledare}\label{leta-uxe4mnesomruxe5de-och-handledare}
  \end{enumerate}
\end{itemize}

När utbildningens slut närmar sig och personliga mål med examensarbete
övervägs är det lämpligt att börja leta ämnesområde och handledare. Det
är viktigt att hitta en intresserad handledare och en examinator som är
insatt i ämnet. Vissa har kanske redan bestämt sig för ett visst ämne
och letar forskningsnära examensarbete på en relevant institution, medan
andra är mer intresserade av arbetslivskontakt i en viss bransch.

\begin{enumerate}
\def\labelenumi{\arabic{enumi}.}
\item
  Företag och myndigheter
\end{enumerate}

På många större företag och organisationer har man upparbetade rutiner
för att initiera och handleda examensarbeten. Många har webbsidor med
aktuella examensarbetesförslag och några har speciellt ansvariga
personer som koordinerar examensarbeten. Andra företag, ofta mindre, är
inte lika vana vid att hantera examensarbeten och vissa känner knappt
till möjligheten och vilka vinster det kan ge.

Man kan studera det lokala näringslivets webbsidor och därefter ringa
runt till intressanta företag och fråga om det finns möjlighet att göra
examensarbete. Har företaget inte något utlyst förslag kan du komma med
egna idéer som du tror är relevanta för företaget. Har du hittat en
person som har ett tydligt ingenjörsproblem inom ditt kompetensområde
och denna person är villig att agera handledare, så kan
målformuleringsprocessen starta i samarbete med en examinator på
högskolan.

Det är vanligt att en student kommer till lärare vid en institution med
ett förslag som redan är förankrat hos ett företag. Det händer också att
lärare har kontakt med företag som letar examensarbetare i väntan på
lämplig student. När student(er) och ämne finns, är det ofta
studierektorn eller annan koordinator för examensarbeten som letar
lärare som kan agera examinator. Därefter kan det detaljerade arbetet
med måldokumentet börja. Detta arbete koordineras av studenten som
samlar in och bearbetar olika intressenters perspektiv. Det är lämpligt
att hålla ett möte med högskolans examinator, uppdragsgivare och
företagets handledare när målformuleringsprocessen gjort intressenterna
redo för åtagande.

\begin{enumerate}
\def\labelenumi{\arabic{enumi}.}
\item
  Internt på högskolan
\end{enumerate}

Vissa studenter har ett specifikt intresse för en viss frågeställning
eller ett visst ämne. Kanske har du redan läst fördjupningskurser vid en
viss institution och blivit speciellt intresserad av ett visst ämne.
Kanske finns redan ett pågående forskningsprojekt vid högskolan som
söker examensarbetare. Eller så vill du kanske, i samarbete med en
lärare, fritt formulera en relevant uppgift. En startpunkt kan vara
webbsidor och beskrivningar av forskningsprojekt vid högskolan.
Kontakter och diskussioner med lärare i fortsättningskurser kan också
vara en bra början. Ofta finns webbsidor vid ditt lärosäte med förslag
på examensarbeten och information om regler och formalia.

Forskning och ämneskompetens finns ofta organiserad i forskargrupper med
specifik inriktning. Ibland spänner forskargrupperna över
organisationsgränser inom högskolan och det kan vara klokt att ta reda
på vilken forskning som bedrivs var och i samarbete med vilka.

Om man vill göra examensarbete internt på högskolan är det lämpligt att
redan i ett tidigt skede kontakta relevant institutions studierektor
eller annan koordinator för att diskutera möjligheterna. Innan denna
kontakt tas är det lämpligt att bilda sig en god uppfattning om vilka
regler som gäller, samt tänka igenom sitt intresseområde och sin
utbildnings fördjupningsprofil.

\begin{enumerate}
\def\labelenumi{\arabic{enumi}.}
\item
  Anmälan och inskrivning
\end{enumerate}

När examensarbetet ska startas så måste detta registreras administrativt
hos den institution som ska examinera examensarbetet. Det går också
kontakta studievägledning och studiekontor för att söka tillgänglig
information kring examensarbete. För att formellt påbörja examensarbetet
krävs inskrivning som medför att examensarbetet registreras som en
poänggivande kurs.

Vid inskrivningstillfället ska alla formella regler för när
examensarbete får påbörjas vara uppfyllda, t ex att studenten klarat av
ett visst antal poäng eller att vissa obligatoriska förkunskapskurser är
godkända. Det ska också vara klart vem som är handledare och examinator.
Målformuleringsprocessen bör vara i sitt slutskede och en genomarbetad
version av måldokumentet bör vara förankrat hos de inblandade parterna
innan inskrivning sker.

\section{Sammanfattning}\label{sammanfattning}

Målformuleringen ger fokus och avgränsning för examensarbetet och
diskuteras ingående med berörda parter. Målformuleringen resulterar i
ett måldokument som lägger grunden för en mer detaljerad planering.
Målen ska ta hänsyn till utbildningsmål, studentens personliga mål och
uppdragsgivarens mål. Utbildningsmålen berör bl a vetenskaplighet och
ingenjörsmässighet. I måldokumentet bör det tydligt framgå hur följande
tre delar finns representerade i examensarbetet:

\begin{itemize}
\item
  \emph{Fördjupning} i ett visst ämne med tillhörande litteratursökning.
\item
  \emph{Eget bidrag} som är nyskapande.
\item
  \emph{Utvärdering} av det egna bidraget på ett vetenskapligt sätt.
\end{itemize}
