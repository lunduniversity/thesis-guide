\section{Efterord}\label{efterord}

Examensarbetet är höjdpunkten på utbildningen då studenten får prova på
att självständigt genomföra ett stort och avancerat uppdrag. Rapporten
blir ett bestående resultat som visar studentens förmåga att sätta
samman sina förvärvade kunskaper i ett specifikt sammanhang för att
fördjupa sig och lösa ett svårt problem på ett ingenjörsmässigt och
vetenskapligt sätt. Resan mot målet att ta examen fylls av motiverande
utmaningar och användbara insikter. Lärdomarna från examensarbetet blir
en flygande start i den kommande yrkesrollen och förtroendet för den
egna förmågan befästs när resultaten av ansträngningarna blir konkreta.
Rapporten kan användas som referens i mötet med en potentiell
arbetsgivare. Examensarbetet leder ganska ofta direkt till anställning.
Studenterna lär känna handledare och examinatorer och dessa har i många
fall fortsatt kontakt efter examen. Examensarbetet är inte bara
slutförandet av en utbildning utan också startpunkten för yrkeslivet. Vi
hoppas att ni som läst denna bok har haft nytta av den under resan mot
den hägrande examen men även inför kommande utmaningar.

Lund, oktober 2006,

Martin Höst, Björn Regnell och Per Runeson
