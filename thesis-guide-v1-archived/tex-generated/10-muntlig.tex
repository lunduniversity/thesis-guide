\section{\texorpdfstring{\\
}{ }}\label{section}

\section{Muntlig presentation}\label{muntlig-presentation}

Den skriftliga rapporten utgör huvudresultatet från examensarbetet, men
det är ganska få som läser hela eller ens delar av rapporten. Genom
muntliga presentationer har man möjlighet att nå ut till fler
intressenter och kanske få fler personer att läsa hela eller delar av
rapporten.

Som en del av examensarbetet ges minst en muntlig presentation. Denna
följs oftast av opposition (se kapitel 9) där en annan studentgrupp
granskar det presenterade arbetet. Om examensarbetet utförts på ett
företag eller en organisation ger man ibland utöver den på högskolan en
muntlig presentation på företaget som riktar sig till dem som arbetar
där. Då får man möjlighet att väcka intresse för rapporten, och ibland
utgör det också ett tillfälle att marknadsföra sig själv som
arbetssökande.

Det finns många aspekter att ta hänsyn till när man planerar sin
presentation:

\begin{itemize}
\item
  Vem är åhöraren?
\item
  Vad är syftet?
\item
  Hur bör materialet disponeras?
\item
  Vilka hjälpmedel ska man använda?
\item
  Hur ska det framföras?
\end{itemize}

Dessa områden behandlas översiktligt nedan.

\section{Åhöraren}\label{uxe5huxf6raren}

Vem ska jag presentera mitt arbete för? Svaret styr till stor del hur
man lägger upp presentationen. Dels måste nivån på presentationens form
anpassas till åhörarnas kunskapsnivå. Vilka termer och begrepp är
åhörarna bekanta med? Vilka måste jag definiera och förklara innan jag
kan använda dem? Dels behöver presentationens form anpassas till
åhörarnas intresse för innehållet. Om åhörarna är genuint intresserade
av det tekniska innehållet kan de förbise brister i presentation. Om
åhöraren är där för att kritiskt granska arbetet, eller kanske dig själv
inför en anställning, så blir framförandet viktig för att ge ett
positivt intryck.

I ett examensarbete i samarbete med företag eller andra organisationer
finns olika typer av åhörare. De akademiska åhörarna -- examinator och
opponenter -- intresserar sig för arbetets metodik och genomförande,
medan åhörarna i företaget eller myndigheten mer intresserar sig för
arbetets resultat och konsekvenser. Om man har olika
presentationstillfällen för dessa olika grupper av åhörare, kan man med
fördel låta det påverka sin presentation.

I förberedelsen kan du sätta åhöraren i fokus genom att:

\begin{itemize}
\item
  Skriva ner vilka åhörare du tänker dig, och vad som karaktärisera dem,
  t ex förkunskapsnivå och motiv för att lyssna. Om det finns olika
  målgrupper, prioritera vilken som är viktigast för dig.
\item
  Skriva ner frågor som åhörarna kan tänkas vilja ha svar på utifrån din
  presentation. Skriv ner många frågor, prioritera sedan de som är
  viktigast för den prioriterade målgruppen, och säkerställ att dessa
  frågor blir besvarade i presentationen. Om du vill nå flera
  målgrupper, se till att du besvarar frågor som kommer från flera olika
  målgrupper.
\end{itemize}

En grundprincip i kommunikationen är att det avgörande är vad åhörarna
tar in under presentationen, inte vad talaren basunerar ut!

\section{Syfte}\label{syfte}

Vad vill du åstadkomma med din presentation? Det muntliga mediet är bra
för att förmedla entusiasm och intresse, men är betydligt sämre på att
förmedla stora mängder information. Därför är det mest effektiva
användningen av den muntliga presentationen att marknadsföra ditt
examensarbete, förmedla en översiktsbild och väcka intresse för att
åhöraren senare ska vilja fördjupa sig i arbetets detaljer.

Presentationens syfte ska relateras till åhöraren. Vad vill du ska hända
med åhörarna under din presentation? Ska de minnas något särskilt
efteråt? Ska de bli övertygade om något? Ska de göra något när de kommer
tillbaka till sin arbetsplats?

Samtidigt som den muntliga presentationen är mest lämpad för att
förmedla intryck och känslor så är den innehållsmässiga delen viktig för
trovärdigheten i arbetet. Bedömningen av arbetets trovärdighet och
kvalitet kräver ofta att man presenterar undersökningsmetoder och
-procedurer på en ganska detaljerad nivå. För att få balans mellan att
ge en översikt och vara intresseväckande å ena sidan, och att prata om
detaljer och förmedla information å den andra, krävs en bra disposition
för presentationen.

\section{Disposition}\label{disposition}

Det finns två olika dispositionsmodeller (Rosengren och Arvidson 2002,
Hemlin 2003):

\begin{itemize}
\item
  \emph{Den logisk-historiska modellen} följer undersökningsmodellen med
  presentation av bakgrund och syfte, frågeställningar och relaterat
  arbete, en empirisk del med datainsamling och till sist analys och
  slutsatser.
\item
  \emph{Löpsedelsmodellen} presenterar stoffet i den ordning som det är
  intressant för målgruppen, med det viktigaste först.
\end{itemize}

Den logisk-historiska modellen är lämplig för att disponera den
skriftliga rapporten (se kapitel 7). Rapporten kan läsaren välja att
läsa i vilken ordning man vill och läsaren kan därmed välja ut vad man
finner mest intressant. Dispositionens viktigaste funktion för den
skriftliga rapporten är att hjälpa läsaren att hitta rätt.

Löpsedelsmodellen är lämplig för den muntliga rapporten, då målgruppen
inte kan välja vilken ordning man tar till sig materialet. Som åhörare
till en muntlig presentation är man som mest uppmärksam de första
minuterna av föredraget, varefter koncentrationen sjunker efter hand.
Genom olika hjälpmedel kan man sedan fånga in uppmärksamheten igen och
mot slutet av presentationen kan man få intresset att stiga igen.

En presentation som disponeras enligt löpsedelsprincipen presenterar det
viktigaste budskapet under de första minuterna då åhörarna är som mest
mottagliga, försöker väcka intresse efterhand för viktiga delar, och
avslutar presentationen med en sammanfattning. Det viktigaste budskapet
är oftast det som berör åhöraren mest. Försök därför börja
presentationen med att uttrycka det viktigaste budskapet på ett sätt så
att det berör åhöraren. Det ger störst chanser att väcka intresse och
därmed behålla åhörarens uppmärksamhet under resten av presentationen.

En disposition för en presentation ett examensarbete enligt
löpsedelsprincipen kan t ex se ut enligt följande:

\begin{itemize}
\item
  \protect\hypertarget{OLE_LINK21}{}{\protect\hypertarget{OLE_LINK11}{}{}}Koncentrerad
  sammanfattning av syfte, arbetsmetod och de viktigaste resultaten
\item
  Fördjupad presentation av bakgrund, syfte och frågeställningar
\item
  Presentation arbetsmetod och genomförande
\item
  Presentation av analys och slutsatser
\item
  Sammanfattning av syfte, arbetsmetod och de viktigaste resultaten
\end{itemize}

En ytterligare fördel med löpsedelsprincipen, utöver att den tar hänsyn
till åhörarens förutsättningar, är att man som talare kan anpassa
presentationens längd genom att ta bort delar i slutet, utan att man
därmed tappar bort något som är mycket viktigt för åhöraren. I exemplet
ovan kan man i princip när som helst under de tre punkterna i mitten gå
över till sammanfattningen om man har dragit över tiden på en av de
tidigare punkterna.

\section{Hjälpmedel}\label{hjuxe4lpmedel}

Genom att stödja en muntlig presentation med audiella och visuella
hjälpmedel kan man hjälpa åhöraren att minnas betydligt mer än från en
enbart muntlig presentation.

\emph{Overheadbilder}, på plastfilm eller via datorprojektor, hjälper
åhöraren att följa den struktur som presentationen har. Rubriker och
centrala begrepp kan presenteras och därmed markerar man vad som är just
centrala begrepp. Bilder kan komplettera det man säger och hjälpa
åhöraren att minnas bättre. Det finns dock många risker med
overheadbilder och man gör klokt i att undvika dem genom att:

\begin{itemize}
\item
  Ha lagom mycket information på bilderna. Skriv högst sex ord per rad,
  högst sex rader per bild, minst sex mm bokstavshöjd. Om du behöver
  visa detaljerade diagram, överväg att dela ut dem på papper i stället.
\item
  Låta bilderna ge ett lugnt intryck. Använd en mörk grundfärg till
  text, och två-tre färger för att markera. Undvik gula färger som syns
  dåligt, och tänk på att färgblinda har svårt att skilja på rött och
  grönt. Undvik för mycket information i marginalerna. Presentationens
  titel och ditt namn behöver inte upprepas på varje sida. Använd
  animeringar mycket restriktivt. Det låser dig som talare till en
  förutbestämd sekvens och det distraherar åhöraren om det inte fyller
  ett specifikt syfte.
\item
  Ha lagom många bilder. Presentationen får inte bli ''nästan film''.
  Två-tre minuter per bild är en lämplig grundregel. Öva in
  presentationen i en tom lokal, och ta tiden för din egen del. Åhöraren
  måste hinna se hela bilden och ta den till sig innan det är dags för
  nästa. Att släcka projektorn eller visa en svart bild kan ge ett
  välkommet avbrott i presentationen.
\item
  Tala till åhörarna -- inte till bilderna. Om man tittar och pekar på
  den projicerade bilden riskerar man att vända ryggen till åhörarna.
  Peka i stället på projektorn eller via datorn, så behåller du
  uppmärksamheten i rätt riktning.
\end{itemize}

\emph{Vita eller svarta tavlan} är ett klassiskt hjälpmedel som
fortfarande är användbart. En fördel med tavlan är att informationen
växer fram under presentationens gång och blir en dokumentation av det
som sagts. Tavlan kan med fördel komplettera overheadbilder för
information som man återkommer senare i presentationen. När man skriver
på tavlan måste man vända ryggen mot åhörarna, men det är viktigt att
man snabbt återknyter ögonkontakten med dem när man skrivit klart.

\emph{Musik, ljudinspelning och filmavsnitt} kan användas för att
illustrera något man presenterar. Genom att visa en film från det
problem- eller lösningsområde som examensarbetet gäller, kan man sätta
in åhöraren i ett sammanhang och ge en upplevelse av närvaro. Men
teknikens möjligheter bör användas med förnuft och man måste behärska
tekniken, så att det blir smidiga övergångar mellan olika medier.

\emph{Demonstrationer} av resultat från examensarbetet, t ex
datorprogram och simuleringar, är ett konkret sätt att visa vad som
åstadkommits. Men på samma sätt som med film och ljud är det viktigt att
det fyller ett syfte och verkligen når fram till åhöraren. En skärmdump
av programmet -- som går att förstora upp till läsbar storlek -- är
bättre än att köra den verkliga versionen med för litet typsnitt.

\emph{Saker} som skickas runt bland åhörarna är också ett sätt att ge en
konkret känsla av närvaro. Nackdelen med att skicka runt saker är att
koncentrationen störs under presentationen, både för talare och åhörare.
Det är bättre att lägga sakerna på ett bord och inbjuda åhörarna att
känna och se efteråt, vilket också kan öppna för samtal.

\section{Framförandet}\label{framfuxf6randet}

När du är framme vid framförandet av presentationen handlar det om att
ge liv åt den disposition du tagit fram, stödd av de hjälpmedel du valt.
Ha gärna ett manus som stöd i förberedelsearbetet, men använd det så
lite som möjligt under presentationen. Öva tills du känner dig trygg med
din presentation. Om ni är två ställer det ytterligare krav på
förberedelser. Försök hitta naturliga övergångar mellan talarna, och
växla gärna vid ett par tillfällen.

\begin{itemize}
\item
  Under presentationen är det allra viktigast att ha ögonkontakt med
  åhörarna, så att du upplevs som närvarande.
\item
  Variera gärna rösten som i ett normalt samtal -- det hjälper åhörarna
  att behålla uppmärksamheten.
\item
  Var inte rädd för att ta pauser -- det vinner både talare och åhörare
  på.
\item
  Rör dig på ett sätt som känns naturligt. Håll händerna fria -- inte
  bakom ryggen eller i fickan -- så får du hjälp av ditt kroppsspråk.
\end{itemize}

När det kommer upp frågor under presentationen kan man hantera dem på
lite olika sätt. Om det handlar om att klargöra något som varit otydligt
kan man med fördel besvara det direkt. Om det berör något som kommer
senare i presentation kan man be att få återkomma till frågan lite
senare. Om du känner dig trygg med presentationen kan du med fördel
anknyta till frågan direkt, men då kan man också behöva disponera om
sina hjälpmedel och kanske visa overheadbilder i en annan ordning.

När nervositeten gör sig påmind inför och under presentationen, kom ihåg
att ingen kan ditt examensarbete så bra som du. Och har du gjort ditt
förberedelsearbete väl, så kan ingen din presentation så bra som du
heller.

\section{Sammanfattning}\label{sammanfattning}

Den muntliga presentationen är ett tillfälle att ''marknadsföra''
examensarbetet och dig själv. Noggranna förberedelser bidrar till en bra
presentation. Det är viktigt att definiera vilken målgruppen är och
planera utifrån den. En disposition enligt löpsedelsmodellen bidrar till
ett intresseväckande upplägg och väl valda hjälpmedel lyfter fram
budskapet. Ett väl inövat framträdande hjälper till att hålla
nervositeten under kontroll.
