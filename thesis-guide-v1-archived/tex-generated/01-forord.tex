\section{Förord}\label{fuxf6rord}

Efter mer än 10 års erfarenhet av handledning av examensarbeten vill vi
förmedla en del av de lärdomar vi gjort. Denna bok vänder sig i första
hand till dem som ska genomföra eller handleda examensarbete på en
ingenjörsutbildning. Syftet med boken är att ge praktiska råd i hela
processen, från målformuleringen tills dess att arbetet är slutfört och
rapporten färdig. Därför innehåller boken fördjupade diskussioner kring
projektarbete och metodik. Metodiken bygger på etablerad
forskningsmetodik och är besläktad med den metodik som används i
uppsatsarbeten inom andra tillämpade områden. Ingenjörsstudenters
examensarbeten är dock annorlunda i några avseenden. Förutom att de
utförs inom en teknisk disciplin, så innehåller de ofta en tydlig
utvecklingsdel, t ex att utveckla en modul till en existerande
programvara. Arbetena utförs också ofta på ett företag, vilket innebär
att studenten har detta som sin arbetsplats och även får handledning
där. Studenternas fokus på tekniska ämnen gör också att de har
förhållandevis goda förkunskaper i ämnen som programmering, matematik
och statistik. Utifrån denna bakgrund har vi sett ett behov av en bok
som lyfter fram metodikfrågor för examensarbeten inom den tekniska
disciplinen.

Vi vill gärna passa på att tacka tidigare och nuvarande kollegor för
diskussioner om handledning av examensarbeten. Dessutom vill vi tacka
Bertil Nilsson och Jens Andersson för värdefulla kommentarer på tidiga
versionen av boken, samt Torgny Roxå för inspiration till kapitlet om
muntlig presentation.

Lund, oktober 2006,

Martin Höst, Björn Regnell och Per Runeson
