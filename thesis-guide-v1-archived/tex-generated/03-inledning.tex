\section{Inledning}\label{inledning}

\section{Målsättning med
examensarbetet}\label{muxe5lsuxe4ttning-med-examensarbetet}

Många utbildningar avslutas med ett examensarbete. I examensarbetet är
tanken att man som student ska utnyttja all den kunskap man har skaffat
sig under sin utbildning och tillämpa den i ett större arbete. Denna
uppgift ska likna det arbete man utbildar sig till i en yrkesinriktad
utbildning. Man ska även lära sig att utföra ett större arbete
självständigt, med allt vad det innebär. För en civilingenjörsstudent
kan examensarbetet t ex röra sig om att utveckla någon del av ett större
system eller att göra ett utredningsarbete på uppdrag av ett företag
eller en myndighet.

För det mesta brukar examensarbetet vara ungefär en termin och de flesta
studenter brukar arbeta heltid med det. Det är en ganska stor mängd
arbete och volymmässigt motsvarar det många av de projektuppgifter som
man sedan kommer att utföra när man är färdig med sina studier.

I examensarbetet är det kanske första gången man arbetar så fokuserat
med en enda uppgift. Detta är givetvis positivt, men det ställer stora
krav på moment som planering, uppföljning och rapportering. I arbetet
krävs det att man kan arbeta självständigt med vad man kan kalla ett
disciplinerat arbetssätt.

Enligt \emph{Högskolelagen} (1992:1434) 1 kap 9 §, finns följande krav
på de utbildningsprogram som erbjuds av universitet och högskolor:

''Den grundläggande högskoleutbildningen skall ge studenterna

\begin{itemize}
\item
  förmåga att göra självständiga och kritiska bedömningar,
\item
  förmåga att självständigt urskilja, formulera och lösa problem, samt
\item
  beredskap att möta förändringar i arbetslivet.
\end{itemize}

Inom det område som utbildningen avser skall studenterna, utöver
kunskaper och färdigheter, utveckla förmåga att

\begin{itemize}
\item
  söka och värdera kunskap på vetenskaplig nivå,
\item
  följa kunskapsutvecklingen, och
\item
  utbyta kunskaper även med personer utan specialkunskaper inom
  området.''
\end{itemize}

Examensarbetet är ett självständigt arbete som tillgodoser flera av
dessa punkter. Det ingår t ex att formulera och givetvis att lösa ett
problem. Det ingår även att söka och välja ut litteratur. Examensarbetet
kräver även att man följer kunskapsutvecklingen eftersom det ofta görs
inom ett område som är i forskningens framkant eller som utgör ledande
praxis i ett företag. Dessutom kräver det att man kan kommunicera både
skriftligt och muntligt med en mängd olika människor som på olika sätt
är intresserade av arbetet. På många utbildningar ingår det även att
skriva en populärvetenskaplig sammanfattning av arbetet.

\section{Utförande}\label{utfuxf6rande}

Hur examensarbetet utförs beror på inom vilket ämne det utförs. I en
undersökning som gjordes på Uppsala Universitet såg man att arbetet
kunde utföras på i princip tre olika sätt som kortfattat kan
sammanfattas enligt följande (Eriksson et al. 1997):

\begin{itemize}
\item
  \emph{Alternativ 1}: Arbetet utförs på en institution med en lärare
  vid institutionen som handledare. Arbetet bedöms och examineras av en
  examinator och presenteras på ett seminarium på institutionen.
\item
  \emph{Alternativ 2}: Arbetet utförs i ett forskningsprojekt på en
  institution där den som är ansvarig för projektet är handledare.
  Arbetet bedöms och examineras av en examinator och presenteras på ett
  seminarium inom forskningsprojektet.
\item
  \emph{Alternativ 3}: Arbetet utförs på ett företag eller en myndighet
  utanför universitetet och handleds både från universitetet och
  företaget/organisationen. Arbetet bedöms och examineras av en
  examinator på universitetet och presenteras på
  företaget/organisationen.
\end{itemize}

Man såg att vissa institutioner hade examensarbeten enligt mer än ett av
alternativen, men att man i huvudsak kan säga att ämnen som historia,
engelska och sociologi kan sorteras in under alternativ 1, ämnet fysik
kunde sorteras in under alternativ 2 och ämnen som företagsekonomi, kemi
samt dataingenjörsprogrammen ofta handleddes och utfördes enligt
alternativ 3. Denna bok är skriven i huvudsak utifrån hur examensarbeten
inom den tekniska disciplinen utförs, men stödjer arbeten enligt alla
tre alternativen.

Arbetet brukar utföras av en eller två personer och det utförs som
nämnts antingen i universitetsmiljö på en institution eller på något
externt företag eller myndighet. Storleken på arbetet varierar mellan
olika utbildningar; ibland är det en halv termin, men oftast rör det sig
om arbete motsvarande ett halvårs heltidsstudier.

Exakt vad som ingår i ett examensarbete skiljer sig från utbildning till
utbildning, men ett typiskt exempel på vad som kan ingå är (LTH 2006):

\begin{itemize}
\item
  En skriftlig rapport som presenterar arbetet.
\item
  En separat sammanfattning av arbetet som kan vara utformad som en
  populärvetenskaplig artikel eller som en mer forskningsinriktad
  artikel.
\item
  En presentation vid ett offentligt seminarium.
\item
  Opposition av ett annat examensarbete i samband deras seminarium.
\end{itemize}

För många studenter är examensarbetet ett nytt sätt att arbeta jämfört
med tidigare kurser. Från att ha varit förhållandevis styrda i kurserna
ska man nu klara sig själva i större utsträckning. I tidigare kurser har
det funnits en lärare som har kunnat berätta hur allt arbete ska
utföras. Nu finns det istället en handledare som följer arbetet, men som
inte har något ''facit'' för hur precis allting ska utföras. Det är
också uppenbart att det i denna typ av arbete inte alltid är enkelt att
veta vilka metoder och resultat som är ''rätt'' och vilka som är
''fel''. Denna nya situation upplevs inte bara som positiv. Vad man kan
göra är att arbeta med målformulering, planering och uppföljning av
arbetet. Mer konkret hur detta utförs beskrivs i senare kapitel.

Ett examensarbete kan betraktas som ett projekt och utföras i följande
steg som beskrivs mer ingående i kapitel 4:

\begin{itemize}
\item
  \emph{Uppstart}, där alla berörda parter kommer överens om målen och
  inriktningen med arbetet.
\item
  \emph{Planering}, där en detaljerad tidplan för arbetet tas fram och
  metodiken för arbetet bestäms.
\item
  \emph{Genomförandet}, där själva arbetet utförs. Detta är den längsta
  delen av examensarbetet.
\item
  \emph{Avslutning}, där rapporten blir färdig och en muntlig
  presentation hålls.

  \begin{enumerate}
  \def\labelenumi{\arabic{enumi}.}
  \item ~
    \section{Roller }\label{roller}
  \end{enumerate}
\end{itemize}

Följande \emph{roller}, eller \emph{intressenter} som det också kallas,
är inblandade i ett examensarbete:

\begin{itemize}
\item
  \emph{Examensarbetare:} Det är examensarbetarna som utför själva
  arbetet med hjälp av handledare, examinatorn och andra personer som
  stöd. Normalt sett är det en eller två studenter som utför ett
  examensarbete.
\item
  \emph{Handledare:} Med detta menar vi den handledare som finns vid
  institutionen på lärosätet. I en del examensarbeten finns det
  ytterligare en handledare på ett företag där arbetet utförs.
  Handledaren har som mål att efter bästa förmåga hjälpa
  examensarbetaren att nå fram till målet med arbetet. Baserat på sin
  erfarenhet kan handledaren t ex hjälpa till att formulera
  målsättningen med arbetet, komma med förslag på litteratur, granska
  tidplaner för att avgöra om de är rimliga, samt läsa och kommentera
  rapporten i olika steg.
\item
  \emph{Extern handledare:} Då examensarbetet utförs på ett företag
  eller en myndighet finns ytterligare en handledare. Denne hjälper
  givetvis också till att formulera målsättningen med arbetet så att det
  blir intressant ur utbildningssynpunkt, men även ur företagets
  synvinkel. Att handleda examensarbetet kostar en hel del resurser för
  företaget så det ligger i deras intresse att de får ut ett resultat
  som de är intresserade av. Den externa handledaren har också en viktig
  uppgift i att hjälpa examensarbetaren att sätta sig in i hur företaget
  fungerar, hjälpa till att identifiera och ta fram företagsinterna
  dokument, samt knyta kontakter med ytterligare personer på företaget.
  Dessutom kan den externa handledaren hjälpa till att ordna praktiska
  saker som tillgång till en arbetsplats och en dator.
\item
  \emph{Uppdragsgivare:} Detta kan t ex vara en person på ett företag
  eller en myndighet som har en idé om ett arbete som kan utföras och
  som är ansvarig för att det finns resurser att handleda och ta hand om
  examensarbetaren. Detta kan vara samma person som den externa
  handledaren, men det behöver inte vara det. Uppdragsgivaren kan också
  vara en person på lärosätet, t~ex en deltagare i ett
  forskningsprojekt.
\item
  \emph{Examinator:} Det är examinatorn som avgör om examensarbetet är
  godkänt eller inte. Det är givetvis ingen bra idé att examinatorn
  ställs inför ett färdigt arbete som han eller hon aldrig har sett
  innan. Examinatorn måste vara med tidigt i arbetet för att ha
  synpunkter på vad som måste ingå i målsättningen för att det ska kunna
  bli godkänt.
\end{itemize}

Exakt vilka roller som är definierade är lite olika på olika
utbildningar, men i denna bok kommer de rollbeteckningar som använts
ovan att användas. Det kan också finnas olika krav på olika fakulteter.
I vissa fall kan en och samma person vara både handledare och
examinator, men ofta är inte det tillåtet.

\section{Bokens upplägg }\label{bokens-uppluxe4gg}

Tanken med denna bok är att den ska kunna användas som en handbok av dem
som utför sitt examensarbete. Det ingår teori om t ex forskningsmetodik,
vilken också omsätts i konkreta råd om vad man som student ska göra i
olika skeden av arbetet.

Boken är i första hand skriven utifrån hur examensarbeten inom
tillämpade vetenskapsområden utförs. Det betyder att stycken om
fallstudier, råd vid intervjuer etc får ett förhållandevis stort
utrymme. Eftersom reglerna för examensarbeten inte är exakt de samma på
alla lärosäten så det viktigt att man som student undersöker vilka
regler som gäller där man studerar.

I de resterande kapitlen i denna bok avhandlas följande:

\begin{itemize}
\item
  Att formulera \emph{målen} med arbetet (kap 2): Hur hittar man ett
  examensarbete på ett företag? Hur hittar man en handledare på
  universitetet? Vilka krav ställer utbildningsmålen på arbetet? Hur
  bestämmer man vad som ska göras?
\item
  Att välja \emph{metodik} (kap 3): Vilka övergripande principer för
  examensarbete finns? När använder man dessa?
\item
  \emph{Planering och uppföljning} (kap 4): Hur bryter man ner arbetet i
  delmål? Hur vet man hur lång tid varje del tar? Vilka
  avstämningspunkter ska man ha? Vad bör ingå i en projektplan för ett
  examensarbete? Hur följer man upp arbetet så att man kommer att bli
  klar i tid? Vad ska man göra om man måste planera om?
\item
  \emph{Litteraturstudie} (kap 5): Hur söker man litteratur? Vilka
  databaser finns tillgängliga? Hur mycket måste man läsa? Hur kan man
  bedöma litteraturens kvalitet?
\item
  \emph{Genomförandet} (kap 6): Hur genomför man en fallstudie, ett
  experiment eller en kartläggningsstudie rent praktiskt? Hur tar man
  vara på viktig information genom att t ex föra dagbok? Hur utvärderar
  man ett lösningsförslag? Hur analyserar man data? Hur drar man
  slutsatser? Vad är skillnaden mellan en prototyp och en färdig
  produkt?
\item
  \emph{Rapporten} (kap 7): Vad ska ingå i en rapport och hur ska den
  disponeras? Hur formellt ska språket vara? När kan olika delar
  skrivas? Hur sker granskningen av rapporten? Hur ska en referenslista
  se ut?
\item
  \emph{Muntlig presentation} (kap 8): Vad ska ingå och hur kan
  presentationen disponeras? Hur ser bra presentationsmaterial ut? Hur
  gör man för att det ska ta lagom lång tid? Vad kan man göra för att
  minska nervositeten?
\item
  \emph{Opposition} (kap 9): Vilka typer av frågor ska man ställa? Hur
  bedömer man rapportens kvalitet?
\end{itemize}

Dessutom finns ett appendix med checklistor för examensarbetets
uppstart, planering, genomförande och avslutning.
