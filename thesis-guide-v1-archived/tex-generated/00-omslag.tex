\section{Omslag}\label{omslag}

\subsubsection{Titel}\label{titel}

Att genomföra examensarbete

\subsubsection{Författare}\label{fuxf6rfattare}

Martin Höst, Björn Regnell, Per Runeson

\subsubsection{Omslagsbild}\label{omslagsbild}

Vi ser gärna att ni utgår från följande bild:

Om ni vill anpassa den så att den blir layoutmässigt mer anpassad till
ett omslag så är vi öppna för det. Men vi vill gärna att detaljerna i
form av text, pilar, etc. ska finnas kvar.

\subsubsection{Text till baksida}\label{text-till-baksida}

Alla civilingenjörs- och ingenjörsutbildningar avslutas med ett
examensarbete.

Detta innebär ett helt nytt sätt att arbeta jämfört med tidigare kurser.
Från att ha varit förhållandevis styrd i tidigare kurser ska man nu
klara sig själv i mycket större utsträckning. Detta ställer stora krav
på målformulering, metodval, planering, systematiskt arbete och
uppföljning från studentens sida. Denna bok ger stöd i hela arbetet från
målformulering till muntlig presentation och skriftlig rapport.

Alla tre författarna är verksamma vid Lunds tekniska högskola i
forskargruppen SERG (Software Engineering Research Group). De har stor
handledningserfarenhet från att i mer än 10 år ha varit handledare för
examensarbeten i olika civilingenjörs- och högskoleingenjörsprogram,
samt i forskarutbildningen.

Martin Höst är civilingenjör, teknisk doktor och docent från Lunds
tekniska högskola. Han undervisar i huvudsak om ingenjörsprocessen för
utveckling av stora programvarusystem och är biträdande programledare
för civilingenjörsprogrammet i Datateknik på Lunds tekniska högskola.
Hans forskning fokuserar på kvalitetsarbete, processförbättring och
mätningar inom programvaruutveckling.

Björn Regnell är civilingenjör, teknisk doktor och docent vid Lunds
tekniska högskola. Han är studierektor i programvarusystem och
undervisar om programvaruutveckling och kravhantering. Han erhöll Lunds
Universitets pedagogiska pris 2005. Hans forskning fokuserar på
effektiva metoder för identifiering, specificering och prioritering av
krav på system och produkter med programvara i.

Per Runeson är civilingenjör, teknisk doktor och professor vid Lunds
tekniska högskola. Han leder forskargruppen SERG och undervisar om
programvaruutveckling och -testning. Hans forskning fokuserar på test
och kvalitetssäkring i utvecklingen av stora programvarusystem, samt
utveckling av forskningsmetodik.
