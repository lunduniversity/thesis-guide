\section{Att välja metodik}\label{att-vuxe4lja-metodik}

\section{Vad är metodik?}\label{vad-uxe4r-metodik}

Metodik är det grundläggande arbetssätt man väljer för sitt
examensarbete. Den sätter upp ramarna eller principerna för hur man går
till väga. Metodiken föreskriver inte i detalj vad och hur man ska göra
eller inte göra, utan är snarare en hjälp till att komma från en
övergripande målsättning, i lämpliga steg i riktning mot ökad kunskap
kring frågan.

Vilken metodik man bör välja beror på arbetets mål och karaktär. Arbetet
kan ha olika övergripande syften (anpassat från Robson 2002):

\begin{itemize}
\item
  \emph{Beskrivande} (eng. descriptive) studier har som huvudsakligt
  syfte att ta reda på och beskriva hur något fungerar eller utförs.
\item
  \emph{Utforskande} (eng. exploratory) studier syftar till att på
  djupet förstå hur något fungerar eller utförs.
\item
  \emph{Förklarande} (eng. explanatory) studier söker orsakssamband och
  förklaringar till hur något fungerar eller utförs.
\item
  \emph{Problemlösande} studier har till syfte att hitta en lösning till
  något problem som identifierats.
\end{itemize}

Vid en teknisk högskola är problemlösande studier vanligast, men ofta
finns inslag av andra syften i en sådan studie. Ett arbete kan också
bestå av flera delstudier. Till exempel kan ett problem identifieras i
en beskrivande eller utforskande delstudie, som sedan löses i en
problemlösande delstudie. Det är viktigt att ett examensarbete verkligen
angriper och försöker lösa det identifierade problemet. Det finns tyvärr
exempel på det omvända, att man har ett förslag till lösning som man
försöker hitta ett lämpligt problem till. Vidare är själva processen att
identifiera problemet en viktig del i att förstå det och bidrar till att
träna analysförmåga och syntes av olika kunskaper, vilket är ett
centralt mål för ett examensarbete.

För att genomföra sitt arbete bör man välja en lämplig metod, eller
kombinationer av sådana. Utifrån dessa gör man sedan en konkret plan för
sitt undersökningsarbete. För olika metoder kan man använda olika slags
''verktyg'' för t ex datainsamling och analys. Verktyg för datainsamling
kan vara enkäter, intervjuer, observationer och dokumentanalyser. Dessa
verktyg beskrivs i kapitel 6. Litteraturstudier ingår i alla
examensarbeten för att kartlägga kunskapsfronten inom området, men kan i
en del fall utgöra den huvudsakliga datainsamlingen, se kapitel 5.

Data som samlas in kan vara \emph{kvantitativ} eller \emph{kvalitativ}.
Kvantitativ data utgörs av sådant som kan räknas eller klassificeras:
antal, andel, vikt, färg etc. Kvalitativ data utgörs av ord och
beskrivningar, och är rik på detaljer och nyanser. Kvantitativ data kan
bearbetas med statistisk analys, medan kvalitativ data kräver
analysmetoder som bygger på sortering och kategorisering. För komplexa
problem, särskilt sådana som innefattar människor och deras agerande, är
en kombination av kvalitativ och kvantitativ data i många fall att
föredra.

De fyra mest relevanta metoderna för examensarbeten inom tillämpade
vetenskapsområden beskrivs i detalj i följande delkapitel:

\begin{itemize}
\item
  \protect\hypertarget{OLE_LINK2}{}{\protect\hypertarget{OLE_LINK1}{}{}}\emph{Kartläggning}
  (eng. survey): sammanställning och beskrivning av nuläget för det
  studerade objektet eller fenomenet. Ofta syftar kartläggningen till
  att beskriva en bred fråga.
\item
  \emph{Fallstudie} (eng. case study): djupgående studium av ett eller
  flera fall, där man försöker påverka det studerade objektet så lite
  som möjligt.
\item
  \emph{Experiment}: jämförande analys av två eller flera alternativ,
  där man försöker isolera ett fåtal faktorer och manipulera en av dem.
\item
  \emph{Aktionsforskning:} (eng. action research) -- en noggrant
  övervakad och dokumenterad studie av en aktivitet som syftar till att
  lösa ett problem.
\end{itemize}

En metodik kan vara av \emph{fix} eller \emph{flexibel} natur (Robson
2002). I en studie som använder en fix metodik är studien huvudsakligen
definierad innan man påbörjat genomförandet. Till exempel i en
kartläggning med hjälp av enkäter kan man inte lägga till nya frågor
eller svarsalternativ när man skickat ut enkäten till hälften av
personerna som ska tillfrågas. Då får man inte jämförbara svar från alla
tillfrågade. Kartläggningar och experiment är alltså huvudsakligen fixa.
En studie som använder flexibel metodik kan anpassas kontinuerligt efter
förändrade förutsättningar under studiens gång. Fallstudier och
aktionsforskning är huvudsakligen flexibla.

Genom att använda flera olika metoder, flera typer av data eller flera
personer som studerar ett objekt, kan man få en mer heltäckande bild av
det man studerar. Detta kallas \emph{triangulering} (Robson 2002, s
174).

\section{Kartläggning }\label{kartluxe4ggning}

Om man har som syfte att beskriva en företeelse är kartläggning en
lämplig metodik. En kartläggning är ''stickprovsmässig frågeundersökning
med i första hand beskrivande eller ofta också förklarande syfte''
(Rosengren och Arvidson 2002). Ett par exempel på kartläggningar är att
beskriva hur många som använder ett visst datorprogram, och att ta reda
på vilka problem som anses viktigast att åtgärda inom ett företag.

Vem ska man fråga i en kartläggning? Om gruppen, eller
\emph{populationen} är liten kan man naturligtvis fråga alla, men om
populationen är stor får man välja ut några representanter som man
frågar. Denna grupp kallas ett \emph{urval} (eng. sample). Utifrån
urvalets svar kan man dra slutsatser om hela gruppen.

Det finns olika sätt att göra urval från den s k urvalsramen, alltså en
förteckning av individer eller enheter som man vill välja ur. För att få
ett representativt urval måste man använda en slumpbaserad urvalsmetod.
\emph{Slumpmässigt urval} väljer med hjälp av någon slumptalsgenerering
ut en delmängd av urvalsramen. \emph{Systematiskt urval} väljer ut var
N:te individ eller enhet. \emph{Stratifierat urval} definierar först ett
antal kategorier eller \emph{strata}, och väljer sedan ut ett urval ur
respektive kategori. Om urvalet omfattar alla individer eller enheter
kallas det ett \emph{fullständigt urval}.

Även om man väljer ut ett urval ur den undersökta populationen blir det
ofta många personer som ska tillfrågas, snarare hundratals än tiotals.
Man behöver därför ett effektivt sätt att samla in och analysera data.
Enkäter -- strukturerade listor med frågor och svarsalternativ -- ger
möjlighet att samla in svar på ett likartat sätt från många olika
personer (Ejlertsson 2005). Enkäten kan vara muntlig -- den som gör
kartläggningen ställer frågor och fyller i ett svarsschema -- eller
skriftlig -- den som svarar på frågorna fyller i ett svarsschema på
papper eller via webben. Fördelar med den muntliga formen är att man kan
göra förtydliganden under utfrågningen och att risken är mindre att den
svarande avslutar i förtid. Nackdelen är att det kräver mer tid för den
som genomför undersökningen.

Ett fördefinierat svarsschema lämpar sig bäst för kvantitativ data.
Frågor av typen ''hur länge?'', ''hur ofta?'', ''hur mycket?'' kan
besvaras med en siffra eller ett intervall, medan mindre precisa frågor
är svårare att få entydiga svar på. Det finns dock två sätt att inom
enkätens ram ställa kvalitativa frågor. Dels kan man lämna fält för
kommentarer. Dels kan man ställa frågor i form av påståenden där den som
svarar graderar hur väl påståendet stämmer överens med hennes/hans
uppfattning. Man kan då t ex använda den s k Likert-skalan (Robson 2004,
s 292-308) som illustreras i tabell 3.1.

Oavsett om kartläggningen samlar in kvalitativ eller kvantitativ data är
den av typen fix design. Det går inte att i efterhand lägga till eller
formulera om frågor när kartläggningen har börjat. Därför är det viktigt
att förbereda en kartläggning väl och prova ut enkäter i en
referensgrupp, som inte ingår i undersökningen. Resultaten från
referensgruppen används för att ge synpunkter på utformning och
formuleringar, men då kan inte deras data användas i studien. Enkäten
ser ju inte likadan ut efter uppdateringen.

Tabell 3.1. Enkätfrågor med Likert-skala

\begin{longtable}[]{@{}lllllll@{}}
\toprule
Påstående & Instämmer helt & Instämmer delvis & Neutral & Instämmer inte
& Instämmer inte alls & Inte tillämpligt\tabularnewline
Jag tycker att mina studier är mycket stimulerande & & X & & &
&\tabularnewline
\ldots{} & & & & & &\tabularnewline
Mina lärare är mycket intressanta att lyssna till & X & & & &
&\tabularnewline
\bottomrule
\end{longtable}

För att analysera kvantitativ data som samlats in i en kartläggning
använder man olika slags statistiska metoder. Det är lämpligt att börja
med att presentera sina data med hjälp av deskriptiv statistik, t ex
medelvärde och varians. Det är viktigt att inte bara använda
medelvärden, utan också analysera spridningen i data. Vidare kan man
analysera korrelationen mellan hur man svarar på olika frågor. Data kan
också presenteras med grafiska metoder, till exempel box-plot eller
histogram, se vidare kapitel 6.6.1.

Om man i sin kartläggning utgår från ett antagande, en \emph{hypotes},
kan man dra slutsatser med hjälp av hypotesprövning, dvs att med
statistiska metoder avgöra om man kan förkasta en uppställd hypotes till
förmån för en alternativ hypotes. Dock är det viktigt att komma ihåg att
man inte kan dra några slutsatser om kausalitet (orsak-verkan) utan bara
om samvariation.

\section{Fallstudie }\label{fallstudie}

En studie som har till syfte att på djupet beskriva ett fenomen eller
ett objekt använder lämpligen fallstudiemetodiken. Fallstudier används
för att studera samtida fenomen, särskilt då fenomenet är svårt att
skilja ut från sin omgivning (Yin 1994). Fallstudier görs till exempel i
en organisation för att förstå hur man arbetar.

En fallstudie beskriver ett specifikt fall som man oftast väljer ut med
ett specifikt syfte, och man gör inga anspråk på att slutsatserna från
detta fall är direkt generaliserbara till andra fall. Å andra sidan, om
man har två fall som har likartade förutsättningar, är sannolikheten
naturligtvis stor att slutsatserna kommer att bli desamma i båda. Om man
gör en serie av fallstudier ökar också sannolikheten för att man har
kommit ett generellt mönster på spåren. Men vi har fortfarande inga
''bevis'' eller statistiskt säkerställda resultat, eftersom man inte har
valt ut fallen genom ett slumpmässigt urval, som i kartläggningar.

Fallstudier kan å andra sidan ge kunskaper på djupet, som kartläggningar
inte kan ge. Designen för en fallstudie är flexibel
\protect\hypertarget{OLE_LINK4}{}{\protect\hypertarget{OLE_LINK3}{}{}}--
man kan alltså förändra frågor och inriktning under studiens gång -- och
data som samlas in är huvudsakligen kvalitativ. I en kvalitativ studie
bör man försöka välja att studera eller intervjua
personer/roller/dokument som är så olika som möjligt för att hitta flest
möjliga variationer i det observerade fenomenet. Genom att variera yttre
faktorer som t ex ålder, befattning, kön och utbildningsbakgrund, får
man sannolikt också en variation i det man vill observera, t ex åsikter
kring ett fenomen.

I en fallstudie använder man sig ofta av följande tekniker för
datainsamling:

\begin{itemize}
\item
  Intervjuer
\item
  Observationer
\item
  Arkivanalys
\end{itemize}

Intervjuer som fokuseras på ett visst frågeområde kan vara
strukturerade, halv-strukturerade eller öppet riktade (Lantz 1993).
\emph{Strukturerade} intervjuer baseras på en fördefinierad frågelista
som följs exakt. Detta motsvarar i princip en muntlig enkät.
\emph{Halv-strukturerade} intervjuer har en uppsättning frågor som stöd
för intervjun, men man kan ändra ordning och formuleringar efter vad
intervjusituationen medger. \emph{Öppet riktade} intervjuer låter den
intervjuade till stor del styra vad som tas upp. Styrningen inskränker
sig till att man säkerställer genom frågor och respons att man håller
sig inom ämnesområdet för undersökningen. Data som samlas in i
intervjuer bör spelas in på ett ljudmedium och sedan transkriberas till
skriven text för senare analys. Detta är en arbetskrävande process, men
den bidrar till att man får bättre och mer tillförlitliga resultat. Som
alternativ kan man bygga på mötesanteckningar som kompletteras med
ljudinspelning. Man kan då gå tillbaka till det inspelade ljudet för att
kontrollera vad som sagts.

Observationer innebär att man studerar ett skeende och noterar vad som
sker (Robson 2002). Man kan vara en \emph{deltagande observatör}, dvs
man har en roll i skeendet som ska observeras, t ex som
projektdeltagare, och observerar samtidigt det som sker. Man kan vara en
\protect\hypertarget{OLE_LINK6}{}{\protect\hypertarget{OLE_LINK5}{}{}}\emph{fullständig
observatör} och inte delta i det som observeras, utan bara notera och
beskriva. Fördelen med en deltagande observatör är att man får en
delaktighet som skapar ett förtroende för den som studerar; nackdelen
att man riskerar att tappa distansen till studieobjektet. Den
fullständiga observatören riskerar å andra sidan att få för stor distans
till det, och inte riktigt släppas in i skeendet. Detta beskrivs mer i
detalj i kapitel 6.3.4. Data från observationer kan samlas in genom
dagboksanteckningar eller genom mer systematisk dokumentation, baserat
på kodningsscheman (Rosengren och Arvidson 2002, s 161-181; Robson 2002,
s 309-345).

Arkivanalyser består i att man går igenom dokumentation som tagits fram
för något annat syfte än den aktuella undersökningen. Till exempel kan
man gå igenom slutrapporter från projekt i ett företag, för att studera
företagets utveckling över en viss tidsperiod. Dock är det viktigt att
ta hänsyn till det ursprungliga syftet för dokumenten. Reklammaterial
för ett företag kanske inte ger samma bild som intern dokumentation i
samma företag. Data som samlas in i dokumentanalyser kan vara både
kvalitativ och kvantitativ.

Analys av kvantitativ data som samlas in i en fallstudie, utförs med
samma metoder som i t ex kartläggningsstudier. Observera dock att
resultatens generalitet är begränsad. Kvalitativ data analyseras med
helt andra metoder. Datainsamling och analys är mer integrerade för
kvalitativ data och kan ske iterativt. I grunden handlar kvalitativ
analys om att skapa struktur på det insamlade materialet. Till hjälp i
analysen kan man sätta upp matriser, där man samlar observationer under
olika kategorier för t ex olika intervjupersoner. Att dra slutsatsen att
3 av 10 svarat på ett eller annat sätt i en kvalitativ studie är
meningslöst. Däremot kan det vara intressant ifall dessa tre personer
tillhör en viss kategori. Mer information om kvalitativ analys finns i
kapitel 6.6.2.

\section{Experiment }\label{experiment}

För att kunna hitta orsakssamband och förklara vad olika fenomen beror
på, behövs mer styrd metodotid än kartläggningar och fallstudier.
Experiment är en sådan. Man kan genom experiment jämföra olika tekniska
lösningar, till exempel om bildbehandlingsalgoritm A är bättre än
algoritm B under vissa förutsättningar. I ett experiment kan man
undersöka flera parametrars inverkan på det studerade fenomenet genom
att variera och upprepa det. Man kan undersöka samma algoritmer med
avseende på hur de fungerar för t ex olika typer av bilder och för olika
tröskelvärden i bildbehandlingen. Det blir dock snabbt många
kombinationer som man skulle vilja undersöka. För att få mest kunskap ut
från en uppsättning experiment bör man tillämpa systematisk
\emph{försöksplanering} (Montgomery 2001; Bergman och Klefsjö 2002, s
83-98; kapitel 6.3.7). Om det finns några slumpmässiga inslag i
experimentet kan man med systematisk försöksplanering mäta såväl
huvudeffekt av faktorerna, som samverkan mellan dem och
konfidensintervall för de stokastiska variationerna.

Experiment kan också involvera människor och deras beteende. Då har man
två eller flera grupper som utför samma uppgift på olika sätt (Wohlin et
al 2000) och försöker ge grupperna så lika förutsättningar som möjligt,
förutom den faktor som man vill undersöka. Till exempel, i ett
experiment som jämför två metoder för att utföra samma arbetsuppgift,
försöker man få deltagarna i experimentet (eng. \emph{subjects}) så jämt
fördelade som möjligt mellan grupperna med avseende på erfarenheter och
kunskaper.

Eftersom experiment är en fix design -- man kan inte ändra något i
uppsättningen när man väl har startat experimentet -- är det viktigt att
planera sin undersökning väl. Det första steget innebär att
\emph{definiera målen} med studien. Vad ska analyseras? För vilket
syfte? Enligt vilka kriterier? Från vilket perspektiv? I vilket
sammanhang? Utifrån målen formuleras en \emph{hypotes}, ett antagande om
det som ska undersökas. Om man vill jämföra två metoder, formuleras den
så kallade \emph{nollhypotesen} att det inte finns någon skillnad mellan
dem. Denna hypotes kan man förkasta i analysen till förmån för en
\emph{alternativ hypotes}, vilket t ex kan vara att metod A tar mindre
tid än metod B. Nästa steg i planeringen är att välja ut vilka faktorer
som kan påverka det undersökta fenomenet. Dessa benämns \emph{oberoende
variabler}, se figur 3.1\emph{.} De faktorer som man vill undersöka
effekten av kallas \emph{behandling}, och de som man vill mäta
resultatet på kallas \emph{beroende variabler}.

Om experimentet utförs på försökspersoner bör dessa väljas ut med
urvalsmetoder från en population, på samma sätt som i en
kartläggningsstudie. Om inte detta är möjligt, utan man har en grupp av
tillgängliga personer, talar man om ett \emph{kvasiexperiment}, vilket
naturligtvis begränsar möjligheten att generalisera slutsatserna från
experimentet.

Utifrån hypotes, variabler (beroende och oberoende) och i förekommande
fall försökspersoner definieras experimentets \emph{design}.
Experimentdesignen bestämmer vilka variabler man vill låsa på fasta
nivåer och vilka man varierar som experimentets behandling. Om man har
försökspersoner involverade bestämmer designen också vilka som ska
använda de olika behandlingarna.

Det finns tre grundläggande principer för denna tilldelning,
randomisering, blockering och balansering. \emph{Randomisering} innebär
att man låter slumpen styra genom sampling vilka kombinationer som
testas och i vilken ordning. Detta är en viktig princip för många av de
statistiska metoder som kan användas för att analysera experimentets
resultat. \emph{Blockering} används för att styra en variabel som kan
påverka det observerade fenomenet, men vi tror att den påverkar båda
behandlingarna lika, och är inte intresserade av att undersöka effekten.
T ex kan vi anta att olika erfarenhet hos försökspersonerna leder till
att man är olika effektiv. Blockering innebär i detta fall att man
säkerställer att personer med olika erfarenhet finns i båda
experimentgrupperna. \emph{Balansering} innebär att man har lika många
entiteter i båda experimentgrupperna. Detta underlättar ofta den
statistiska analysen.

I experiment med försökspersoner som agerar är experimentets
\emph{instrument} det material som försökspersonerna arbetar med i
experimentet, såväl det som rör själva arbetsuppgiften som blanketter
för datainsamling. En särskild svårighet uppträder då man vill jämföra
en behandling med en ''icke-behandling'', s k \emph{placebo}. Inom
medicinen kan man ge sockerpiller så att alla försökspersoner uppfattar
det som att de får någon behandling. Om man vill utvärdera värdet av en
metod är det oklart hur man definierar jämförelsen med ''ingen metod''.
I dessa fall brukar man säga att man jämför den utvärderade metoden med
en ''ad hoc''-metod, dvs när arbetet utförs utan några riktlinjer alls.
När man tagit fram sina instrument är det är lämpligt att prova ut dem
på en pilotgrupp före huvudexperimentet, för att säkerställa att de
fungerar som det är tänkt.

Experimentets försökspersoner bör arbeta med sin uppgift under så
skyddade former som möjligt, för att inte ovidkommande faktorer ska
påverka experimentet. Å andra sidan är detta inte en så realistisk
arbetsmiljö. Arbete utförs normal i en omgivning som störs på olika
sätt. För att säkerställa en relevant jämförelse mellan behandlingarna,
är det dock viktigt att eventuella störningar berör alla försökspersoner
på samma sätt.

De data som samlas in i ett experiment är huvudsakligen kvantitativa.
Dock kan det finnas bedömningar i tolkningen av resultaten, t ex av vad
som är en korrekt eller felaktigt utförd uppgift. Utöver kvantitativa
data kan man i ett experiment komplettera med kvalitativa
undersökningar, t ex om hur försökspersonerna upplevde de olika
behandlingarna.

Analys av kvantitativ data från experiment sker med samma metoder som
för kartläggningar. Man använder alltså deskriptiv statistik, grafiska
metoder och hypotesprövning. Dessa metoder beskrivs i detalj i kapitel
6.6.1.

\section{Aktionsforskning }\label{aktionsforskning}

För ett arbete som har till syfte att förbättra något samtidigt som man
studerar det, kan man använda aktionsforskningsmetodik (Robson 2004, s
215-219). För examensarbete av problemlösande karaktär är denna metodik
ett värdefullt stöd. Ibland beskriver man aktionsforskning som en
variant på fallstudier, men vi väljer att beskriva den separat.

Aktionsforskning börjar med att man \emph{observerar} en situation eller
ett fenomen för att identifiera eller tydliggöra det problem som ska
lösas. För detta kan man använda kartläggnings- eller fallstudiemetoder
enligt ovan. Nästa steg är att ta fram ett förslag till \emph{lösning}
och att genomföra det. Därefter följer en viktig, men ofta försummad
del, nämligen \emph{utvärdering} av lösningen, genom att observera den i
sitt sammanhang, och att analysera och reflektera över hur det fungerat.
Detta är en iterativ process som upprepas igen, baserat på
utvärderingen. Om problemen kvarstår, eller nya har uppstått börjar man
med att tydliggöra dem, för att så småningom lösa dem.

Detta arbetssätt är besläktat med hur man arbetar med
kvalitetsförbättring (Bergman och Klefsjö 2002) eller processförbättring
(NASA/SEL 1996). Shewart-cykeln, som är en generell metod för
förbättring innehåller just de beskrivna stegen ovan (Bergman och
Klefsjö 2002, s 130):

\begin{itemize}
\item
  Planera -- identifiera problemet och dess orsaker.
\item
  Gör -- föreslå och genomför förbättringar som löser problemet
\item
  Studera -- kontrollera att de utförda åtgärderna lett till förbättring
\item
  Lär -- om åtgärderna var lyckade ska den nya lösningen permanentas
\end{itemize}

Aktionsforskningsmetodiken illustreras genom ett exempel på ett
problemlösande examensarbete i tabell 3.2. Examensarbetet genomför de
tre första delstegen, medan permanentningen av lösningen ligger utanför
ramen för det.

Aktionsforskning syftar till att påverka en situation och att observera
och utvärdera den samtidigt. Detta innebär naturligtvis problem med
oberoendet, eftersom man har svårt att vara kritiskt utvärderande till
det man varit med och utfört. Genom att ställa upp kriterier för
utvärderingen kan man åstadkomma en mer objektiv bedömning.

Tabell 3.2. Ett exempel på ett problemlösande examensarbete som syftar
till att ta fram ett datorverktyg för att administrera ett företags
orderhantering.

\begin{longtable}[]{@{}ll@{}}
\toprule
\textbf{Steg} & \textbf{Aktiviteter}\tabularnewline
Planera & Förutsättningar och behov i företagets orderhantering
kartläggs genom enkäter till kunder (kap 6.3.2), intervjuer med
orderhanterare och tillverkningschef (kap 6.3.3) samt observationer av
lagerarbetet (kap 6.3.4). Erfarenheter från andra företag studeras i
litteraturen (kap 5).\tabularnewline
Gör & Tre förslag till lösning skisseras, ett egenutvecklat system, ett
anpassat open source-system och ett inköpt. Dessa utvärderas med hjälp
av kriterier (kap 6.1.2). Det anpassade systemet väljs och en prototyp
implementeras (kap 6.4). Systemet används i ett pilotprojekt i en mindre
del av verksamheten.\tabularnewline
Studera & Erfarenheter från användningen av systemet samlas in genom
enkäter, intervjuer och observationer (kap 6.3.2-6.3.4).\tabularnewline
Lär & Företagsledningen bestämmer sig för att använda ett produktifierat
system i hela orderhanteringen. Detta arbete ligger dock utanför
examensarbetet och utförs av företagets egen IT-personal.\tabularnewline
\bottomrule
\end{longtable}

Studier med aktionsforskningsmetodik syftar ofta till att följa en
utveckling över tid. Ofta behöver man genomlöpa cykeln flera varv för
att åstadkomma den förändring som behövs. Det är viktigt att man inte
tappar distansen till det man undersöker när man arbetar med det under
en lång tid. Man bör vidta åtgärder för att minska detta hot, t ex genom
att definiera utvärderingskriterier, och att någon extern person kommer
in och granskar arbetet.

I ett tekniskt examensarbete kan lösningen ofta vara någon form av
prototyp. Denna kan användas för att utvärdera om de förslagna
åtgärderna leder till förbättring, men för att användas permanent i en
organisation behöver prototypen ''produktifieras''. Detta innebär bl a
högre krav på användarvänlighet, robusthet, effektivitet, integration
med andra system samt dokumentation. I många fall krävs en omfattande
test- eller certifieringsprocedur för att en produkt ska få användas i
kommersiell drift. Sådant arbete ligger ofta utanför ett examensarbetes
ram, och får utföras i efterhand om man vill göra prototypen
kommersiellt användbar.

\section{Giltighet}\label{giltighet}

En studie kan vara giltig i olika avseenden; att slutsatserna är väl
underbyggda, att den verkligen adresserar det fenomen man vill studera,
och att resultaten är generella. Dessa kategorier brukar benämnas
(Rosengren och Arvidson 2002):

\begin{itemize}
\item
  \emph{Reliabilitet:} tillförlitligheten i datainsamlingen och analysen
  med avseende på slumpmässiga variationer
\item
  \emph{Validitet:} att man mäter vad man avser att mäta, dvs fokus på
  systematiska problem
\item
  \emph{Representativitet:} att slutsatserna är generella
\end{itemize}

Det finns andra kategoriseringar, se t ex (Robson 2002; Wohlin et al
2000), och vissa hävdar att det behövs olika typer för olika slags
studier. För enkelhetens skull håller vi oss till dessa tre grupper.

För att åstadkomma bra \emph{reliabilitet} gäller att man är noggrann i
sin datainsamling och analys. Genom att redovisa hur man arbetat kan
läsaren göra en bedömning av hur man gått tillväga. Att låta någon
kollega granska datainsamlingen och analysen är ett sätt att få hjälp
att hitta svagheter i arbetet som kan stärkas upp. Att presentera data i
sammanställd form för intervjupersonerna är ett sätt att säkerställa att
man har uppfattat dem rätt. För kvantitativa studier är användningen av
de statistiska metoderna central i analysen. Urvalet är också en viktig
faktor för reliabiliteten, t ex att försökspersonerna har valts ut
slumpmässigt ur populationen.

\emph{Validitet} handlar om kopplingen mellan det objekt man vill
undersöka och vad man faktiskt mäter. T ex om man vill mäta personers
erfarenhet genom att mäta antalet anställningsår, bör man ta hänsyn till
vad man gjort under dessa anställningsår (Runeson och Wohlin 1998). För
att öka validiteten i en studie kan man tillämpa triangulering, dvs att
man studerar samma objekt med olika metoder.

\emph{Representativiteten} för ett resultat beror till stor del på
urvalet. En kartläggning och ett experiment kan i strikt mening bara
generaliseras till den population som urvalet är hämtat från. En faktor
som bidrar till bra representativitet är att bortfallet inte är för
stort, eller drabbar en viss kategori av försökspersoner. Fallstudier
och aktionsforskning är i princip inte generaliserbara. Å andra sidan,
om kontexten som man vill generalisera till påminner om den där studien
är genomförd, är sannolikheten större att det observerade objektet beter
sig likartat i den nya kontexten. En bra och detaljrik beskrivning av
den undersökta kontexten kan bidra till ökad representativitet.

\section{Sammanfattning}\label{sammanfattning}

Detta avsnitt presenterar fyra grundläggande forskningsmetoder, som kan
användas i studier för olika syften. Data som samlas in kan vara
kvalitativ eller kvantitativ, och en studie kan vara av fix eller
flexibel design. Tabell 3.3 sammanfattar dessa och kopplar samman det
vanligaste huvudsyftet och typ av data för respektive metod. Det finns
dock exempel på andra kombinationer av metod, syfte, data och design. En
studie som består av en kombination av flera metoder är också att
föredra framför att bara utforska en fråga från ett perspektiv.

Tabell 3.3. Sammanfattning av forskningsmetoder och dess huvudsakliga
syften, data och typ av design.

\begin{longtable}[]{@{}llll@{}}
\toprule
\textbf{Metod} & \textbf{Huvudsyfte} & \textbf{Primärdata} &
\textbf{Design}\tabularnewline
Kartläggning & Beskrivande & Kvantitativ & Fix\tabularnewline
Fallstudie & Utforskande & Kvalitativ & Flexibel\tabularnewline
Experiment & Förklarande & Kvantitativ & Fix\tabularnewline
Aktionsforskning & Problemlösande & Kvalitativ & Flexibel\tabularnewline
\bottomrule
\end{longtable}
