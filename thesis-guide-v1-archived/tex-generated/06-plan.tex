\section{Planering och uppföljning}\label{planering-och-uppfuxf6ljning}

\section{Projektarbete}\label{projektarbete}

I detta delkapitel beskrivs kortfattat allmänna begrepp för
projektarbete. Ett projekt kan definieras som \emph{en temporär satsning
för att framställa en unik produkt, tjänst eller annat resultat}, se t
ex Project Management Institute (2004). Ett examensarbete är ett typiskt
projekt och det är därför lämpligt att gå in lite mer i detalj på vad de
olika delarna av definitionen betyder.

Att projektet är en \emph{temporär} satsning innebär att det är ett
arbete som har en tydlig starttid, en begränsad varaktighet och därmed
en tydlig sluttid. Det är alltså inte ett arbete som sker kontinuerligt.
Att arbetet sker under en begränsad tid innebär att det utförs av en
tillfälligt definierad grupp av personer. Ett examensarbete varar
givetvis under en begränsad tid och den projektgrupp som är tillsatt är
just examensarbetarna och övriga roller runt om dem.

Att ett projekt ska resultera i en unik produkt, tjänst eller ett annat
resultat innebär att projektet resulterar i något som inte finns
tidigare. Det innebär att ett projekt är \emph{målorienterat} och att
det är viktigt med målformulering som resulterar i tydliga mål för ett
projekt. Eftersom man i ett projekt har begränsade resurser är det
viktigt att de mål som formuleras för ett projekt är rimliga. Att
formulera orimliga mål leder bara till problem senare i projektet.

För att lyckas i ett projekt krävs det alltså metoder för
projektplanering och projektuppföljning. Projektplanering innebär att
man tar fram en plan för hur arbetet ska utföras i projektet.

I alla projekt är det svårt att veta hur långt man egentligen har kommit
och ifall den tidplan man tagit fram verkligen kommer att hålla. Detta
kan bero på att \emph{insynen} i ett projekt inte är stor nog. Man kan t
ex inte veta att man kommit halvvägs bara för att t ex halva den
planerade tiden gått. Ett sätt att öka insynen i ett projekt är att
definiera tydliga \emph{milstolpar} vid viktiga händelser i projektet.
Dessa ska definieras så att man entydigt kan fastställa huruvida de är
uppnådda eller inte under ett projekts gång. Huruvida en milstolpe är
uppnådd eller inte bestäms av alla berörda parter, t~ex på ett
granskningsmöte enligt kapitel 1.4.1 om projektuppföljning.

Ett exempel på en milstolpe i ett examensarbete är att projektplanen är
färdig. När denna milstolpe är uppnådd vet både examensarbetarna och
handledaren hur arbetet ska ske och att de efterföljande faserna kan ta
vid. Milstolparna bör ingå i den tidplan man gjort, vilket innebär att
de bör listas tillsamman med aktiviteterna och markeras i grafiska
beskrivningar.

Milstolpar kan antingen vara generella till sin natur och därmed samma i
många olika projekt eller specifika och därmed definierade för just ett
projekt. För examensarbeten kan man t ex tänka sig att en generell
milstolpe är nådd när målformuleringen är färdig. Målen ska ju
definieras i alla examensarbeten. Ett exempel på en specifik milstolpe
kan vara när första prototypen för ett utvecklat datorprogram är
utvärderad. Denna milstolpe är inte generell för alla examensarbeten.

Särskilt under genomförandet kan det vara lämpligt att definiera
ytterligare milstolpar som t ex när litteraturstudien är färdig, när
första versionen av en prototyp är färdig, när intervjuer är utförda
etc. Hur detta ska göras är dock olika beroende på vad examensarbetet
går ut på, vilket innebär att detta måste bestämmas i detalj för varje
examensarbete.

Ett examensarbete kan som andra projekt delas in i fyra steg enligt
följande (Eriksson och Lilliesköld, 2004):

\begin{itemize}
\item
  Uppstart
\item
  Planering
\item
  Genomförande
\item
  Avslutning
\end{itemize}

Dessa steg presenteras närmare i kapitel 4.2-4.5. Till varje steg kan
milstolpar och resultat, eller så kallade leverabler, kopplas enligt
figur 4.1. Dessa steg och milstolpar gäller för i stort sett alla
projekt av denna storlek. Man har alltid en uppstartsfas som går ut på
att ta fram de övergripande målen, man måste alltid göra en plan baserat
på målen och de resurser som finns tillgängliga, man måste alltid utföra
själva arbetet och man måste alltid avsluta och rapportera projektet.

I ett projekt finns det en projektorganisation, dvs en uppsättning
personer som arbetar i projektet. I alla projekt finns det t ex en
projektledare som är ansvarig för planering, uppföljning, etc. I
projektet finns det också andra personer som kan vara ansvariga för
mindre delar eller utföra andra uppgifter. Runt om projektet finns det
roller som ''sponsorer'' och ''uppdragsgivare'' som beviljar projektet
medel och är beställare av resultaten. Det är projektledarens roll att
rapportera till rollerna runt om projektet för att få fortsatt
förtroende och därmed kunna fortsätta med projektet.

I ett examensarbete utförs själva arbetet givetvis av examensarbetarna.
Examensarbetarna får därmed ta på sig både utförarroller och
projektledarrollen. Det är därmed examensarbetarna som ''driver''
projektet, dvs planerar det, utför det, följer upp det, rapporterar hur
det går, eventuellt förhandlar om målen och planerar om, etc. Givetvis
följer även handledarna upp att arbetet går framåt, men det huvudsakliga
ansvaret att t ex identifiera och lösa en eventuell försening ligger på
examensarbetarna.

\section{Uppstart}\label{uppstart}

Första delen av examensarbetet går ut på att definiera målen med
examensarbetet. Det är viktigt att målen är realistiska under de
förutsättningar som finns för arbetet. Det finns en begränsad arbetstid
per examensarbetare så arbetet måste kunna utföras under denna tid. På
samma sätt måste övriga intressenter under detta steg tycka att målen är
intressanta och att den tid de investerar i arbetet är rimlig.

Detta steg resulterar i en målformulering enligt kapitel 2. Milstolpen
är godkänd när alla berörda parter, dvs examensarbetare, examinator,
handledare och uppdragsgivare, godkänner målformuleringen.

\section{Planering}\label{planering}

Under planeringssteget planeras arbetet mer i detalj. Det innebär att
aktiviteterna identifieras och att en tidplan tas fram för att beskriva
när varje aktivitet ska startas upp och avslutas. Planeringssteget
resulterar i en projektplan för hela examensarbetet och består av
följande aktiviteter:

\begin{itemize}
\item
  \emph{Aktivitetsnedbrytning}, vilket innebär att hela arbetet bryts
  ner i mindre delar som kan utföras antingen i sekvens eller
  parallellt.
\item
  \emph{Schemaläggning}, vilket innebär att man planerar vilken ordning
  de identifierade aktiviteterna ska utföras i.
\item
  \emph{Definition av milstolpar}, vilket innebär att man definierar
  avstämningspunkter som kan användas för uppföljning under projektets
  gång.

  \begin{enumerate}
  \def\labelenumi{\arabic{enumi}.}
  \item
    \protect\hypertarget{_Ref137390038}{}{}Aktivitetsnedbrytning
  \end{enumerate}
\end{itemize}

Aktivitetsnedbrytning innebär att hela arbetet bryts ner i mindre delar.
För varje aktivitet kan man först specificera i detalj vad som ska
göras, vilka tidigare resultat som måste vara färdiga för att arbetet
ska kunna påbörjas, samt vilka resultat aktiviteten kommer att resultera
i. Därefter kan man skatta hur mycket resurser arbetet kommer att kräva
i termer av arbetstid.

Som ett exempel på en aktivitet som kan definieras i ett examensarbete
kan vi studera ''sök litteratur''. För att denna aktivitet ska kunna
börja kan man välja att det ska finnas en färdig målformulering med
examensarbetet, så att det är tydligt vilken litteratur man ska söka
efter. Man kan definiera att aktiviteten ska resultera i en
litteraturlista.

Fördelen med att dela upp arbetet i mindre aktiviteter är att man får
lättare att göra tidplanering och att följa upp arbetet. Hur man delar
upp arbetet i ett projekt är dock inte självklart. Man måste t~ex beakta
hur stora aktiviteterna ska vara. Om man gör dem för stora så får man
inte så många aktiviteter och det blir svårt att tidplanera och följa
upp arbetet. Om man å andra sidan gör aktiviteterna för små så får man
för många, vilket kan bli för byråkratiskt och ohållbart.

Inte heller är det självklart hur mycket arbetstid som krävs för en
aktivitet och hur lång tid man är beredd att lägga på aktiviteten. Man
får helt enkelt lita till sin erfarenhet när man skattar detta och man
kan försöka tänka ut så konkret som möjligt vad det egentligen är som
ska göras under aktiviteten. För litteratursökningen kan man tänka ut
konkreta moment som att läsa litteraturlistor för liknande
examensarbeten, besöka universitetsbiblioteket ett antal gånger och läsa
inledningar och referenslistor i artiklar. Baserat på detta kan man
förhoppningsvis uppskatta hur lång tid arbetet kommer att ta. Om det är
för svårt att uppskatta kan man prova att dela upp aktiviteten i mindre
delar som är lättare att uppskatta. Man kan också försöka ta hjälp av
experter som har mer erfarenhet än man själv har. I ett examensarbete
kan detta t ex vara handledaren eller andra studenter som redan har
påbörjat eller genomfört sitt examensarbete.

Aktivitetsnedbrytningen resulterar i en tabell med alla aktiviteter, som
t ex i tabell 4.1.

Tabell 4.1. Exempel på aktiviteter.

\begin{longtable}[]{@{}lll@{}}
\toprule
\textbf{Aktivitet} & \textbf{Tid} & \textbf{Beror på
aktivitet}\tabularnewline
... & &\tabularnewline
A4: Målformulering & X dagar &\tabularnewline
A5: Litteratursökning & Y dagar & A4\tabularnewline
A6: Litteratursammanfattning & Z dagar & A5\tabularnewline
... & &\tabularnewline
\bottomrule
\end{longtable}

I tabellen visas endast tre aktiviteter, A4-A6. Som synes har varje
aktivitet fått ett nummer (A1, A2, A3,...) och man anger den skattade
tiden för varje aktivitet. Baserat på vilka utdata aktiviteterna ger och
vilka indata aktiviteterna kräver så har man också angett vilka andra
aktiviteter som aktiviteterna beror på. Man kan t ex se att aktivitet A5
inte kan börja förrän aktivitet A4 är avslutad.

\begin{enumerate}
\def\labelenumi{\arabic{enumi}.}
\item
  Schemaläggning
\end{enumerate}

Schemaläggning, eller tidplanering som det också kallas, innebär att man
bestämmer när i tiden de olika aktiviteterna ska påbörjas. Till sin
hjälp har man den information som framkom under aktivitetsnedbrytningen.
Ett sätt att planera är att alla aktiviteter påbörjas direkt när de
aktiviteter som aktiviteten beror på är färdiga, men ibland kan det vara
lämpligt att vänta längre med en aktivitet. Efter schemaläggningen vet
man vilket datum varje aktivitet ska börja.

Vid schemaläggningen måste man ofta bestämma vem som ska delta i varje
aktivitet. Om ett examensarbete utförs av två personer så innebär detta
förmodligen att flertalet aktiviteter utförs av just dessa två
examensarbetare om man inte väljer att dela på någon aktivitet så att
den utförs av endast en av examensarbetarna. Det finns dock ett viktigt
undantag till detta. I vissa aktiviteter krävs det att andra roller är
inblandade. Man kan t ex planera att handledaren ska läsa rapporten ett
visst datum.

Resultatet av detta arbete kan presenteras antingen som en tabell eller
grafiskt. Exempel på grafiska metoder är de aktivitetsnätverk och
Gantt-diagram som visas i figur 4.2. I dessa diagram är även datum
utsatta. De skall t ex för aktivitet A8 tolkas som att aktiviteten pågår
från och med den 5:e juni. Eftersom det är en helgdag den 6:e juni så
avslutas aktiviteten den 8:e juni, vilket medför att aktiviteten från
och med den 9:e juni inte längre pågår. De svarta stolparna anger när
aktiviteterna pågår och de streckade stolparna anger hur mycket
aktiviteterna kan bli försenade utan att hela projektet blir försenat.
De aktiviteter som inte kan bli försenade utan att hela projektet blir
försenat sägs utgöra den \emph{kritiska vägen} för projektet. Det är
givetvis möjligt att skriva ut datum även i ett aktivitetsdiagram.

Mer information om hur denna typ av diagram ritas ges t ex av
Sommerville (2007) och Dawson (2000). Det finns flera exempel på
datorverktyg som kan användas för att rita denna typ av nätverk och
diagram.

En relevant fråga att ställa sig är vad man ska göra om man i
tidplaneringen upptäcker att tiden inte räcker till och att projektet
inte kan bli klart inom den tid som man förfogar över. Man kan se över
vad man tänker göra i aktiviteterna och undersöka om det finns möjlighet
att korta ner arbetet. Det innebär kanske att man måste gå tillbaka och
definiera om målen med projektet och aktiviteterna. Projektplaneringen
är en iterativ process och det kan krävas flera iterationer innan
tidplanen är klar.

\begin{enumerate}
\def\labelenumi{\arabic{enumi}.}
\item
  Definition av milstolpar
\end{enumerate}

I samband med att tidplanen tas fram bestämmer man exakt vilka
milstolpar man ska ha. Som nämndes i kapitel 4.1 så har man dels de
generella milstolparna och de som är speciella för just det
examensarbete som planeras. Alla milstolpar måste specificeras i
projektplanen.

\begin{enumerate}
\def\labelenumi{\arabic{enumi}.}
\item
  En projektplan för ett examensarbete
\end{enumerate}

Följande delar bör ingå i en projektplan för ett examensarbete:

\begin{itemize}
\item
  \emph{Inledning:} Ge en kort beskrivning av syftet med dokumentet, dvs
  att beskriva hur ett examensarbete är planerat.
\item
  \emph{Mål:} Beskriv kortfattat målsättningen med examensarbetet. Detta
  är samma målformulering som togs fram i uppstartssteget i
  examensarbetet (se kapitel 2.2.5).
\item
  \emph{Organisation:} Beskriv och namnge alla roller i examensarbetet,
  som t ex examensarbetarna, handledare, extern handledare,
  uppdragsgivare, examinator och eventuell referensgrupp.
\item
  \emph{Resultatlista:} Beskriv alla resultat som ska produceras, som
  t~ex metodbeskrivning och litteratursammanfattning.
\item
  \emph{Aktivitetsnedbrytning:} Beskriv alla aktiviteter. Beskriv för
  varje aktivitet vilka resultat från tidigare aktiviteter som behövs,
  estimerad arbetstid, samt resultatdokument. Ta gärna fram en tabell
  över alla aktiviteter såsom det beskrivs i kapitel 1.3.1. Beskriv även
  de milstolpar som definierats.
\item
  \emph{Tidplan:} Beskriv tidplanen med startdatum och avslutningsdatum
  för varje aktivitet, vem som är ansvarig för varje aktivitet, samt
  planerade datum för milstolpar.
\item
  \emph{Uppföljning:} Beskriv hur uppföljning ska ske i examensarbetet.
  Här kan man t ex ta upp regelbundna avstämningar av examensarbetarna,
  handledningsmöten och genomläsningar av material. Ange vem som är
  ansvarig för vad, samt hur eventuella problem som identifieras ska
  behandlas.
\end{itemize}

Till dessa delar kan man lägga ytterligare delar om man så önskar. Man
kan t ex göra en riskanalys där man identifierar tänkbara projektrisker,
samt förslag på åtgärder som kan tas till för att reducera dem eller i
händelse att de blir verklighet.

Tidplanen ska godkännas av samtliga roller.

\section{Genomförande}\label{genomfuxf6rande}

Genomförandet är den största delen av examensarbetet och det är här
själva arbetet utförs. Hur arbetet utförs beror på vilken typ av arbete
det är och det är därför inte möjligt att i en bok som denna ange i
detalj hur det går till, eller exakt vilka milstolpar som ska finnas
under detta steg. Dessa måste definieras för varje examensarbete.

Genomförandesteget avslutas när allt tekniskt arbete är färdigt och det
bara är rapportering och presentation kvar. Om ett arbete t ex går ut på
att ta fram en prototyp för mätdatainsamling så kan genomförandefasen
avslutas när man inte längre ska bygga vidare på prototypen och alla
tester och utvärderingar av den är färdiga.

\begin{enumerate}
\def\labelenumi{\arabic{enumi}.}
\item
  \protect\hypertarget{_Ref148099069}{}{}Projektuppföljning
\end{enumerate}

Under genomförandesteget utförs arbetet och man måste följa upp att
arbetet verkligen sker enligt den plan man gjort. Det viktigt att kunna
visa för handledare, och andra intressenter att man blir färdig i tid.
Projektplanering och uppföljning sker ungefär enligt följande
''pseudokod'' i ett projekt:

identifiera projektets begränsningar och mål;

gör tidplan;

så länge (projektet inte klart och inte avslutat i förtid) \{

arbeta enligt tidplan under lämplig tid;

följ upp att de resultat som ska ha nåtts har nåtts;

om (planen behöver justeras) \{

om (målen måste justeras) \{

justera målen tillsammans med berörda roller;

\}

justera tidplanen tillsammans med berörda roller;

\}

\}

Det är alltså möjligt att man måste göra om planen i ett projekt eller
till och med definiera om målen. Projektplanering behöver alltså ofta
göras flera gånger under ett projekt. Ifall omplanering behöver göras så
är det viktigt att det sker tillsammans med berörda parter. I ett
examensarbete ska omplaneringen ske tillsammans med handledare,
uppdragsgivare och examinator.

Det är här på sin plats att poängtera att det är viktigt att verkligen
ta tag i omplaneringen om den behövs. Ofta kan det ''gå ett tag till''
utan att man planerar om. Roller runt omkring projektet som
uppdragsgivare etc märker inte lika tydligt som projektdeltagarna att
projektet håller på att bli försenat. Det är dock i stort sett alltid
bättre att starta omplaneringen tidigt än att vänta tills senare. Dels
är personer runt om projektet ofta beredda på att omplanering ibland
behöver ske och dels hinner deras förväntningar inte växa i onödan.

Ett viktigt redskap för uppföljning i projekt är granskningar av
delresultat (eng ''inspections'') som t ex programkod, design och
avsnitt från rapporten. Granskningar har gjorts i så stor utsträckning
att man har definierat en förhållandevis generell process för dessa, se
t ex Fagan (1976). I ett examensarbete är examensarbetarna ansvariga för
granskningarna och de kan gå till enligt följande när ett resultat ska
granskas:

\begin{itemize}
\item
  \emph{Planering:} Examensarbetarna identifierar vem som ska granska,
  när granskningen ska ske, när mötet enligt nedan ska vara etc.
  Lämpliga personer att välja ut kan vara handledare, extern handledare
  och andra personer som kan ha insyn i arbetet.
\item
  \emph{Individuell granskning:} De som granskar läser materialet
  individuellt för att hitta problem, otydligheter och frågor de vill
  diskutera. Det är bra om examensarbetarna har förberett t ex en
  checklista med frågor som granskarna kan ha som stöd när de granskar.
\item
  \emph{Möte:} Alla som granskat samlas tillsammans med författarna och
  går igenom vad man hittat. Examensarbetarna kan t ex leda deltagarna
  igenom det granskade materialet och alla får efterhand säga vad de har
  för synpunkter och frågor. Någon av deltagarna utses att vara
  ordförande och någon utses att vara sekreterare som skriver ner alla
  synpunkter i ett gransknings-protokoll. Det är viktigt att man inte
  fastnar i detaljer och börjar diskutera lösningar till de problem som
  tas upp. Dessa frågor får tas efter mötet med rätt personer. På mötet
  kan man besluta om ett dokument är godkänt som det är, om det måste
  uppdateras innan man kan ta ett nytt beslut, eller om det endast är
  mindre ändringar som måste göras. Om det granskade objektet är kopplat
  till en milstolpe i projektet kan en godkänd granskning innebära att
  man fastslår att milstolpen är uppfylld.
\end{itemize}

I granskningar i industriella projekt har man ofta fler steg och flera
roller definierade, men den variant som presenteras här är förmodligen
tillräcklig i de flesta examensarbeten.

\begin{enumerate}
\def\labelenumi{\arabic{enumi}.}
\item
  Ytterligare kunskaper
\end{enumerate}

Det krävs kunskaper inom de ämnen som nämnts ovan för att bli en bra
projektledare. Dessutom krävs det kunskap om ett antal lite ''mjukare''
områden. En bra projektledare har dessutom förmåga att t ex motivera och
entusiasmera deltagarna i ett projekt när detta behövs, samt att kunna
lösa problem och konflikter när dessa uppstår. Kunskap om detta får man
ofta genom erfarenhet och det går inte att kortfattat beskriva hur man
gör i en text som denna.

Se gärna examensarbetet som en möjlighet att träna även på dessa
områden. Även om ni inte är mer än två examensarbetare så är det
fortfarande viktigt att arbetet sker på ett bra sätt och att båda två i
efterhand kan se tillbaka på arbetet på ett positivt sätt.

\section{Avslutning}\label{avslutning}

Avslutningssteget är det sista steget i examensarbetet. En viktig del av
detta steg är att sammanställs rapporten. Observera dock att arbetet med
rapporten bör ha kommit långt redan när detta steg påbörjas. Flera av de
kapitel som ingår i rapporten bör redan vara skrivna och granskade av
handledaren. Övriga uppgifter i detta steg är t ex att förbereda och
hålla en muntlig presentation, samt att opponera på någon annans arbete.

\section{Sammanfattning}\label{sammanfattning}

Viktiga delar i en projektplanering och projektledning är
aktivitetsnedbrytning, schemaläggning och projektuppföljning.
Projektplanering är ett iterativt arbete som ofta måste ske mer än en
gång under ett examensarbete. I ett examensarbete är examensarbetarna
ansvariga för detta arbete.
