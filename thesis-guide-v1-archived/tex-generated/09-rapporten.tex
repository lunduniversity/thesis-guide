\section{Rapporten }\label{rapporten}

\begin{enumerate}
\def\labelenumi{\arabic{enumi}.}
\item ~
  \section{Innehåll och disposition}\label{innehuxe5ll-och-disposition}

  \begin{enumerate}
  \def\labelenumii{\arabic{enumii}.}
  \item
    Målsättning
  \end{enumerate}
\end{enumerate}

Målet med rapporten är att någon annan ska kunna ta del av vad som
gjorts under arbetet och få reda på vilka slutsatser som dragits. För
att en läsare ska kunna använda resultatet måste det vara så väl
beskrivet att han/hon kan sätta sig in i tillräckligt många detaljer,
samt att hela arbetsgången i arbetet är så väl beskriven att man kan
bedöma resultaten. Det räcker inte att skriva att man kommit fram till
en viss slutsats utan att berätta hur denna slutsats har dragits.
Arbetsgången måste vara så tydligt beskriven att läsaren förstår på
vilka grunder man kommit fram till slutsatserna och att han/hon själv
kan skapa sig en bild av slutsatsernas giltighet.

Förutom att rapporten, som andra tekniska rapporter, syftar till att
sprida resultaten av arbetet så utgör den även en grund för examination.
Examinatorn får den största mängden information om arbetet från
rapporten. Även om examinatorn kan hålla kontakt med examensarbetare och
handledare under arbetets gång så tas själva beslutet om examination
till stor del baserat på innehållet i rapporten. Även detta betyder att
examensarbetets alla delar måste beskrivas tydligt i rapporten. Det
måste t ex vara tydligt att examensarbetarna satt sig in i relevant
litteratur, valt en lämplig metod, analyserat data korrekt, samt dragit
korrekta slutsatser.

\begin{enumerate}
\def\labelenumi{\arabic{enumi}.}
\item
  Målgruppen
\end{enumerate}

Rapporten ska presentera hela arbetet som utförts. Det ska alltså inte
vara nödvändigt för en läsare som tillhör målgruppen att läsa en stor
mängd andra dokument, böcker eller rapporter för att förstå innehållet.
Från detta förstår man att det är nödvändigt att ha målgruppen klart för
sig under skrivandet. Det går inte att uttala sig generellt om vilken
målgruppen är, eftersom detta kan skilja sig lite från examensarbete
till examensarbete. Det är därmed viktigt att komma överens med
handledaren och examinatorn vem målgruppen är. Ofta kan man säga att en
lämplig målgrupp är studenter som är i ungefär samma situation som man
själv var i precis innan examensarbetet startade. Det betyder att de är
studenter som läst ganska många kurser och de är intresserade av ämnet,
men de är inte insatta i några detaljer om företag, produkter, processer
eller liknande som behandlas i arbetet.

\begin{enumerate}
\def\labelenumi{\arabic{enumi}.}
\item
  Disposition
\end{enumerate}

Nedan presenteras ett exempel på hur man kan disponera rapporten.

\begin{itemize}
\item
  \emph{Inledande del:} I stort sett alla rapporter börjar med ett antal
  kortare delar, vilka består av:

  \begin{itemize}
  \item
    \emph{Förstasida:} Här brukar man t ex skriva ut titeln på
    examensarbetet, namnen på examensarbetarna, handledarna och
    examinatorn, datum, samt eventuellt rapportnummer.
  \item
    \emph{Kort sammanfattning} (eng. \emph{abstract}): Beskriv arbetet
    kortfattat på ungefär en halv sida. Tänk på att hela arbetet ska
    presenteras, dvs inledande beskrivning av varför arbetet är viktigt,
    vad målsättningen har varit, hur arbetet har utförts, samt vilka
    resultat man kommit fram till och vilka slutsatser man drar.
  \item
    \emph{Förord eller tillkännagivanden} (eng. \emph{acknowledgment}):
    Här kan man tacka de personer som man fått hjälp av under arbetet.
  \item
    \emph{Innehållsförteckning:} Bestäm hur många rubriknivåer som ska
    tas med i innehållsförteckningen. Ofta räcker det med två nivåer.
  \end{itemize}
\end{itemize}

Rubrikerna för kapitel i den inledande delen brukar inte numreras.

\begin{itemize}
\item
  \emph{Inledning:} I inledningen bör man gå igenom vad den
  grundläggande tanken med arbetet är och motivera varför det är
  viktigt, utan att gå in på detaljer. Om man t ex gör ett examensarbete
  där man utvärderar skattningsmetoder för projektkostnader, kan man
  berätta om hur viktigt det är med denna typ av skattningar, samt att
  det finns ett behov av att utvärdera ett antal metoder. Man brukar
  avsluta inledningen med att presentera hur resten av rapporten är
  disponerad, dvs vad som ingår i de kommande kapitlen.
\item
  \emph{Bakgrund och relaterat arbete:} I detta kapitel går man igenom
  arbete som utförts tidigare och annan information som läsaren behöver
  vara insatt i för att kunna förstå resten av rapporten. Dessutom
  placerar man det arbete som presenteras i ett sammanhang och så att
  det blir tydligt att det finns ett unikt bidrag i rapporten. Om man t
  ex gör ett examensarbete där man utvärderar skattningsmetoder för
  projektkostnader kan man börja med att på ett generellt sätt beskriva
  hur projekt utförs och sedan gå igenom vilka kostnader som brukar
  finnas i projekt för att sedan gå igenom vilka skattningsmetoder för
  denna typ av kostnader som finns i litteraturen.
\item
  \emph{Frågeställning och metodbeskrivning:} Eftersom frågeställningen
  och den valda metoden hör samman så kan det vara bra att presentera
  dessa direkt efter varandra. Frågeställningen presenteras lämpligen
  som ett antal konkreta frågor som man får svar på om man läser
  rapporten. I metodbeskrivningen presenterar man den metod enligt
  kapitel 3 man valt och man motiverar varför denna metod är lämplig.
  Det är också lämpligt att presentera de verktyg och tekniker man
  använder enligt kapitel 5.
\end{itemize}

Det är ofta lämpligt att även diskutera validiteten av resultaten i
denna del av rapporten även om man ibland gör detta i samband med
slutsatserna. Om det går att förstå validitetsdiskussionen utan att ha
tagit del av resultat, diskussion och slutsatser kan man ha den här,
annars kan det vara lämpligt att ha den i samband med t ex slutsatserna.

\begin{itemize}
\item
  \emph{Presentation av utfört arbete och resultat:} I denna del går man
  igenom alla resultat av arbetet. Om man t ex tagit fram en metod med
  ett iterativt angreppssätt presenterar man erfarenheter och mätningar
  från varje iteration, samt vad man kom fram till i varje iteration.
  Detta är en av de stora delarna av rapporten och det är i många fall
  lämpligt att dela upp denna del i flera kapitel. I fallet med iterativ
  utveckling kan man t ex ha ett kapitel eller delkapitel för varje
  iteration.
\item
  \emph{Diskussion:} I denna del kan man utifrån de frågeställningar man
  har diskutera de resultat som presenterades i den förra delen av
  rapporten. I fallet med den iterativa utvecklingen kan man t ex
  diskutera varför man dragit olika slutsatser i olika versioner och vad
  som är bra och mindre bra med den metod man utvecklat.
\item
  \emph{Slutsatser:} Utifrån de resultat man kommit fram till och den
  diskussion man fört i förra delen kan man formulera sina slutsatser.
  Detta brukar göras i ett förhållandevis kort kapitel där man svarar på
  var och en av de frågeställningar man formulerat tidigare i rapporten.
  I samband med slutsatserna är det ofta lämpligt att ge förslag på
  fortsatt arbete inom området.

  \begin{enumerate}
  \def\labelenumi{\arabic{enumi}.}
  \item ~
    \section{Skrivprocessen }\label{skrivprocessen}
  \end{enumerate}
\end{itemize}

Rapporten skrivs givetvis inte enbart i slutet av examensarbetet.
Istället bör man skriva olika delar efterhand som arbetet pågår. I
projektplanen kan man definiera när varje del av rapporten ska vara
färdig och när varje del ska granskas av t~ex handledarna. Dessa
granskningar är mycket viktiga för att i så stor utsträckning som
möjligt kunna bli klar med olika delar så tidigt som möjligt.

Ofta är det lämpligt att tänka igenom hela rapportens disposition i
samband att man formulerar målen med arbetet. Då kan man också formulera
vilken den tänkta målgruppen för rapporten är. Diskutera gärna
målgruppen med handledare och examinator i ett tidigt skede av
examensarbetet.

Rapporten kan, som nämnts ovan, skrivas i olika inkrement. När
tidplaneringen är klar är det förmodligen möjligt att lämna en första
version av ett par delar av rapporten. Vid denna tidpunkt är
målbeskrivning färdig, metodiken bestämd, samt en tidplan färdig. Detta
innebär att målbeskrivning och metodbeskrivning kan skrivas på en form
som passar i rapporten.

Litteraturstudien resulterar i en sammanfattning av den litteratur man
valt. Detta innebär att bakgrund och relaterat arbete kan skrivas och
granskas förhållandevis tidigt i arbetet. Det är förmodligen lämpligt
att formulera en milstolpe om detta efter att tidplanen är färdig och
godkänd.

En viktig aktivitet i samband med skrivandet är att få rapporten
granskad. Det är inte lämpligt att vänta med att lämna ifrån sig
rapporten till handledare och examinatorn först då man tycker att den är
helt färdig. Då finns en uppenbar risk att kommentarerna är ganska
allvarliga och innebär att stora ändringar måste göras. Det är bättre
att försöka få delar granskade efter hand. Först kan man t ex få
kommentarer på dispositionen. Efter det kan man få kommentarer efter
hand som kapitel blir klara. Det betyder att inledande kapitel kan
granskas förhållandevis tidigt i examensarbetet och sedan mer eller
mindre läggas åt sidan om det inte dyker upp saker som bör läggas till
eller ändras i dessa under arbetets gång. Även om kapitel granskas efter
hand är det viktigt att inte lämna ifrån sig kapitel innan de är
någorlunda färdiga. Det är jobbigt för en granskare att granska ett
kapitel med halvfärdiga stycken, etc. Även om det innebär en del extra
jobb för författaren så är det bättre att vänta med att lämna ifrån sig
ett kapitel eller en del av ett kapitel tills det är granskningsbart.
Tänk också på att använda de hjälpmedel som finns för stavningskontroll.
Onödiga stavfel gör bara rapporten svårläst och risken finns att de som
läser får lägga så stor energi på att kommentera dessa fel att mer
allvarliga fel missas och inte upptäcks förrän vid ett senare tillfälle.
Vi granskning av rapporten kan den metod som presenteras i kapitel 4.4.1
användas.

Även om olika delar granskas efter hand så behövs givetvis en eller ett
par granskningsomgångar av rapporten när den är helt färdig. Detta måste
man planera för då man gör tidplanen för examensarbetet.

Förutom att handledare kan granska rapporten så kan det vara lämpligt
att hitta ytterligare personer som kan hjälpa en att granska. Två
studenter som gör olika examensarbeten samtidigt kan t ex komma överens
om att granska varandras texter under arbetets gång.

\begin{enumerate}
\def\labelenumi{\arabic{enumi}.}
\item ~
  \section{Utformning }\label{utformning}

  \begin{enumerate}
  \def\labelenumii{\arabic{enumii}.}
  \item
    Språket
  \end{enumerate}
\end{enumerate}

De vanligaste språken att skriva på är engelska och svenska. De råd och
riktlinjer som ges i denna bok är samma oavsett vilket språk rapporten
skrivs på.

Rapporten är en teknisk rapport och språket bör därför vara
förhållandevis formellt. Det är därmed inte lämpligt att använda
talspråk, onödiga förstärkningar etc. Man ska givetvis inte skriva
''strömstyrkan var jättestor'' och det är ibland inte heller lämpligt
att skriva ''strömstyrkan var mycket stor''. Vad är det för skillnad på
en ''stor'' och en ''mycket stor'' strömstyrka? Det är för otydligt och
det är förmodligen bättre att skriva att strömstyrkan var ''stor'' eller
ännu hellre hur stor den var och sedan t ex konstatera att den är större
än en annan strömstyrka, att den är dödlig, eller något annat som är
relevant i rapporten.

Inom många områden är det vanligare att skriva i passiv form än i aktiv
form. Det betyder att man t ex skriver att ''en modell togs fram baserat
på mätningar'' istället för ''vi tog fram en modell baserat på
mätningar''. Vilket som är bäst kan diskuteras, men man ska ha klart för
sig att om man väljer den aktiva formen så kan den uppfattas ganska
informell av läsarna om de är vana vid den passiva formen. Ett sätt kan
vara att använda den passiva formen, men att byta till den aktiva vid
just de tillfällen när man själv tagit ett beslut och man tydligt vill
visa detta.

\begin{enumerate}
\def\labelenumi{\arabic{enumi}.}
\item
  Textens uppbyggnad
\end{enumerate}

Rapportens huvudkapitel består oftast av text, bilder och tabeller.
Texten är uppdelad i kapitel vilka har en rubrik och efterföljande text.
Varje kapitel brukar börja på en ny sida. Rubrikerna brukar numreras och
de utgör den översta rubriknivån. Varje kapitel brukar delas in i
delkapitel i olika nivåer. Man kan t ex ha den uppdelning som ses i
figur 7.1.

Själva texten delas sedan in i stycken för att det ska bli lättare att
läsa. Antingen kan man dela av stycken med en blankrad (som i denna bok)
eller med ett indrag på första raden i varje stycke. Det är viktigt att
vara konsekvent så att formen blir den samma i hela rapporten.

...

5. Resultat

5.1 Resultat från mätningar

\textless{}text\textgreater{}

5.2 Resultat från intervjuer

5.2.1 Intervjuomgång 1

\textless{}text\textgreater{}

5.2.2 Intervjuomgång 2

\textless{}text\textgreater{}

...

Figur 7.1 Rubriknivåer

I en teknisk rapport är det viktigt att man ska kunna referera till
textstycken i rapporten. Man kan t ex någonstans i rapporten vilja
referera till det som står i kapitel 5.1. Det är därför olämpligt att ha
text mellan rubrik 5 och rubrik 5.1 i exemplet ovan, eftersom det då
blir märkligt att t ex i kapitel 5.2 referera till denna text. Man
refererar då till samma rubrik som där referensen kommer ifrån. Vill man
ha text mellan rubrik 5 och rubrik 5.1 så kan man t ex skapa en ny
rubrik ''5.1 Inledning'' direkt efter rubrik 5. Samma sak gäller på alla
rubriknivåer, t ex mellan rubrik 5.2 och 5.2.1 i figur 7.1. Det ska dock
noteras att man ibland bryter mot detta i texter som inte är så
formella. I denna bok finns det t ex inledande texter i början av
kapitel som det inte finns referenser till.

Använd gärna figurer. Grafiska illustrationer, diagram, foton, etc. är
ofta en mycket viktig del av rapporten. Tänk på att varje figur ska ha
en figurtext och ett nummer, samt att varje figur ska refereras minst en
gång i texten. Man brukar ha figurtexten under figuren.

Precis som figurer så ska varje tabell ha ett nummer, samt en
tabellrubrik. Varje tabell ska refereras minst en gång i texten. Man
brukar ha tabellrubriken ovanför tabellen.

För att få ett enhetligt utseende och för att förenkla under arbetets
gång rekommenderas användningen av datorverktyg som t ex MS Word, Star
Office eller LaTeX. Det är förmodligen självklart för de flesta, men det
kan vara bra att sätta sig in i de lite mer avancerade funktionerna, som
hur man sätter ihop flera filer till en text, hur man formaterar
rubriker, hur man använder formatmallar, hur man gör korsreferenser,
referenslistor och innehållsförteckningar etc om man inte redan vet det.

En given fråga att ställa sig är hur lång rapporten ska vara. Vissa
personer skriver ganska kompakt på få sidor, medan andra skriver ganska
mycket och har svårt att begränsa sig. Det är helt omöjligt att ge en
definitiv rekommendation eftersom det givetvis är innehållet som avgör
och inte antalet sidor. Trots det kan man rekommendera att om en rapport
är över säg 150 sidor så bör man överväga att korta ner texten eller
flytta en del av detaljerna till appendix. Om rapporten är
förhållandevis tunn, säg under 40 sidor, så bör man kontrollera att man
verkligen fått med all information som behövs.

\begin{enumerate}
\def\labelenumi{\arabic{enumi}.}
\item
  Referenser
\end{enumerate}

När man i texten vill hänvisa till någon annan text gör man det med
referenser. Det är t ex i en bakgundsbeskrivning eller
littereturgenomgång viktigt att visa var fakta kommer ifrån. Det betyder
att man brukar ha ganska många referenser i inledande kapitel i en
rapport. Det är viktigt att vara tydlig med var information kommer
ifrån, samt att ange detta så fullständigt att en läsare kan hitta den
källa man haft. Nedan ges exempel på vilken information som måste ges
för ett antal vanliga källtyper (information inom parentes är inte
alltid relevanta att presentera):

\begin{itemize}
\item
  \emph{Bok:} författare, publiceringsår, titel, (upplaga,) förlag,
  (geografisk plats för publicering).
\item
  \emph{Tidskriftsartikel:} författare, publiceringsår, artikelns titel,
  tidskriftens namn, volym och nummer för tidskriftsnumret, sidnummer
  för artikeln.
\item
  \emph{Konferensartikel:} författare, publiceringsår, artikelns titel,
  konferensens namn, geografisk plats och datum för konferensen,
  sidnummer för artikeln.
\item
  \emph{Teknisk rapport:} författare, publiceringsår, rapportens titel,
  rapportnummer, utgivande institution.
\item
  \emph{Information från internet:} författare eller utgivande
  institution, publiceringsår, (titel,) datum för senaste access, URL.
\end{itemize}

Det finns många olika sätt att utforma referenserna, men de flesta
bygger på att man i texten har en pekare till en post i en referenslista
som finns i ett av de senare kapitlen. I denna bok har vi valt
Harvard-systemet, eller parentessystemet som det också kallas (Curtin
University of Technology 2006).

I texten kan man t ex skriva att ''Blom et al. (2005) beskriver
t-testet...'' eller att ''sambanden analyseras med ett t-test (Blom et
al. 2005)''. Man refererar alltså på olika sätt beroende på om man
presenterar referensen som att författaren sagt något eller om man
spränger in referensen i en mening. I referenslistan kan man sedan se
att referensen pekar på en bok av Blom med flera, skriven 2005 och
utgiven av Studentlitteratur.

Referenslistan placeras i slutet av dokumentet. I denna sorteras
informationen om varje dokument, sorterade efter första författarnas
efternamn.

I tabell 7.1 presenteras ett antal exempel på hur referenserna kan se ut
med Harvard-systemet. Exemplen ges på engelska.

Tabell 7.1. Referensexempel.

\begin{longtable}[]{@{}lll@{}}
\toprule
\textbf{Typ} & \textbf{Referens i texten} & \textbf{Beskrivning i
referenslistan}\tabularnewline
\vtop{\hbox{\strut Bok, en}\hbox{\strut författare}} & ''Specification
is one step in development (Sommerville 2007, p. 8)'', ``Sommerville
(2007) lists the steps in development.'' & Sommerville, I. 2007,
\emph{Software Engineering,} 8:th ed., Addison-Wesley.\tabularnewline
\vtop{\hbox{\strut Bok, två}\hbox{\strut författare}} & ''The sign test
(Siegel and Castellan 1988) can be used'' & Siegel, S. and Castellan,
N.J. 1988, \emph{Nonparametric Statistics for the Behavioral Sciences},
2:nd ed., McGraw-Hill.\tabularnewline
Tidskrift, många författare & (Basili et al. 1996) & Basili, V.R.,
Green, S., Laitenberger, O., Lanubile, F., Shull, F., Sørumgård, S., and
Zelkowits. M.V. 1996, The Empirical Investigation of Perspective-Based
Reading, \emph{Empirical Software Engineering}, Vol. 1, No. 2, pp.
133--164.\tabularnewline
Konferens-artikel & (Kitchenham et al. 2004) & Kitchenham, B.A., Dybå,
T., and Jørgensen, M. 2004, Evidence-Based Software Engineering,
\emph{Proceedings of the 26:th International Conference on Software
Engineering}, May 23-28, Edinburgh, Scotland, UK, pp.
273--281.\tabularnewline
\bottomrule
\end{longtable}

\begin{longtable}[]{@{}lll@{}}
\toprule
Teknisk rapport & (Gibson et al. 2006) & Gibson, D.L., Goldenson, D.R.,
Kost, K. 2006, \emph{Performance Results of CMMI-Based Process
Improvement}, Technical report CMU/SEI-2006-TR-004, ESC-TR-2006-004,
Software Engineering Institute, Carnegie Mellon University,
USA.\tabularnewline
Examens-arbete & (Innab 2006) & Innab, S. 2006, \emph{Requirements on a
Minimalist Process Modeling Tool}, M.Sc. thesis, Lund University,
Sweden, CODEN:LUTEDX(TETS-5562)/1-76/(2006) \& local 12.\tabularnewline
\bottomrule
\end{longtable}

I tabell 7.1 presenteras endast ett urval av de vanligaste typerna av
referenser. Det finns fler tänkbare källor som t ex doktorsavhandlingar
och olika sorters internetsidor, se t ex (Curtin University of
Technology 2006). Huvudregeln är dock att man ska vara konsekvent med
hur andra referenser skrivits och välja ett format som i så stor
utsträckning som möjligt följer samma struktur.

Det finns andra sätt skriva referenslistan som är vanliga. Det är t~ex
möjligt att ha en numrerad lista och om boken är nummer 5 i listan så
blir referenserna t ex ''Blom et al. {[}5{]} beskriver t-testet...''
eller ''sambanden analyseras med ett t-test {[}5{]}''. Olika förlag har
ofta olika regler för hur referenserna och referenslistan ska skrivas.
Det är också möjligt att den institution där examensarbetet utförs har
regler som måste följas. En grundregel är dock, som nämndes ovan, att
man ska vara konsekvent. Man ska t ex alltid skriva författarnamn på
samma sätt, dvs om man har med förnamnet på en författare så ska man ha
det på alla författare, om man skriver efternamnet före förnamnet för en
författare så ska man göra det för alla, etc. Även om grundformatet för
alla referenser är samma så brukar formen skilja lite för referenser
till böcker, tidskrifter, konferensartiklar, tekniska rapporter och
internetsidor, men samma huvudregel gäller här. Om årtalet kommer sist i
en tidskriftspost ska det komma sist i alla tidskriftsposter, om
tekniska rapportnumret kommer precis innan årtalet för en teknisk
rapport så ska det komma precis innan årtalet för alla tekniska
rapporter.

För internetsidor brukar man ha med det senaste datumet man besökt
sidan, eftersom det kan hända att information på dessa sidor ändras.

\section{Sammanfattning}\label{sammanfattning}

Rapporten är ett av de viktigaste resultaten av ett examensarbete. När
andra människor ska ta del av arbetet i efterhand så sker detta i
huvudsak genom rapporten. Dessutom ligger den till grund för examination
av arbetet. När man skriver rapporten är det viktigt att definiera
målgruppen och att sedan skriva den så att läsaren inte behöver söka upp
externa källor medan han/hon läser rapporten.

Rapporten kan disponeras i en inledning, en bakgrundsbeskrivning med
relaterat arbete, en del med tydligt listade frågeställningar och
metodikbeskrivning, en beskrivning av utfört arbete och resultat, samt
en diskussionsdel och en sammanfattning av slutsatser. Språket bör vara
förhållandevis formellt och referenser kan t ex skrivas enligt
Harvard-systemet.
